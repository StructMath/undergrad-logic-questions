\documentclass[12pt]{article}

\usepackage{amsthm}
\usepackage{amsmath, amsfonts, mathabx}
\usepackage{mathtools,array,booktabs,mathabx,stmaryrd}
\usepackage{graphicx}
\usepackage{natbib}
\usepackage[inline]{enumitem}
\usepackage{forest}
\usepackage{diagbox}
\usepackage{bussproofs}
\usepackage[tableaux]{prooftrees}
\EnableBpAbbreviations
\usepackage{qtree}
\usepackage{pdflscape}
\usepackage{multicol}
\usepackage[top=2cm]{geometry}
\usepackage{xepersian}

\settextfont[Scale=1.3]{IRBadr}
\linespread{1.2}

\newcommand*{\maximumNumberOfQuestions}{100}

\theoremstyle{definition}
\newtheorem{exc}[section]{سؤال}
\newcommand{\exn}[1]{\setcounter{section}{#1}\addtocounter{section}{-1}\exc}

\newtheoremstyle{named}{}{}{}{}{\bfseries}{}{.5em}{\thmnote{#3}#1}
\theoremstyle{named}
\newtheorem*{namedtheorem}{}

\usepackage{versions}
\excludeversion{ans}

\makeatletter
\providecommand*{\sledom}{%
  \mathrel{%
    \mathpalette\@sledom\models
  }%
}
\newcommand*{\@sledom}[2]{%
  \reflectbox{$\m@th#1#2$}%
}
\makeatother

\renewcommand\beginmarkversion{\\\textbf{پاسخ: }}
\renewcommand\endmarkversion{}

\forestset{line numbering = false, just sep = 0em, check with = {:}}

\begin{document}

\title{اشکال‌های پرتکرار در کوییز اول}
\author{}
\date{}

\maketitle

\vspace{-2.5cm}

\vspace{0.5cm}
در صورتی که نکات زیر برایتان واضح نیست حتماً از مسئول گروه حل تمرین خود سؤال کنید.

\exc
تعداد زیادی از دانشجویان در مشخص کردن دقیق مراحل استدلال استقرایی اشتباه کرده بودند. در حین انجام استدلال استقرایی لازم است اثبات حکم برای پایهٔ استقرا و گام استقرایی را به طور مشخص از هم جدا کنیم. همچنین در گام استقرایی لازم است دقیقاً فرض استقرایی را تعیین کنیم.


در حل سؤال اول کوییز بعضی از دانشجویان فرض کرده بودند اگر طول فرمول $\psi$
که می‌خواهیم حکم را برای آن ثابت کنیم برابر
$k$
باشد، طول اجزای سازنده‌اش
$k-1$
است که درست نیست. برخی دیگر از دانشجویان فرض کرده بودند اجزای سازندهٔ فرمول
$\psi$
حتماً یک ادات و یک یا دو فرمول اتمی هستند که درست نیست. عده‌ای دیگر هم فرض کرده بودند که فرمول‌ها تنها به یک شیوه ساخته می‌شوند در حالی که در زبان منطق گزاره‌ها (آنطور که در جزوه تعریف شده)
فرمول‌ها هم با ادات یک‌تایی نقیض و هم با ادات‌های دوتایی ساخته می‌شوند که هر دو حالت باید در گام استقرایی مورد توجه قرار بگیرد. 

\exc
یکی از پاسخ‌های رایج به این سؤال این فرمول بوده است:
$$(p\to q)\wedge(\neg(p\to q)\to r)$$
توجه کنید که این فرمول معادل با
$p\to q$
است!

\begin{namedtheorem}[نکته‌ای کلی و مهم!]
    یکی از مهم‌ترین مهارت‌هایی که قرار است در طی دورهٔ کارشناسی خود به دست بیاورید نحوهٔ درست فکر کردن دربارهٔ موضوعات ریاضی و در مرتبهٔ بعد نحوهٔ انتقال درست نتیجهٔ افکارتان به دیگران است. توجه کنید که پاسخ سؤال‌های این درس و همچنین دیگر دروس دانشگاهی ساختار پیچیده‌تری به نسبت مسائل ریاضیات دبیرستانی
    (مثل محاسبهٔ یک انتگرال یا یک مشتق یا حل یک معادلهٔ دومجهولی)
    دارند و بنابراین نوشتن گام‌های کلیدی راه‌حل با نمادهای ریاضی کافی نیست و لازم است مراحل را به زبان فارسی نیز توضیح بدهید. لطفاً حتی در صورتی که جواب سؤال‌های این کوییز را می‌دانید پاسخنامه را مطالعه کنید تا با نحوهٔ صحیح نوشتن پاسخ سؤالات آشنا شوید.
\end{namedtheorem}

\end{document}
