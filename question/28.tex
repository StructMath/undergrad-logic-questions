ثابت کنید
\begin{enumerate}[label=(\alph*)]
\item
اگر
$\phi\models\psi$
و
$\psi\models\sigma$
آنگاه
$\phi\models\sigma$.

\item
$\models\phi\to\psi$
اگرر
$\phi\models\psi$
\end{enumerate}\quad\vspace{-9mm}
\begin{ans}
\begin{enumerate}[label=(\alph*)]
\item
فرض می‌کنیم $\phi\models\psi$
و
$\psi\models\sigma$.
حال ارزیاب دلخواه $v$ را در نظر می‌گیریم و فرض می‌کنیم $\llbracket\phi\rrbracket_v=1$. مطابق فرض اول می‌دانیم که
$\llbracket\psi\rrbracket_v=1$. حال مطابق فرض دوم می‌دانیم که
$\llbracket\sigma\rrbracket_v=1$.
بنابراین برای هر $v$ دلخواه که $\llbracket\phi\rrbracket_v=1$ داریم
$\llbracket\sigma\rrbracket_v=1$.
یا به عبارت دیگر
$\phi\models\sigma$.
\item
فرض می‌کنیم $\models\phi\to\psi$. یک ارزیاب دلخواه $v$ در نظر می‌گیریم که به $\phi$ مقدار صادق نسبت بدهد. از فرض نتیجه می‌گیریم که $v$ به $\phi\to\psi$ مقدار صادق نسبت می‌دهد. حال با توجه به جدول ارزش رابط $\to$ نتیجه می‌گیریم که ارزیاب $v$ تنها در صورتی می‌تواند به هر دوی این گزاره‌ها مقدار صادق نسبت بدهد که به $\psi$ هم مقدار صادق نسبت بدهد. بنابراین ارزیاب $v$ به $\psi$ هم مقدار صادق نسبت می‌دهد. بنابراین برای هر ارزیاب دلخواه $v$ داریم که اگر $v$ به $\phi$ مقدار صادق نسبت دهد به $\psi$ هم مقدار صادق نسبت می‌دهد که معادل است با $\phi\models\psi$.

حال فرض می‌کنیم
$\phi\models\psi$
و یک ارزیاب دلخواه $v$ را در نظر می‌گیریم. ارزیاب $v$ به $\phi$ یا مقدار صادق نسبت می‌دهد یا کاذب. اگر مقدار صادق نسبت دهد مطابق فرض به $\psi$ هم مقدار صادق نسبت می‌دهد و بنابراین به مقدار $\phi\to\psi$ هم مقدار صادق نسبت می‌دهد. در حالت دیگر، یعنی در حالتی که به $\phi$ مقدار کاذب نسبت بدهد هم مطابق جدول ارزش رابط $\to$ به $\phi\to\psi$ مقدار صادق نسبت می‌دهد. بنابراین هر ارزیاب دلخواه $v$ به $\phi\to\psi$ مقدار صادق نسبت می‌دهد که معادل است با $\models\phi\to\psi$.
\end{enumerate}
\end{ans}
