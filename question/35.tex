(\emph{قضیهٔ گلیونکو})
اگر برهانی برای $\varphi$ بدون استفاده از قاعدهٔ \lr{RAA} وجود داشته باشد این فرمول در منطق گزاره‌ای شهودی قابل اثبات است و می‌نویسیم $\vdashIPC\varphi$. همچنین برای اثبات‌پذیری در منطق گزاره‌ای کلاسیک می‌نویسیم $\vdashCPC\varphi$.
\begin{enumerate}[label=(\alph*)]
        \item 
    نشان دهید اگر
    $\vdashIPC\neg\neg\varphi$
    آنگاه
    $\vdashCPC\varphi$.
    \item 
    نشان دهید اگر
    $\vdashCPC\varphi$
    آنگاه
    $\vdashIPC\neg\neg\varphi$.

    (راهنمایی: برهان‌های حاوی
    \lr{RAA}
    را به برهان‌هایی تبدیل کنید که از قاعدهٔ
    \lr{LEM}
    استفاده می‌کنند. سپس رخدادهای
    \lr{LEM}
    را با رخدادهای
    $\neg\neg(\varphi\vee\neg\varphi)$
    که در منطق گزاره‌ای شهودگرایانه قابل اثبات است جایگزین کنید و با استقرا روی تعداد رخداد قاعدهٔ
    \lr{LEM}
    حکم را ثابت کنید.)
    \item نتیجه بگیرید
    $\vdashCPC\neg\varphi$ اگرر $\vdashIPC\neg\varphi$.
\end{enumerate}\quad
\begin{ans}
    \begin{enumerate}[label=(\alph*)]
        \item فرض کنید $\vdashIPC \neg\neg\varphi$. از طرفی همهٔ قضایای منطق شهودی قضایای منطق کلاسیک نیز هستند، در نتیجه $\vdashCPC \neg\neg\varphi$. از طرف دیگر هم به کمک \lr{RAA} می‌توان نشان داد $\neg\neg\varphi \vdashCPC \varphi$:
        \begin{LTR}
            \begin{prooftree}
                \AXC{$[\neg\varphi]^1$}
                \AXC{$\neg\neg\varphi$}
                \negE
                \RAA[1]{$\varphi$}
            \end{prooftree}
        \end{LTR}
        در نتیجه $\vdashCPC \varphi$.

        \item فرض کنید $\vdashCPC \varphi$. به روشی که در پاسخ پرسش \ref{q58} توضیح داده شد، درخت برهان $\varphi$ را به برهانی بدون استفاده از \lr{RAA} و با استفاده از نمونه‌های \lr{LEM} به عنوان فرض باز تبدیل می‌کنیم. فرض کنید تعداد این فرض‌های باز به فرم $\psi_i \vee \neg\psi_i$ برابر با $n$ باشد. یعنی داریم:
        \[ \psi_1 \vee \neg\psi_1 , \dots , \psi_n \vee \neg\psi_n \vdashIPC \varphi \]
        می‌خواهیم نشان دهیم برهانی بدون فرض باز برای $\neg\neg\varphi$ وجود دارد. این حکم را با استقرا روی $n$ ثابت می‌کنیم.
        
        فرض کنید $n = 0$، یعنی $\vdashIPC \varphi$. از طرفی می‌دانیم
        \begin{equation}\label{q35:dn-intro}
            \varphi \vdashIPC \neg\neg\varphi
        \end{equation}. در نتیجه $\vdashIPC \neg\neg\varphi$. برای گام استقرا، فرض کنید $n > 0$. با استفاده از قاعدهٔ معرفی شرطی خواهیم داشت
        \[ \psi_1 \vee \neg\psi_1 , \dots , \psi_{n-1} \vee \neg\psi_{n-1} \vdashIPC (\psi_n \vee \neg\psi_n) \rightarrow \varphi \]
        از فرض استقرا داریم
        \[ \vdashIPC \neg\neg((\psi_n \vee \neg\psi_n) \rightarrow \varphi) \]
        همچنین می‌توان نشان داد که در منطق شهودی نقیض نقیض روی شرطی پخش می‌شود، یعنی
        \begin{equation}\label{q35:dn-dist} \neg\neg((\psi_n \vee \neg\psi_n) \rightarrow \varphi) \vdashIPC \neg\neg(\psi_n \vee \neg\psi_n) \rightarrow \neg\neg\varphi \end{equation}
        در نتیجه
        \[ \vdashIPC \neg\neg(\psi_n \vee \neg\psi_n) \rightarrow \neg\neg\varphi \]
        که با استفاده از قضیهٔ استنتاج یعنی
        \[ \neg\neg(\psi_n \vee \neg\psi_n) \vdashIPC \neg\neg\varphi \]
        از طرفی ادعا می‌کنیم که نقیض نقیض \lr{LEM} در منطق شهودی قابل اثبات است، یعنی
        \begin{equation}\label{q35:dn-lem}
            \vdashIPC \neg\neg(\psi_n \vee \neg\psi_n)
        \end{equation}
        و در نتیجه
        \[ \vdashIPC \neg\neg\varphi \]
        و حکم اثبات می‌شود. برای اثبات ادعاهای \ref{q35:dn-intro}، \ref{q35:dn-dist} و \ref{q35:dn-lem} به پاسخ بخش‌های \ref{q34:a}، \ref{q34:d} و \ref{q34:c} سوال \ref{q34} مراجعه کنید.

        \item فرض کنید $\vdashCPC \neg\varphi$. می‌خواهیم نشان دهیم $\vdashIPC \neg\varphi$. از بخش قبل داریم $\vdashIPC \neg\neg\neg\varphi$. برهان زیر نشان می‌دهد $\neg\neg\neg\varphi \vdashIPC \neg\varphi$.
        \begin{LTR}
            \begin{prooftree}
                \AXC{$[\varphi]^1$}
                \AXC{$[\neg\varphi]^2$}
                \negE
                \negI[2]{$\neg\neg\varphi$}
                \AXC{$\neg\neg\neg\varphi$}
                \negE
                \negI[1]{$\neg\varphi$}
            \end{prooftree}
        \end{LTR}
    \end{enumerate}
    در نتیجه $\vdashIPC \neg\varphi$.
    
    جهت عکس بدیهی است، زیرا هر قضیه از منطق شهودی قضیه‌ای از منطق کلاسیک نیز هست.
\end{ans}