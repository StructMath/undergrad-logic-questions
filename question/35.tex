(\emph{قضیهٔ گلیونکو})
اگر برهانی برای $\varphi$ بدون استفاده از قاعدهٔ \lr{RAA} وجود داشته باشد این فرمول در منطق گزاره‌ای شهودی قابل اثبات است و می‌نویسیم $\vdashIPC\varphi$. همچنین برای اثبات‌پذیری در منطق گزاره‌ای کلاسیک می‌نویسیم $\vdashCPC\varphi$.
\begin{enumerate}[label=(\alph*)]
        \item 
    نشان بدهید اگر
    $\vdashIPC\neg\neg\varphi$
    آنگاه
    $\vdashCPC\varphi$.
    \item 
    نشان بدهید اگر
    $\vdashCPC\varphi$.
    آنگاه
    $\vdashIPC\neg\neg\varphi$

    (راهنمایی: برهان‌های حاوی
    \lr{RAA}
    را به برهان‌هایی تبدیل کنید که از قاعدهٔ
    \lr{LEM}
    استفاده می‌کنند. سپس رخدادهای
    \lr{LEM}
    را با رخدادهای
    $\neg\neg(\varphi\vee\neg\neg\varphi)$
    که در منطق گزاره‌ای شهودگرایانه قابل اثبات است جایگزین کنید و با استقرا روی تعداد رخداد قاعدهٔ
    \lr{LEM}
    حکم را ثابت کنید.)
    \item نتیجه بگیرید
    $\vdashCPC\neg\varphi$ اگرر $\vdashIPC\neg\varphi$.
\end{enumerate}