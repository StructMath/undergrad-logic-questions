(\emph{قضیهٔ جانشینی})
	فرض کنید
	$\varphi_1\dashv\vdash\varphi_2$.
	ثابت کنید
	$\psi[\varphi_1/p]\dashv\vdash\psi[\varphi_2/p]$.
	
	(راهنمایی: از استقرا روی $\psi$ استفاده کنید.)
	\begin{ans}
		\\	ابتدا حالت خاص مسئله را بررسی می‌کنیم. \\ اگر در $\psi$، اتم ${p}$ ظاهر نشده است. آنگاه  داریم : \\
		$$
		\psi[\varphi_1/p] = \psi
		$$
		$$
		\psi[\varphi_2/p] = \psi
		$$
		پس حکم بدیهی است.\\
		پس فرض کنیم که در $\psi$ اتم $p$ ظاهر شده است. به استقرا روی طول فرمول گزاره را ثابت می‌کنیم. \\
		حالت پایه:
		$$ | \psi | = 1 $$
		در این حالت داریم که‌:
		$$ \psi = p $$
		در نتیجه
		$$\psi[\varphi_1/p] = \varphi_1 \dashv\vdash \varphi_2 = \psi[\varphi_2/p]$$
		با استقرا قوی فرض کنیم که حکم برای هر فرمول با طول کمتر از $k$ برقرار است. حال حکم را برای فرمول‌ دلخواه $\psi$ با طول $k$، به ازای حالت‌های مختلف ساخت $\psi$ ثابت می‌کنیم.
			
		اگر $\psi$ به صورت $\neg \sigma$ است. داریم
			$| \sigma | + 1 = | \psi |$،
			که یعنی
			$| \sigma | < k$
			و در نتیجه بنا به فرض استقرا، حکم برای $\sigma$ برقرار است، یعنی
			$\sigma[\varphi_1/p] \dashv\vdash \sigma[\varphi_2/p]$. به کمک برهان ${\sigma[\varphi_2/p] \vdash \sigma[\varphi_1/p]}$ برهان زیر را برای $\neg \sigma[\varphi_1/p] \vdash \neg \sigma[\varphi_2/p]$ می‌سازیم
			\begin{prooftree}
				\quad\LTR
				\AxiomC{$[\sigma[\varphi_2/p]]^1$}
				\noLine\UnaryInfC{$\vdots$}
				\noLine\UnaryInfC{$\sigma[\varphi_1/p]$}
				\AxiomC{$\neg \sigma[\varphi_1/p]$}
				\negE
				\negI[1]{$\neg \sigma[\varphi_2/p]$}
			\end{prooftree} 
			$$
			\Longrightarrow {\neg \sigma_1 \vdash \neg \sigma_2}	
			$$
			
			و به روش مشابه ثایت می‌شود که
			$ \neg \sigma[\varphi_2/p] \vdash \neg \sigma[\varphi_1/p]$
			و در حالت $\psi = \neg \sigma$ حکم صادق است.
			
			برای باقی حالت‌ها، یعنی $\psi = \sigma \rightarrow \theta$, $\psi = \sigma \wedge \theta$ و $\psi = \sigma \vee \theta$ برای کوتاه‌نویسی تعریف می‌کنیم
			$$\sigma_1 := \sigma[\varphi_1/p]$$
			$$\sigma_2 := \sigma[\varphi_2/p]$$
			$$\theta_1 := \theta[\varphi_1/p]$$
			$$\theta_2 := \theta[\varphi_2/p]$$

			اگر $\psi$ به صورت
			$\sigma \rightarrow \theta$
			است، پس 
			$| \sigma |, | \theta | < k$

			از فرض استقرا می‌دانیم
			$\sigma_1 \dashv\vdash \sigma_2$ و
			$\theta_1 \dashv\vdash \theta_2$.
			همچنین، بنا به تعریف جانشینی داریم
			$$
			(\sigma \rightarrow \theta) [\varphi_1/p] = \sigma_1 \rightarrow \theta_1
			$$
			$$
			(\sigma \rightarrow \theta) [\varphi_2/p] = \sigma_2 \rightarrow \theta_2
			$$
			حال به کمک برهان‌های $\sigma_2 \vdash \sigma_1$ و $\theta_1 \vdash \theta_2$ که از فرض استقرا داشتیم، درخت زیر را به عنوان برهانی برای
			$\sigma_1 \rightarrow \theta_1 \vdash \sigma2 \rightarrow \theta_2$
			می‌سازیم:
			\begin{prooftree}
				\quad\LTR
				\AxiomC{$[\sigma_2]^1$}
				\noLine\UnaryInfC{$\vdots$}
				\noLine\UnaryInfC{$\sigma_1$}
				
				\AxiomC{$\sigma_1 \rightarrow \theta_1$}
				
				\toE{$\theta_1$}
				\noLine\UnaryInfC{$\vdots$}
				\noLine\UnaryInfC{$\theta_2$}
				
				\toI[1]{$\sigma_2 \rightarrow \theta_2$}
			\end{prooftree}
			به روش مشابه می‌توان نشان داد
			$\sigma_2 \rightarrow \theta_2 \vdash \sigma_1 \rightarrow \theta_1$. پس در حالت $\psi = \sigma \rightarrow \theta$ نیز حکم صادق است.
 
			اگر $\psi$ به صورت
			$\sigma \wedge \theta$
			باشد، خواهیم داشت 
			$| \sigma | , | \theta | < k$ و در نتیجه
			از فرض استقرا خواهیم داشت
			$\sigma_1 \dashv\vdash \sigma_2$ و $\theta_1 \dashv\vdash \theta_2$
			همچنین از تعریف جانشینی می‌دانیم
			$$
			(\sigma \wedge \theta) [\varphi_1/p] = \sigma_1 \wedge \theta_1
			$$
			$$
			(\sigma \wedge \theta) [\varphi_2/p] = \sigma_2 \wedge \theta_2
			$$
			حال به کمک برهان‌های $\sigma_1 \vdash \sigma_2$ و $\theta_1 \vdash \theta_2$ که از فرض استقرا داشتیم، برهان زیر را برای $\sigma_1 \wedge \theta_1 \vdash \sigma2 \wedge \theta_2$ می‌سازیم:
			
			\begin{prooftree}
				\quad\LTR
				\AxiomC{$\sigma_1 \wedge \theta_1$}
				\wedE {$\sigma_1$}
				\noLine\UnaryInfC{$\vdots$}
				\noLine\UnaryInfC{$\sigma_2$}
				
				\AxiomC{$\sigma_1 \wedge \theta_1$}
				\wedE {$\theta_1$}
				\noLine\UnaryInfC{$\vdots$}
				\noLine\UnaryInfC{$\theta_2$}
				
				\wedI{$\sigma_2 \wedge \theta_2$}
			\end{prooftree}
			به روش مشابه می‌توان نشان داد
			$(\sigma_2 \wedge \theta_2) \vdash (\sigma_1 \wedge \theta_1)$. پس حکم صادق است.
			
			 
			فرض کنید $\psi$ به صورت
			$\sigma \vee \theta$
			باشد، که یعنی  
			$| \sigma | , | \theta | < k$، و در نتیجه از فرض استقرا داریم $\sigma_1 \dashv\vdash \sigma_2$ و $\theta_1 \dashv\vdash \theta_2$.
			همچنین از تعریف جانشینی می‌دانیم
			$$
			(\sigma \vee \theta) [\varphi_1/p] = \sigma_1 \vee \theta_1
			$$
			$$
			(\sigma \vee \theta) [\varphi_2/p] = \sigma_2 \vee \theta_2
			$$
			حال به کمک برهان‌های $\sigma_1 \vdash \sigma_2$ و $\theta_1 \vdash \theta_2$ که از فرض استقرا داشتیم، برهان زیر را برای
			$\sigma_1 \vee \theta_1 \vdash \sigma2 \vee \theta_2	
			$
			می‌سازیم:
			
			\begin{prooftree}
				\quad\LTR
				\AxiomC{$[\sigma_1]^1$}
				\noLine\UnaryInfC{$\vdots$}
				\noLine\UnaryInfC{$\sigma_2$}
				\veeI {$\sigma_2 \vee \theta_2$}
				
				\AxiomC{$[\theta_1]^1$}
				\noLine\UnaryInfC{$\vdots$}
				\noLine\UnaryInfC{$\theta_2$}
				\veeI {$\sigma_2 \vee \theta_2$}
				
				\AxiomC{$\sigma_1 \vee \theta_1$}
				
				\veeE{1}{$\sigma_2 \vee \theta_2$}
			\end{prooftree}
			
			و به روش مشابه می‌توان نشان داد
			$\sigma_2 \vee \theta_2 \vdash \sigma_1 \vee \theta_1$، و در نتیجه حکم در حالت $\psi = \sigma \vee \theta$ نیز صادق است.
		\end{ans}
