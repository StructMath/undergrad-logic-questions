(\emph{قضیهٔ جانشینی})
	فرض کنید
	$\varphi_1\dashv\vdash\varphi_2$.
	ثابت کنید
	$\psi[\varphi_1/p]\dashv\vdash\psi[\varphi_2/p]$.
	
	(راهنمایی: از استقرا روی $\psi$ استفاده کنید.)
	\begin{ans}
		\\	ابتدا حالت خاص مسئله را بررسی می‌کنیم. \\ اگر که در $\psi$، اتم ${p}$ ظاهر نشده است. آنگاه  داریم : \\
		$$
		\psi[\varphi_1/p] = \psi
		$$
		$$
		\psi[\varphi_2/p] = \psi
		$$
		پس حکم بدیهی است.\\
		پس فرض کنیم که در $\psi$ اتم $p$ ظاهر شده است. به استقرا روی طول فرمول گزاره را ثابت می‌کنیم. \\
		حالت پایه:
		$$ \mid \psi \mid = 1 $$
		در این حالت داریم که‌:
		$$ \psi = p $$
		$$\Longrightarrow {\psi[\varphi_1/p] = \varphi_1 \dashv\vdash \varphi_2 = \psi[\varphi_2/p]}$$
		با استقرا قوی فرض کنیم که حکم برای هر فرمول با طول کمتر از $k$ برقرار است. حال حکم را برای فرمول‌ دلخواه $\gamma$ با طول $k$ ثابت می‌کنیم.
		\quad\vspace{0.5 cm}
		\begin{enumerate}
			\item اگر که $\gamma$ به صورت $\neg \alpha$ است. داریم
			$$\mid \alpha \mid + 1 = \mid \gamma \mid$$
			$$\Longrightarrow {\mid \alpha \mid \le k}$$
			$$\Longrightarrow {\alpha[\varphi_1/p] \dashv\vdash \alpha[\varphi_2/p]}$$
			$$
			\alpha_1 := \alpha[\varphi_1/p]
			$$
			$$
			\alpha_2 := \alpha[\varphi_2/p]
			$$
			حال ثابت می‌کنیم که
			$$ \neg \alpha_1 \dashv\vdash \neg \alpha_2$$ 
			\begin{prooftree}
				\quad\LTR
				\AxiomC{$[\alpha_2]^2$}
				\noLine\UnaryInfC{$\vdots$}
				\noLine\UnaryInfC{$\alpha_1$}
				\AxiomC{$[\neg \alpha_1]^1$}
				\negE
				\negI[2]{$\neg \alpha_2$}
				
			\end{prooftree} 
			$$
			\Longrightarrow {\neg \alpha_1 \vdash \neg \alpha_2}	
			$$
			
			و به روش مشابه ثایت می‌شود که
			$ \neg \alpha_2 \vdash \neg \alpha_1$
			و حکم صادق است.
			\item   
			اگر که $\gamma$ به صورت
			$\alpha \rightarrow \beta$
			است، پس 
			$\mid \alpha \mid , \mid \beta \mid \le \mid \gamma\ \mid$
			پس طبق فرض استقرا می‌دانیم که :
			$$ 
			\alpha[\varphi_1/p] =: \alpha_1 \dashv\vdash \alpha_2 := \alpha[\varphi_2/p]
			$$
			$$
			\beta[\varphi_1/p] =: \beta_1 \dashv\vdash \beta_2 := 			\beta[\varphi_2/p]
			$$
			پس داریم :
			$$
			(\alpha \rightarrow \beta) [\varphi_1/p] = (\alpha_1 \rightarrow \beta_1)
			$$
			$$
			(\alpha \rightarrow \beta) [\varphi_2/p] = (\alpha_2 \rightarrow \beta_2)
			$$
			حال نشان می‌دهیم که :
			$$
			\alpha_1 \rightarrow \beta_1 \dashv\vdash \alpha2 \rightarrow \beta_2	
			$$
			
			\begin{prooftree}
				\quad\LTR
				\AxiomC{$[\alpha_2]^2$}
				\noLine\UnaryInfC{$\vdots$}
				\noLine\UnaryInfC{$\alpha_1$}
				
				\AxiomC{$[\alpha_1 \rightarrow \beta_1]^1$}
				
				\toE{$\beta_1$}
				\noLine\UnaryInfC{$\vdots$}
				\noLine\UnaryInfC{$\beta_2$}
				
				\toI[2]{$\alpha_2 \rightarrow \beta_2$}
			\end{prooftree}
			
			پس داریم که
			$(\alpha_1 \rightarrow \beta_1) \vdash (\alpha_2 \rightarrow \beta_2)$ 
			و به روش مشابه 
			$(\alpha_2 \rightarrow \beta_2) \vdash (\alpha_1 \rightarrow \beta_1)$ \\
			پس حکم صادق است.
			
			\item 
			اگر $\gamma$ به صورت
			
			$\alpha \wedge \beta$
			است، پس 
			$\mid \alpha \mid , \mid \beta \mid \le \mid \gamma\ \mid$
			پس طبق فرض استقرا می‌دانیم که :
			$$ 
			\alpha[\varphi_1/p] =: \alpha_1 \dashv\vdash \alpha_2 := \alpha[\varphi_2/p]
			$$
			$$
			\beta[\varphi_1/p] =: \beta_1 \dashv\vdash \beta_2 := \beta[\varphi_2/p]
			$$
			پس داریم :
			$$
			(\alpha \wedge \beta) [\varphi_1/p] = (\alpha_1 \wedge \beta_1)
			$$
			$$
			(\alpha \wedge \beta) [\varphi_2/p] = (\alpha_2 \wedge \beta_2)
			$$
			حال نشان می‌دهیم که :
			$$
			\alpha_1 \wedge \beta_1 \dashv\vdash \alpha2 \wedge \beta_2	
			$$
			
			\begin{prooftree}
				\quad\LTR
				\AxiomC{$[\alpha_1 \wedge \beta_1]^1$}
				\wedE {$\alpha_1$}
				\noLine\UnaryInfC{$\vdots$}
				\noLine\UnaryInfC{$\alpha_2$}
				
				\AxiomC{$[\alpha_1 \wedge \beta_1]^1$}
				\wedE {$\beta_1$}
				\noLine\UnaryInfC{$\vdots$}
				\noLine\UnaryInfC{$\beta_2$}
				
				\wedI{$\alpha_2 \wedge \beta_2$}
			\end{prooftree}
			
			پس داریم که
			$(\alpha_1 \wedge \beta_1) \vdash (\alpha_2 \wedge \beta_2)$ 
			و به روش مشابه 
			$(\alpha_2 \wedge \beta_2) \vdash (\alpha_1 \wedge \beta_1)$ \\
			پس حکم صادق است.
			
			\item 
			اگر $\gamma$ به صورت
			
			$\alpha \vee \beta$
			است، پس 
			$\mid \alpha \mid , \mid \beta \mid \le \mid \gamma\ \mid$
			پس طبق فرض استقرا می‌دانیم که :
			$$ 
			\alpha[\varphi_1/p] =: \alpha_1 \dashv\vdash \alpha_2 := \alpha[\varphi_2/p]
			$$
			$$
			\beta[\varphi_1/p] =: \beta_1 \dashv\vdash \beta_2 := \beta[\varphi_2/p]
			$$
			پس داریم :
			$$
			(\alpha \vee \beta) [\varphi_1/p] = (\alpha_1 \vee \beta_1)
			$$
			$$
			(\alpha \vee \beta) [\varphi_2/p] = (\alpha_2 \vee \beta_2)
			$$
			حال نشان می‌دهیم که :
			$$
			\alpha_1 \vee \beta_1 \dashv\vdash \alpha2 \vee \beta_2	
			$$
			
			\begin{prooftree}
				\quad\LTR
				\AxiomC{$[\alpha_1]^2$}
				\noLine\UnaryInfC{$\vdots$}
				\noLine\UnaryInfC{$\alpha_2$}
				\veeI {$\alpha_2 \vee \beta_2$}
				
				\AxiomC{$[\beta_1]^3$}
				\noLine\UnaryInfC{$\vdots$}
				\noLine\UnaryInfC{$\beta_2$}
				\veeI {$\alpha_2 \vee \beta_2$}
				
				\AxiomC{$[\alpha_1 \vee \beta_1]^1$}
				
				\veeE{2,3}{$\alpha_2 \vee \beta_2$}
			\end{prooftree}
			
			پس داریم که
			$(\alpha_1 \vee \beta_1) \vdash (\alpha_2 \vee \beta_2)$ 
			و به روش مشابه 
			$(\alpha_2 \vee \beta_2) \vdash (\alpha_1 \vee \beta_1)$ \\
			پس حکم صادق است.
			
			
			
			\end {enumerate}
		\end{ans}
