
	جانشینی‌های زیر را انجام دهید. (در صورت آزاد نبودن نام برای متغیر، می‌توانید متغیرهای پابندکننده را تغییرنام دهید.)
	\begin{enumerate}[label=(\alph*)]
		\item $(\exists x (x \leq y))[x+1/y]$
		\item $((\forall z ((x = z) \rightarrow (y = z)))[f(y)/x])[z/y]$
		\item $((\exists x R(x, y))[x/y])[y/x]$
	\end{enumerate}
	\quad\vspace{0.5cm}
	\begin{ans}
		توجه داریم که جانشینی نام $t$ برای متغیر $x$ در یک فرمول آزاد است اگر و تنها اگر هیچ وقوع آزادی از $x$ در محدوده سوری روی یکی از متغیر‌های $t$ نباشد. 
		برای مثال در فرمول $\forall z(P(x, z))$ نام $f(y,z)$ برای متغیر $x$ آزاد نیست.

		\begin{enumerate}[label=(\alph*)]
			\item 
			چون نام $x + 1$ برای متغیر $y$ در فرمول $\exists x (x \le y)$ آزاد نیست، با تغییر متغیر پابندکننده $x$ به $z$، جانشینی را در فرمول $\exists z (z \leq y)$  انجام می‌دهیم.

			$(\exists z (z \leq y))[x+1/y] = \exists z (z \le x + 1)$

			\item
			ابتدا جانشینی اوّل را انجام می‌دهیم.

			$((\forall z ((x = z) \rightarrow (y = z)))[f(y)/x])[z/y] = (\forall z ((f(y) = z) \rightarrow (y = z)))[z/y]$

			نام $z$ برای متغیر $y$ آزاد نیست. پس برای این که جانشینی دوم قابل انجام باشد، باز هم باید متغیر پابندکننده، یعنی $z$ را به یک متغیر جدید، مثلاً $w$ تغییر دهیم.

			$(\forall w ((f(y) = w) \rightarrow (y = w)))[z/y] = \forall w ((f(z) = w) \rightarrow (z = w))$

			\item
			این‌جا نیز نام $x$ برای متغیر $y$ آزاد نیست. در نتیجه ابتدا تغییر متغیر می‌دهیم.

			$((\exists z R(z, y))[x/y])[y/x] = (\exists z R(z, x))[y/x] = \exists z R(z, y)$
		\end{enumerate}
	\end{ans}
