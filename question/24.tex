نشان دهید عبارت‌های $((p_1p_2$ و $()p_1$ فرمول نیستند.


\begin{ans}
    برهان خلف: فرض کنیم که عبارت های داده شده فرمول هستند. فرض کنیم 
    $X$
    کوچکترین مجموعه ای است که تحت ویژگی های زیر بسته است:
    \begin{flushleft}
        $I) \ p_i \in X(i \in N), \bot \in X$
        \\
        $II) \ \phi,\psi \in X \Rightarrow (\phi \ o \ \psi) \in X ((o \in \{\rightarrow, \vee, \wedge, \leftrightarrow \}))$
        \\
        $III) \ \phi \in X \Rightarrow (\neg \phi \in X)$
        
    \end{flushleft}
    و همچین فرض کنیم ((طبق برهان خلف)) که 
    $()p_1, ((p_1p_2 \in X$
    \\
    تعریف می کنیم 
    \begin{flushleft}
        $Y = X - \{()p_1, \ ((p_1p_2\}$
    \end{flushleft}

    طبق تعریف 
    $X$
    می دانیم که به ازای هر 
    $i \in N$
    داریم 
    $p_i \in Y$
    و همچنین
    $\bot \in Y$
    .
    \\
    \\
    از طرف دیگر به ازای هر 
    $\phi,\psi \in Y$
    داریم 
    $\phi, \psi \in X$.
    \\
    طبق ویژگی 
    $II$
    برای مجموعه 
    $X$
    داریم:
    \begin{flushleft}
        $(\phi \ o \ \psi) \in Y (o \ \in \{\rightarrow, \vee, \wedge, \leftrightarrow\})$
    \end{flushleft}

    به طریق مشابه می توان نشان داد که برای هر 
    $\phi \in Y$
    داریم 
    $(\neg\phi\in Y)$

    در نتیجه ما مجموعه ای داریم که ویژکی های 
    $I,II, III$
    را دارد اما کوچکتر از 
    $X$
    است. که با فرض ابتدایی ما که می گوید
    $X$
    کوچکترین مجموعه ای است که در سه ویژگی 
    $I,II,III$
    صدق می کند. 
    پس عبارت های داده شده نمی توانند فرمول درست ساخت باشند.
\end{ans}
