به دستگاه استنتاج طبیعی گنتزن منهای قاعدهٔ
\lr{RAA}
دستگاه استنتاج طبیعی برای منطق شهودگرایانه یا
\lr{IPC}
می‌گوییم. فرض کنید به 
\lr{IPC}
قاعدهٔ زیر را اضافه کنیم:
\begin{prooftree}
    \AXC{}
    \AXC{}
    \LeftLabel{\lr{LEM}}\RightLabel{}
    \BIC{$\varphi\vee\neg\varphi$}
\end{prooftree}
و دستگاه حاصل را
\lr{IPC+LEM} بنامیم.
این قاعده بدون استفاده از هیچ مقدمه‌ای اصل طرد شق ثالث را برای هر فرمول دلخواه به ما می‌دهد. ثابت کنید
\lr{IPC+LEM}
معادل با دستگاه استنتاج طبیعی گنتزن برای منطق کلاسیک است.
(به عبارت دیگر ثابت کنید هر استنتاج در دستگاه استنتاج طبیعی گنتزن برای منطق کلاسیک را می‌توان بدون افزایش مقدمه‌ها به استنتاجی در این دستگاه تبدیل کرد و بالعکس.)
\begin{ans}
    برای تبدیل هر استنتاج در دستگاه حاصل‌شده به استنتاجی در دستگاه استنتاج طبیعی گنتزن برای منطق کلاسیک کافی است هر رخداد قاعدهٔ
    \lr{LEM}
    برای به دست آوردن هر مصداق از
    $\varphi\vee\neg\varphi$
    را با استنتاجی از
    $\varphi\vee\neg\varphi$
    جایگزین کرد:
    \LTR
    \begin{prooftree}
        \AxiomC{$[\varphi]^1$}
        \RightLabel{$\vee$\lr{I}}\UnaryInfC{$\varphi\vee\neg\varphi$}
                                        \AxiomC{$[\neg(\varphi\vee\neg\varphi)]^2$}
        \negE
        \RightLabel{$\neg$\lr{I} $(1)$}\UnaryInfC{$\neg\varphi$}
        \RightLabel{$\vee$\lr{I}}\UnaryInfC{$\varphi\vee\neg\varphi$}
                                        \AxiomC{$[\neg(\varphi\vee\neg\varphi)]^2$}
        \negE
        \RightLabel{RAA $(2)$}\UnaryInfC{$\varphi\vee\neg\varphi$}

    \end{prooftree}
    \RTL


    از طرف دیگر اگر الگوریتمی برای تبدیل هر رخداد قاعدهٔ
    RAA
    به استنتاجی ارائه کنیم که تنها از دیگر قواعد دستگاه استنتاج طبیعی گنتزن به همراه قاعدهٔ
    \lr{LEM}
    استفاده می‌کند و مقدمه‌ای نیز به استنتاج نمی‌افزاید، نشان داده‌ایم هر استنتاج در دستگاه استنتاج طبیعی گنتزن را می‌توان بدون افزایش مقدمه‌ها به استنتاجی در دستگاه حاصل‌شده تبدیل کرد. هر رخداد قاعدهٔ
    RAA
    به شکل زیر است:
    \LTR
    \begin{prooftree}
        \AxiomC{$[\neg\varphi]^1$}
        \noLine\UnaryInfC{$\mathcal{D}$}
        \noLine\UnaryInfC{$\bot$}
        \RightLabel{RAA $(1)$}\UnaryInfC{$\varphi$}
    \end{prooftree}\RTL\noindent
    و می‌توان آن را با این استنتاج جایگزین کرد:
    \LTR
    \begin{prooftree}
        \AxiomC{}
        \RightLabel{\lr{LEM}}\UnaryInfC{$\varphi\vee\neg\varphi$}

        \AxiomC{$[\varphi]^2$}

        \AxiomC{$[\neg\varphi]^2$}
        \noLine\UnaryInfC{$\mathcal{D}$}
        \noLine\UnaryInfC{$\bot$}
        \botE{$\varphi$}
        
        \RightLabel{$\vee$\lr{E} $(2)$}\TrinaryInfC{$\varphi$}
    \end{prooftree}
    \RTL\noindent
    واضح است که این تغییر هیچ مقدمه‌ای به استنتاج اضافه نمی‌کند.
\end{ans}