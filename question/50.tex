در فرمول $A$ به جای هر وقوع اتم $p$ فرمول $\neg p$ را جایگزین می‌کنیم و حاصل را $A'$ می‌نامیم. ثابت کنید $A$ توتولوژی است اگر و فقط اگر $A'$ توتولوژی است.
\begin{ans}
  می‌دانیم هر مصداقی از یک توتولوژی توتولوژی است و $A'$ مصداقی از $A$ است. بنابراین اگر $A$ توتولوژی باشد $A'$ نیز توتولوژی است. حال فرض کنید $A'$ توتولوژی باشد. در اینصورت $A'[\neg p/p]=A[\neg\neg p]$ هم توتولوژی است. اما از آنجا که $p$ و $\neg p$ هم‌ارزند، $A[\neg\neg p]$ و $A$ نیز هم‌ارزند. بنابراین از توتولوژی بودن $A[\neg\neg p]$ توتولوژی بودن $A$ نتیجه می‌شود.
\end{ans}