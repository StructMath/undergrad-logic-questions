فرض کنید $a$ یکی از نمادهای $\neg$، $\forall$، $\exists$ و یا یک نماد محمولی باشد. نشان دهید هر وقوع $a$ در یک فرمول مرتبهٔ اول در ابتدای یک زیرفرمول است.

\emph{(راهنمایی: از استقرا روی ساختار فرمول استفاده کنید و از این که رابطهٔ زیرفرمول بودن تراگذری است کمک بگیرید.)}
\begin{ans}
  فرض کنید $a \in \{\neg, \forall, \exists\} \cup \{\text{محمول‌ها}\}$ و $\varphi$ یک فرمول مرتبهٔ اول باشد. با اسقرا روی ساختار $\varphi$ حکم را اثبات می‌کنیم.

  برای پایه‌ی استقرا، یعنی وقتی $\varphi$ یک نماد محمولی باشد، تنها وقوع $a$ می‌تواند همان نمان محمولی باشد و حکم بدیهی است.

  برای گام استقرا، حالت‌های زیر را در نظر بگیرید.
  
  فرض کنید $\varphi = \psi \circ \theta$ باشد که $\circ \in \{\vee, \wedge, \rightarrow\}$. بنا به فرض استقرا حکم برای $\psi$ و $\theta$ برقرار است، یعنی هر وقوع $a$ در $\psi$ و $\theta$ در ابتدای یک زیرفرمول از $\psi$ یا $\theta$ است. از آن‌جا که تمام زیرفرمول‌های $\psi$ و $\theta$ زیرفرمول $\varphi$ نیز هستند، و همچنین هیچ وقوع دیگر از $a$ نیز در $\varphi$ وجود ندارد، پس حکم برای $\varphi$ هم برقرار است.

  فرض کنید $\varphi = \neg \psi$. اگر وقوع $a$ همان $\neg$ ابتدای $\varphi$ باشد حکم بدیهی است. وقوع‌های احتمالی $a$ در $\psi$ نیز، بنا به فرض استقرا، در ابتدای یکی از زیرفرمول‌های $\psi$ (که زیرفرمول $\varphi$ نیز هست) قرار دارند. پس در این حالت نیز حکم برای $\varphi$ برقرار است.

  در حالت آخر، فرض کنید $\varphi = \mathcal{Q} x (\psi)$ که $\mathcal{Q} \in \{\forall, \exists\}$.
  اگر وقوع $a$ همان $\mathcal{Q}$ باشد که حکم بدیهی است. اگر $a$ وقوعی در $\psi$ داشته باشد، بنا به فرض استقرا این وقوع باید در ابتدای زیرفرمولی از $\psi$ باشد، و چون این زیرفرمول، زیرفرمول $\varphi$ نیز هست، حکم برای $\varphi$ هم برقرار است.
\end{ans}
%ANS Incomplete