	
	مطابق اثبات قضیه درستی ثابت کنید که اگر 
	$$\Gamma \vdash \phi$$
	و $\Gamma$ ارضا پذیر است؛ آنگاه
	$$\Gamma \vDash \phi$$\\
	(راهنمایی از استقرا روی طول برهان استفاده کنید.)
	
	\quad\vspace{0.5 cm}
	\begin {ans}
		طبق راهنمایی از استقرا به روی طول برهان استفاده میکنیم. 
		اگر که طول برهان برابر 1 است. پس داریم 
		$$
			\infer{\phi}{[\Gamma]}
		$$
		پس داریم که
		$\phi \in \Gamma$
	پس حکم مسئله صادق است و هر مدل
	 	$\Gamma$
	  یک مدل برای
	  	$\phi$
	    است.\\
		به استقرا فرض کنیم که برای هر برهان به طول کمتر از 
		$k$
		برقرار است. حال فرض کنیم که طول برهان استفاده شده از 
		$\Gamma$
		به 
		$\phi$
		برابر 
		$k$
		است. 
		
		
		توجه داریم که در حالت کلی
		$\phi = \neg \psi$
		یا اینکه 
		$\phi = \alpha \circ \beta$
		برای
		$\circ \in \{\wedge, \vee, \to\}$
		پس تنها قواعدی که می‌توانند آخرین قاعده باشند موارد زیر است.
		
		\quad\vspace {0.5 cm}
		\begin {enumerate}
			\item اگر که آخرین قاعده استفاده شده
			$RAA$ است. پس داریم 
			
			$$
				\infer[RAA_2]{\phi}{
					\kern 1cm	
					\infer*{\bot}{
						[\Gamma]_1
						&
						[\neg \phi]_2
					}
					\kern 1cm
				}
			$$
			
			پس طول برهان از 
			$\Gamma$ و $\neg \phi$
			به $\bot$ کمتر از $k$ است.
			و چون $\bot$ مدل ندارد و چون $\Gamma$ ارضا پذیر است پس برای هر تابع ارزش مانند $\mathcal {V}$ که بدانیم
			$[\Gamma]_\mathcal{V} = 1$ 
			، چون $\mathcal {V}$ نباید مدل $\Gamma$ و $\neg \phi$ باشد پس 
			$[\neg \phi]_\mathcal{V} = 0$
			پس 
			$[\neg \phi]_\mathcal{V} = 1$
			و حکم مسئله صادق است.
			
			
			
		\item 
		 اگر که آخرین قاعده استفاده شده
		$\neg I$ است. پس داریم 
		 که $\phi = \neg \psi$ 
		 برای یک فرمول
		  $\psi$
		   پس
		$$
		\infer[\neg I]{\phi}{
			\kern 1cm	
			\infer*{\bot}{
				[\Gamma]_1
				&
				[\psi]_2
			}
			\kern 1cm
		}
		$$
		
		پس طول برهان از 
		$\Gamma$ و $\psi$
		به $\bot$ کمتر از $k$ است.
		و چون $\bot$ مدل ندارد و چون $\Gamma$ ارضا پذیر است پس برای هر تابع ارزش مانند $\mathcal {V}$ که بدانیم
		$[\Gamma]_\mathcal{V} = 1$ 
		، چون $\mathcal {V}$ نباید مدل $\Gamma$ و $\neg \psi$ باشد پس 
		$[\psi]_\mathcal{V} = 0$
		پس 
		$[\neg \psi]_\mathcal{V} = 1$
		و حکم مسئله صادق است.
		
		
		
		\item اگر که آخرین قاعده استفاده شده
		$\vee I$ است. پس داریم 
		که
		 $\phi = \alpha \vee \beta$ 
		برای 2 فرمول
		 $\alpha , \beta$\\
		  پس بدون از دست دادن کلیت فرض کنیم که
		$$
		\infer[\vee I]{\alpha \vee \beta}{
			\kern 1cm	
			\infer*{\alpha}{
				[\Gamma]_1
			}
			\kern 1cm
		}
		$$
		
		پس طول برهان از 
		$\Gamma$
		به
		 $\alpha$
		  کمتر از $k$ است.
		 چون $\Gamma$ ارضا پذیر است پس برای هر تابع ارزش مانند $\mathcal {V}$ که بدانیم
		$[\Gamma]_\mathcal{V} = 1$ 
		، طبق فرض استقرا می‌دانیم که
		$[\alpha]_\mathcal{V} = 1$
		پس 
		$[\alpha \vee \beta]_\mathcal{V} = 1$
		و حکم مسئله صادق است.
		
		
		
		\item اگر که آخرین قاعده استفاده شده
		$\wedge I$ است. پس داریم 
		که
		$\phi = \alpha \wedge \beta$ 
		برای 2 فرمول
		$\alpha , \beta$\\
		پس 
		$$
		\infer[\wedge I]{\alpha \wedge \beta}{
			\kern 1cm	
			\infer*{\alpha}{
				[\Gamma]_1
			}
			\kern 1cm
			&
			\kern 1cm	
			\infer*{\beta}{
				[\Gamma]_1
			}
			\kern 1cm
		}
		$$
		
		پس طول برهان از 
		$\Gamma$
		به
		$\alpha , \beta$ 
		کمتر از $k$ است.
		چون $\Gamma$ ارضا پذیر است پس برای هر تابع ارزش مانند $\mathcal {V}$ که بدانیم
		$[\Gamma]_\mathcal{V} = 1$ 
		، طبق فرض استقرا می‌دانیم که
		$[\alpha]_\mathcal{V} = 1$ , $[\beta]_\mathcal{V} = 1$
		پس 
		$[\alpha \wedge \beta]_\mathcal{V} = 1$
		و حکم مسئله صادق است
		
		
		\item اگر که آخرین قاعده استفاده شده
		$\vee I$ است. پس داریم 
		که
		$\phi = \alpha \vee \beta$ 
		برای 2 فرمول
		$\alpha , \beta$\\
		پس بدون از دست دادن کلیت فرض کنیم که
		$$
		\infer[\vee I]{\alpha \vee \beta}{
			\kern 1cm	
			\infer*{\alpha}{
				[\Gamma]_1
			}
			\kern 1cm
		}
		$$
		
		پس طول برهان از 
		$\Gamma$
		به
		$\alpha$
		کمتر از $k$ است.
		چون $\Gamma$ ارضا پذیر است پس برای هر تابع ارزش مانند $\mathcal {V}$ که بدانیم
		$[\Gamma]_\mathcal{V} = 1$ 
		، طبق فرض استقرا می‌دانیم که
		$[\alpha]_\mathcal{V} = 1$
		پس 
		$[\alpha \vee \beta]_\mathcal{V} = 1$
		و حکم مسئله صادق است.
		
		
		
		\item اگر که آخرین قاعده استفاده شده
		$\to I$ است. پس داریم 
		که
		$\phi = \alpha \to \beta$ 
		برای 2 فرمول
		$\alpha , \beta$\\
		پس 
		$$
		\infer[\to I_2]{\alpha \to \beta}{
			\kern 1cm	
			\infer*{\beta}{
				[\Gamma]_1
				&
				[\alpha]_2
			}
			\kern 1cm
		}
		$$
		
		پس طول برهان از 
		$\Gamma , \alpha$
		به
		$\beta$ 
		کمتر از $k$ است.
		چون $\Gamma$ ارضا پذیر است پس 
		$\Gamma , \alpha$ 
		ارضا پذیر است. (چرا؟)
		 برای هر تابع ارزش مانند
		  $\mathcal {V}$ 
		  که بدانیم
		$[\Gamma]_\mathcal{V} = 1$ و $[\alpha]_\mathcal{V} = 1$ 
		، طبق فرض استقرا می‌دانیم که
		$[\beta]_\mathcal{V} = 1$
		پس طبق قضیه استنتاج می‌دانیم
		$[\alpha \to \beta]_\mathcal{V} = 1$
		و حکم مسئله صادق است \\
		قواعد حذف نیز مانند بالا نتیجه می‌شوند
		
		
		\end {enumerate}
				
	\end {ans}
	
