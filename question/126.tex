~\marginpar[left]{\textbf{(۱۰ نمره)}}
ثابت کنید $\langle \mathbb{N}, <, 0 \rangle$ به جز خودش زیرساخت مقدماتی دیگری ندارد.
\begin{ans}
  با برهان خلف، فرض کنید چنین باشد. پس ساختاری مانند $\mathfrak{A} = \langle A, <_A, 0 \rangle$ وجود دارد که زیرساخت مقدماتی $\mathfrak{B} = \langle \mathbb{N}, <, 0 \rangle$ است و در آن $A$ زیرمجموعه‌ای سره و شامل $0$ از $\mathbb{N}$ و $<_A$ تحدید $<$ به $A$ است. پس می‌توانیم فرض کنیم $n \neq 0$ در $\mathbb{N}$ وجود دارد که در $A$ نیست. بنا به تعریف زیرساخت مقدماتی، می‌دانیم به ازای هر فرمول $\varphi(x)$ و هر $a \in A$ باید داشته باشیم
  \[ \mathfrak{B} \vDash \varphi(\bar{a}) \iff \mathfrak{A} \vDash \varphi(\bar{a}) \]
  در دو حالت مختلف مسئله را بررسی می‌کنیم. فرض کنید عضوی بزرگتر از $n$ در $A$ نباشد. برزگترین عضو کوچکتر از $n$ در $A$ را $s$ بنامید. توجه کنید $s$ وجود دارد زیرا $0 \in A$ است. پس فرمول $\exists x (\bar{s} < x)$ در $\mathfrak{B}$ صادق است امّا در $\mathfrak{A}$ خیر، که با تعریف زیرساخت مقدماتی در تناقض است. اگر عضوی برزگتر از $n$ در $A$ وجود داشته باشد، کوچکترین عضو بزرگتر از $n$ در $A$ را $r$ بنامید. حال فرمول $\exists x (\bar{s} < x \wedge x < \bar{r})$ در $\mathfrak{B}$ صادق است و در $\mathfrak{A}$ نه. در نتیجه $\mathfrak{A}$ نمی‌تواند زیرساخت مقدماتی $\mathfrak{B}$ باشد.
\end{ans}