
	$\Gamma \vdash \phi \to \psi$
	و $\Gamma$ ارضا پذیر است؛ آنگاه
	$\Gamma \vDash \phi \to \psi$\\
	(راهنمایی از استقرا روی طول برهان استفاده کنید.)
	
	\quad\vspace{0.5 cm}
	\begin {ans}
	طبق راهنمایی از استقرا به روی طول برهان استفاده میکنیم. 
	اگر که طول برهان برابر ۱ است. پس داریم 
	$$
	\infer[\to I_2]{\psi}{[\Gamma]_1 & [\phi]_2}
	$$
	
	دقت داریم که چون طول برهان ۱ است؛
	پس از قائده
	$RAA$
	استفاده نشده است.(چرا که به $\bot$ نرسیده‌ایم). تنها حالتی که باقی‌ می‌ماند این است
	$\psi \in \Gamma$
	پس حکم مسئله صادق است و هر مدل
	$\Gamma$ و $\phi$
	یک مدل برای
	$\psi$
	است.\\
	
	به استقرا فرض کنیم که برای هر برهان به طول کمتر از 
	$k$
	برقرار است. حال فرض کنیم که طول برهان استفاده شده از 
	$\Gamma$
	به 
	$\phi \to \psi$
	برابر 
	$k$
	است. 
	پس $$
	\infer[\to I_2]{\phi \to \psi}{
		\kern 1cm	
		\infer*{\psi}{
			[\Gamma]_1
			&
			[\phi]_2
		}
		\kern 1cm
	}
	$$
	
	پس طول برهان از 
	$\Gamma , \phi$
	به
	$\psi$ 
	کمتر از $k$ است.
	چون $\Gamma$ ارضا پذیر است پس 
	$\Gamma \cup \{\phi\}$ 
	ارضا پذیر است. (چرا؟)
	برای هر تابع ارزش مانند
	$\mathcal {V}$ 
	که بدانیم
	$[\Gamma]_\mathcal{V} = 1$ و $[\phi]_\mathcal{V} = 1$ 
	، طبق فرض استقرا می‌دانیم که
	$[\psi]_\mathcal{V} = 1$
	پس طبق قضیه استنتاج می‌دانیم
	$[\phi \to \psi]_\mathcal{V} = 1$
	و حکم مسئله صادق است \\
	
	
	\end {ans}
