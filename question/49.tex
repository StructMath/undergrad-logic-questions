برای هر فرمول $A$ و اتم $p$ فرض کنید
$A^*=A[\top/p]\vee A[\bot/p]$.
ثابت کنید:
\begin{enumerate}
\item
$A\models A^*$
\item
اگر
$A\models B$
و $p$ در $B$ موجود نباشد، آنگاه
$A^*\models B$.
\item
اگر
$A\models B$
آنگاه فرمولی مانند $C$ موجود است که اتم‌های آن مشترک میان $A$ و $B$ است و $A\models C$ و $C\models B$.
\end{enumerate}\quad
\begin{ans}
  \begin{enumerate}
  \item
  فرض کنید $v$ ارزیابی است که $v\models A$. یا $v(p)=T$ و یا $v(p)=F$. در حالت اول مشخص است که
  $v(A[p/\top])=T$
  و در حالت دوم مشخص است که
  $v(A[p/\top])=T$.
  بنابراین در هر دو حالت
  $v\models A^*$.

  \item
  فرض کنید $v$ ارزیابی است که $v\models A^*$. قرار می‌دهیم
  $$
  v_T(x)=
  \begin{cases}
  T & \text{if}~~x=p\\
  v(x) & \text{otherwise}
  \end{cases}
  $$

  ارزیاب $v_F$ را نیز مشابهاً تعریف می‌کنیم. از آنجا که $p$ در $B$ واقع نمی‌شود واضح است که $v(B)=v_T(B)=v_F(B)$. با توجه به اینکه فرض کرده‌ایم $v\models A$ واضح است که لااقل یکی از دو ارزیاب $v_T$ و $v_F$ باید $A$ را ارضا کند. بدون از بین رفتن عمومیت مسئله فرض می‌کنیم $v_T\models A$. در این صورت مطابق فرض $A\models B$ داریم $v_T\models B$ و بنابراین داریم $v\models B$.

  \item
  فرض کنید تعداد اتم‌های غیرمشترک میان $A$ و $B$، $n$ باشد. حکم را با استقرا روی $n$ ثابت می‌کنیم. اگر $n=1$ فرض کنید اتم غیرمشترک $p$ است و قرار دهید $C=A^*$. حکم از دو بخش قبلی نتیجه می‌شود. حال فرض کنید حکم برای $n-1$ ثابت شده است و $p$ یکی از اتم‌های غیرمشترک میان $A$ و $B$ است.. مطابق بخش‌های قبلی می‌دانیم $A\models A^*$ و $A^*\models B$. اما تعداد اتم‌های غیرمشترک میان $A$ و $B$، $n-1$ است. بنابراین فرمول $C$ای وجود دارد چنانکه $A^*\models C$ و $C\models B$. واضح است که همچنین داریم $A\models C$. بنابراین حکم برای $n$ ثابت شده است.
  \end{enumerate}
\end{ans}