برای هر فرمول $\varphi$ و اتم $p$ فرض کنید
$\varphi^*=\varphi[\top/p]\vee \varphi[\bot/p]$.
ثابت کنید:
\begin{enumerate}
\item
$\varphi\models \varphi^*$
\item
اگر
$\varphi\models \psi$
و $p$ در $\psi$ موجود نباشد، آنگاه
$\varphi^*\models \psi$.
\item
(\emph{قضیهٔ درونیابی کریگ})
اگر
$\varphi\models \psi$
آنگاه فرمولی مانند $\sigma$ موجود است که اتم‌های آن مشترک میان $\varphi$ و $\psi$ است و $\varphi\models \sigma$ و $\sigma\models \psi$.
(به فرمول $\sigma$ \emph{درونیاب} می‌گوییم.)\footnote{
  نظیر این قضیه را می‌توان برای رابطهٔ
  $\vdash$
  هم مستقیماً اثبات کرد. طبیعتاً مطابق قضایای درستی و تمامیت رابطه‌های
  $\vdash$
  و
  $\models$
  معادل‌اند و بنابراین قضیهٔ درونیابی کریگ برای هر یک از آن‌ها قضیهٔ درونیابی کریگ برای دیگری را نتیجه می‌دهد.
}

\end{enumerate}\quad\vspace{-1cm}
\begin{ans}
  \begin{enumerate}
  \item
  فرض کنید $v$ ارزیابی است که $v\models \varphi$. یا $v(p)=T$ و یا $v(p)=F$. در حالت اول مشخص است که
  $v(\varphi[p/\top])=T$
  و در حالت دوم مشخص است که
  $v(\varphi[p/\top])=T$.
  بنابراین در هر دو حالت
  $v\models \varphi^*$.

  \item
  فرض کنید $v$ ارزیابی است که $v\models \varphi^*$. قرار می‌دهیم
  $$
  v_T(x)=
  \begin{cases}
  T & \text{if}~~x=p\\
  v(x) & \text{otherwise}
  \end{cases}
  $$

  ارزیاب $v_F$ را نیز مشابهاً تعریف می‌کنیم. از آنجا که $p$ در $\psi$ واقع نمی‌شود واضح است که $v(\psi)=v_T(\psi)=v_F(\psi)$. با توجه به اینکه فرض کرده‌ایم $v\models \varphi$ واضح است که لااقل یکی از دو ارزیاب $v_T$ و $v_F$ باید $\varphi$ را ارضا کند. بدون از بین رفتن عمومیت مسئله فرض می‌کنیم $v_T\models \varphi$. در این صورت مطابق فرض $\varphi\models \psi$ داریم $v_T\models \psi$ و بنابراین داریم $v\models \psi$.

  \item
  فرض کنید تعداد اتم‌های غیرمشترک میان $\varphi$ و $\psi$، $n$ باشد. حکم را با استقرا روی $n$ ثابت می‌کنیم. اگر $n=1$ فرض کنید اتم غیرمشترک $p$ است و قرار دهید $\sigma=\varphi^*$. حکم از دو بخش قبلی نتیجه می‌شود. حال فرض کنید حکم برای $n-1$ ثابت شده است و $p$ یکی از اتم‌های غیرمشترک میان $\varphi$ و $\psi$ است.. مطابق بخش‌های قبلی می‌دانیم $\varphi\models \varphi^*$ و $\varphi^*\models \psi$. اما تعداد اتم‌های غیرمشترک میان $\varphi$ و $\psi$، $n-1$ است. بنابراین فرمول $\sigma$ای وجود دارد چنانکه $\varphi^*\models \sigma$ و $\sigma\models \psi$. واضح است که همچنین داریم $\varphi\models \sigma$. بنابراین حکم برای $n$ ثابت شده است.
  \end{enumerate}
\end{ans}