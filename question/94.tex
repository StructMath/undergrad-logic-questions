~\marginpar[left]{\textbf{(۱۵ نمره)}}
فرض کنید ماشینی داریم که ورودی آن مجموعهٔ متناهی  \textsc{\lr{Atoms}} از گزاره‌های اتمی و مجموعهٔ متناهی $\Gamma$ از فرمول‌های منطق گزاره‌ای است که تنها از اتم‌های داخل \textsc{\lr{Atoms}} استفاده می‌کند. در صورتی که ارزیابی وجود داشته باشد که تمام اعضای $\Gamma$ را ارضا کند، ماشین آن ارزیاب را به ما خروجی می‌دهد و در غیر این صورت اعلام می‌کند $\Gamma$ ارضاناپذیر است. قصد داریم به کمک این ماشین جدول سودوکوی زیر را حل کنیم. (قوانین بازی در زیر آمده است.) مجموعهٔ \textsc{\lr{Atoms}} از اتم‌ها و مجموعهٔ $\Gamma$ از گزاره‌ها را معرفی کنید که اگر به عنوان ورودی ماشین بالا استفاده شود، خروجی ماشین راه حل جدول را به ما بدهد. توضیح بدهید چگونه می‌توان از خروجی ماشین راه‌حل را استخراج کرد.
\vspace*{5mm}
\begin{LTR}\centering
  \cluefont{\normalsize}
  \cellsize{1\baselineskip}
  \sudoku{sudoku.sud}
\end{LTR}
\vspace*{5mm}
(قوانین بازی سودوکو: جدولی ۹ در ۹ به ما داده شده که برخی از سلول‌های آن حاوی یک عدد بین ۱ تا ۹ است. هدف قرار دادن یک عدد از میان ۱ تا ۹ در سلول‌های خالی این جدول است به نحوی که جدول حاصل این سه ویژگی را داشته باشد: آ- هیچ دو سلولی که در یک سطر قرار دارند نباید حاوی عدد یکسان باشند. ب- هیچ دو سلولی که در یک ستون قرار دارند نباید حاوی عدد یکسان باشند. ج- در هر یک از نُه مربع ۳ در ۳ که در جدول مشخص شده‌اند، هیچ دو سلولی نباید حاوی عدد یکسان باشند.)
\begin{ans}
  به ازای $\{ 1, \dots, 9 \} \ni i,j,k$، فرض کنید اتم $P_{i,j,k}$ صادق باشد اگر خانه‌ی سطر $i$ و ستون $j$ در جدول حاوی مقدار $k$ باشد. تعریف کنید
  \[ {\textsc{\lr{Atoms}}} = \{ P_{i,j,k} \mid 1 \le i,j,k \le 9 \} \]
  \[ \Gamma_1 = \{ \neg (P_{i,j,k} \wedge P_{i,j',k}) \mid 1 \le i,j,j',k \le 9, j \neq j' \} \]
  \[ \Gamma_2 = \{ \neg (P_{i,j,k} \wedge P_{i',j,k}) \mid 1 \le i,i',j,k \le 9, i \neq i' \} \]
  \[ \Gamma_3 = \{ \neg (P_{3 n + i, 3 m + j, k} \wedge P_{3 n + i', 3 m + j', k}) \mid k \in \{1,\dots,9\}, n,m \in \{0,1,2\}, i,i',j,j' \in \{1,2,3\}, i \neq i', j \neq j' \} \]
  \[ \Gamma_4 = \{ P_{i,j,k} \mid \text{به ازای هر عدد $k$ که در سلول سطر $i$ و ستون $j$ جدول داده‌شده نوشته شده} \} \]
  \[ \Gamma_5 = \{ \bigvee_{k=1}^9 P_{i,j,k} \mid 1 \le i, j \le 9\} \]
  \[ \Gamma_6 = \{ \neg (P_{i,j,k} \wedge P_{i,j,k'}) \mid 1 \le i,j,k,k' \le 9, k \neq k' \} \]
  هر یک از مجموعه‌های تعریف‌شده با برقراری یکی از شروط مسئله ارضاپذیر خواهند بود. به بیان دقیق‌تر:
  \begin{enumerate}[label=\alph*-]
    \item $\Gamma_1$ ارضاپذیر است اگر هیچ دو سلولی از یک سطر حاوی عدد یکسان نباشند.
    \item $\Gamma_2$ ارضاپذیر است اگر هیچ دو سلولی از یک ستون حاوی عدد یکسان نباشند.
    \item $\Gamma_3$ ارضاپذیر است اگر هیچ دو سلولی از یکی از مربع‌های ۳ در ۳ مشخص‌شده حاوی عدد یکسان نباشند.
    \item $\Gamma_4$ ارضاپذیر است اگر اعداد داده‌شده در سلول مناسب نوشته شده باشند.
    \item $\Gamma_5$ ارضاپذیر است اگر تمام سلول‌ها حاوی لااقل یک عدد باشد.
    \item $\Gamma_6$ ارضاپذیر است اگر هیچ سلولی حاوی دو عدد نباشد.
  \end{enumerate}
  تعریف کنید
  \[ \Gamma = \Gamma_1 \cup \Gamma_2 \cup \Gamma_3 \cup \Gamma_4 \cup \Gamma_5 \cup \Gamma_6 \]
  بنابراین، هر مدل از $\Gamma$ تمام شرایط مسئله را برآورده می‌کند. درنتیجه، اگر با ورودی \textsc{\lr{Atoms}} و $\Gamma$، دستگاه مدل $v$ را خروجی دهد، کافی است برای هر $\{1, \dots, 9\} \ni i,j$، عدد $k$ یکتا را پیدا کنیم که $v(P_{i,j,k}) = 1$ و $k$ را در سلول سطر $i$ و ستون $j$ از جدول بنویسیم.
\end{ans}