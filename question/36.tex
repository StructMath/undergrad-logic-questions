(\emph{ترجمهٔ منفی گودل})
عملگر
$\mathord{(-)}^n\colon\prop\to\prop$
را چنین تعریف می‌کنیم:
\begin{gather*}
    \bot^n=\bot\\
    p_i^n=\neg\neg p_i\\
    (\neg\varphi)^n=\neg\varphi^n\\
    (\varphi\wedge\psi)^n=\varphi^n\wedge\psi^n\\
    (\varphi\vee\psi)^n=\neg\neg(\varphi^n\vee\psi^n)\\
    (\varphi\to\psi)^n=\varphi^n\to\psi^n\\
\end{gather*}\quad\vspace{-1.5cm}
\begin{enumerate}[label=(\alph*)]
    \item ثابت کنید برای هر فرمول
        $\varphi$
        داریم
        $\neg\neg\varphi^n\vdashIPC\varphi^n$.
    \item ثابت کنید برای هر فرمول
        $\varphi$
        داریم
        $\varphi\vdashCPC\varphi^n$
        و
        $\varphi^n\vdashCPC\varphi$.
        
        (راهنمایی: از قضیهٔ جانشینی استفاده کنید.)
    \item ثابت کنید
        $\vdashCPC\varphi$
        اگرر
        $\vdashIPC\varphi^n$.%\footnote{
        %     از آنجا که ترجمهٔ منفی گودل نقیض را حفظ می‌کند از قضیهٔ گلیونکو نتیجه می‌شود که منطق گزاره‌ای کلاسیک ناسازگار است اگرر منطق گزاره‌ای شهودگرایانه ناسازگار است.
        % }
        
        (راهنمایی: از قضیهٔ گلیونکو استفاده کنید.)
\end{enumerate}\quad\vspace{-0.5cm}
\begin{ans}
    \begin{enumerate}[label=(\alph*)]
        \item
        حکم را با استقرا روی پیچیدگی
        $\varphi$
        ثابت می‌کنیم. برای پایهٔ استقرا ابتدا قرار دهید
        $\varphi=\bot$.
        در این صورت
        $\varphi^n=\bot$.
        باید نشان بدهیم
        $\neg\neg\bot\vdashIPC\bot$.
        این حکم با درخت برهان زیر ثابت می‌شود:
        \LTR\begin{prooftree}
            \AxiomC{$[\bot]^1$}
            \negI[1]{$\neg\bot$}

            \AxiomC{$\neg\neg\bot$}
            \negE
        \end{prooftree}\RTL
        
        حال فرض کنید
        $\varphi=p_i$
        و
        $\varphi_n=\neg\neg p_i$.
        باید نشان بدهیم
        $\neg\neg\neg\neg p_i\vdashIPC \neg\neg p_i$.
        این حکم با جایگذاری
        $\neg p_i$
        به جای
        $\varphi$
        در بخش
        \ref{q34:b}
        پرسش
        \ref{q34}
        ثابت می‌شود.
        
        حال فرض می‌کنیم
        $\varphi$
        با استفاده از ادات‌های نقیض، عطف، فصل یا شرطی ساخته شده است و حکم برای اجزای آن ثابت شده است. 
        
        برای
        $\varphi=\neg\psi$
        و
        $\varphi^n=\neg\psi^n$
        باید ثابت کنیم
        $\neg\neg\neg\psi^n\vdashIPC \neg\psi^n$.
        این حکم با جایگذاری
        $\psi^n$
        در بخش
        \ref{q34:b}
        پرسش
        \ref{q34}
        ثابت می‌شود.

        برای
        $\varphi=\psi\wedge\chi$
        و
        $\varphi^n=\psi^n\wedge\chi^n$
        باید ثابت کنیم
        $\neg\neg(\psi^n\wedge\chi^n)\vdashIPC \psi^n\wedge\chi^n$.
        مطابق بخش
        \ref{q34:e}
        پرسش
        \ref{q34}
        می‌دانیم
        $\neg\neg(\psi^n\wedge\chi^n)\vdashIPC \neg\neg\psi^n\wedge\neg\neg\chi^n$.
        طبق فرض استقرا می‌دانیم
        $\neg\neg\psi^n\vdashIPC \psi^n$
        و
        $\neg\neg\chi^n\vdashIPC\chi^n$.
        اکنون کافی است با استفاده از قاعدهٔ حذف عطف از
        $\neg\neg\psi^n\wedge\neg\neg\chi^n$
        دو طرف عطف را نتیجه بگیریم و سپس با استفاده از فرض استقرایی
        $\psi^n$
        و
        $\chi^n$
        را نتیجه بگیریم و سپس با قاعدهٔ معرفی عطف نتیجه بگیریم
        $\psi^n\wedge\chi^n$.

        برای
        $\varphi=\psi\vee\chi$
        و
        $\varphi^n=\neg\neg(\psi^n\vee\chi^n)$
        باید ثابت کنیم
        $$\neg\neg\neg\neg(\psi^n\vee\chi^n)\vdashIPC\neg\neg(\psi^n\vee\chi^n)$$
        این حکم با جایگذاری
        $\neg(\psi^n\vee\chi^n)$
        به جای
        $\varphi$
        در بخش
        \ref{q34:b}
        پرسش
        \ref{q34}
        ثابت می‌شود.
        
        برای
        $\varphi=\psi\to\chi$
        و
        $\varphi^n=\psi^n\to\chi^n$
        باید ثابت کنیم
        $\neg\neg(\psi^n\to\chi^n)\vdashIPC\psi^n\to\chi^n$.
        مطابق بخش
        \ref{q34:d}
        پرسش
        \ref{q34}
        می‌دانیم
        $\neg\neg(\psi^n\to\chi^n)\vdashIPC \neg\neg\psi^n\to\neg\neg\chi^n$.
        همچنین مطابق فرض استقرا داریم
        $\neg\neg\chi^n\vdashIPC\chi^n$.
        از بخش
        \ref{q34:a}
        پرسش
        \ref{q34}
        نیز می‌دانیم
        $\psi^n\vdashIPC\neg\neg\psi^n$.
        حکم را با استنتاج زیر ثابت می‌کنیم
        \LTR\begin{prooftree}
            \AxiomC{$[\psi^n]^1$}
            \noLine\UnaryInfC{$\vdots$}
            \noLine\UnaryInfC{$\neg\neg\psi^n$}

            \AxiomC{$\neg\neg(\psi^n\to\chi^n)$}
            \noLine\UnaryInfC{$\vdots$}
            \noLine\UnaryInfC{$\neg\neg\psi^n\to\neg\neg\chi^n$}

            \toE{$\neg\neg\chi^n$}
            \noLine\UnaryInfC{$\vdots$}
            \noLine\UnaryInfC{$\chi^n$}
            
            \toI[1]{$\psi^n\to\chi^n$}

        \end{prooftree}\RTL
        توجه کنید مطابق اطلاعات قبلی می‌دانیم بخش‌هایی از استنتاج که جا انداخته‌ایم را می‌توان با استنتاجی معتبر تکمیل کرد.

        \item
        راه حل بخش (ب)

        \item
        واضح است که $\varphi^n \vdash_{IPC} \neg\neg\varphi^n$. ضمنا در بخش (الف) ثابت کردیم $\neg\neg\varphi^n \vdash_{IPC} \varphi^n$. بنابراین $\varphi^n \dashv\vdash_{IPC} \neg\neg\varphi^n$. نهایتا واضح است که $\varphi^n \dashv\vdash_{CPC} \neg\neg\varphi^n$. به این ترتیب داریم:
        $$\vdash_{CPC} \varphi \xLeftrightarrow{\text{بخش (ب)}} \vdash_{CPC} \varphi^n \xLeftrightarrow{} \vdash_{CPC} \neg\neg\varphi^n \xLeftrightarrow{\text{قضیهٔ گلیونکو}} \vdash_{IPC} \neg\neg\varphi^n \xLeftrightarrow{} \vdash_{IPC} \varphi^n$$
    \end{enumerate}
\end{ans}
%ANS Incomplete
