در سال ۱۹۷۶ کنت آپل و ولفگانگ هاکن ثابت کردند هر نقشهٔ متناهی را می‌توان با چهار رنگ رنگ‌آمیزی کرد، یعنی با چهار رنگ می‌توان هر نقشهٔ متناهی را چنان رنگ‌آمیزی کرد که هیچ دو کشور همجوار هم‌رنگ نباشند. (یکی از دلایل شهرت این قضیه، قضیهٔ چهار رنگ، استفادهٔ گستردهٔ ریاضی‌دانان از رایانه در اثبات آن است.) با فرض گرفتن این قضیه برای نقشه‌های متناهی ثابت کنید همین حکم دربارهٔ نقشه‌های نامتناهی نیز صادق است.


\begin{ans}
    اثبات ما بر این فرض استوار است که قضیه چهار رنگ برقرار است.

    اتم 
    $X_{i,k}$
    را اینگونه تعریف می کنیم.
    \\
    $X_{i,k}$
    اگر و تنها اگر کشور 
    \lr{i}
    دارای رنگ 
    \lr{k}
    باشد.
    \\
    همچنین فرض می کنیم 
    $k \in \{1,2,3,4\}$
    \\
    فرمول زیر را در نظر بگیرید:
    \begin{flushleft}
        $(X_{i,1} \lor X_{i,2} \lor X_{i,3} \lor X_{i,4}) \land (\underset{\text{$1<=k \neq j<= 4$}}{\bigwedge}\neg(X_{i,k} \wedge X_{i,j}))$
    \end{flushleft}
    این فرمول را 
    $\phi_{i}$
    می نامیم.
    \\
    \\
    این فرمول می گوید که کشور 
    \lr{i}
    حتما دارای یک رنگ است و نمی تواند همزمان دارای دو رنگ باشد. 
    \\
    \\
    حال اتم 
    $Y_{i,j}$
    را اینگونه تعریف می کنیم:
    \\
    $Y_{i,j}$
    اگر و تنها اگر کشور های 
    \lr{i}
    و
    \lr{j}
    مجاور هم باشند.
    \\
    \\
    حال فرمول زیر را در نظر بگیرید:
    \begin{flushleft}
        $(Y_{i,j} \wedge \neg(\underset{\text{$1 <= k <= 4$}}{\bigvee} (X_{i,k} \wedge X_{j,k})))$
    \end{flushleft}
    این فرمول را نیز با 
    $\psi_{i,j}$
    نمایش می دهیم. 
    \\
    \\
    این فرمول می گوید کشورهای 
    \lr{i}
    و
    \lr{j}
    در مجاورت یکدیگر هستند و دارای رنگ یکسانی نمی باشند. 
    \\
    \\
    حال برای تمام کشورها فرمول 
    $\phi$
    و برای تمام کشورهای مجاور فرمول
    $\psi$
    را تشکیل می دهیم و در مجموعه 
    $\Sigma$
    قرار می دهیم. 

    <<نکته: فرمول 
    $\psi_{i,j}$
    برای کشور های 
    \lr{i}
    و
    \lr{j}
    فقط زمانی تعریف می شود و در مجموعه 
    $\Sigma$
    قرار می گیرد که کشورهای 
    \lr{i}
    و
    \lr{j}
    در مجاورت یکدیگر باشند.>>
    \\
    \\
    حال با توجه به نامتناهی بودن نقشه ما مجوعه 
    $\Sigma$
    نیز لزوما نامتناهی است. از طرفی هم می دانیم زیر مجموعه های متناهی 
    $\Sigma$
    شامل گزاره هایی درباره نقشه های متناهی است.
    <<زیرا مجموعه های متناهی شامل تعدادی متناهی گزاره است>>
    \\
    \\
    طبق قضیه 4 رنگ هر کدام از زیرمجموعه های متناهی 
    $\Sigma$
    ارضاپذیر است. در نتیجه و طبق قضیه فشردگی برای منطق گزاره‌ای کل مجموعه
    $\Sigma$
    ارضاپذیر است. 

    پس قضیه 4 رنگ برای نقشه های نامتناهی نیز برقرار است.
\end{ans}
