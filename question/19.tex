در سال ۱۹۷۶ کنت آپل و ولفگانگ هاکن ثابت کردند هر نقشه‌ی متناهی را می‌توان با چهار رنگ رنگ‌آمیزی کرد، یعنی با چهار رنگ می‌توان هر نقشه‌ی متناهی را چنان رنگ‌آمیزی کرد که هیچ دو کشور همجوار هم‌رنگ نباشند. (یکی از دلایل شهرت این قضیه، قضیه‌ی چهار رنگ، استفاده‌ی گسترده‌ی ریاضی‌دانان از رایانه در اثبات آن است.) با فرض گرفتن این قضیه برای نقشه‌های متناهی ثابت کنید همین حکم درباره‌ی نقشه‌های نامتناهی نیز صادق است.