در سال ۱۹۷۶ کنت آپل و ولفگانگ هاکن ثابت کردند هر نقشهٔ متناهی را می‌توان با چهار رنگ رنگ‌آمیزی کرد، یعنی با چهار رنگ می‌توان هر نقشهٔ متناهی را چنان رنگ‌آمیزی کرد که هیچ دو کشور همجوار هم‌رنگ نباشند. (یکی از دلایل شهرت این قضیه، قضیهٔ چهار رنگ، استفادهٔ گستردهٔ ریاضی‌دانان از رایانه در اثبات آن است.) با فرض گرفتن این قضیه برای نقشه‌های متناهی ثابت کنید همین حکم دربارهٔ نقشه‌های نامتناهی نیز صادق است.
\begin{ans}
    اثبات ما بر این فرض استوار است که قضیه چهار رنگ برقرار است. مجموعهٔ کشورها را با $I$ و مجموعهٔ رنگ‌ها را با $K = \{1, 2, 3, 4\}$ نشان می‌دهیم. رابطهٔ $R \subseteq I^2$ نیز نشان‌دهندهٔ مجاورت دو کشور است. به ازای هر $k \in K$ و $i \in I$، اتم $p_{i,k}$ را به معنای رنگ‌آمیزی شدن کشور $i$ با رنگ $k$ می‌گیریم.
    به این ترتیب فرمول $\varphi_i$ که به صورت
    \[ (p_{i,1} \vee p_{i,2} \vee p_{i,3} \vee p_{i,4}) \land (\bigwedge_{\substack{1 \le k, j \le 4 \\ k \neq j}} \neg(p_{i,k} \wedge p_{i,j})) \]
    تعریف شده است، می‌گوید که کشور $i$ دقیقاً دارای یک رنگ یکتا است. فرمول $\psi_{i, j}$ نیز که به صورت
    \[ \bigwedge_{1 \le k \le 4} \neg (p_{i,k} \wedge p_{j,k}) \]
    تعریف شده است، می‌گوید که کشورهای $i$ و $j$ دارای رنگ‌های متفاوت هستند. مجموعهٔ $\Sigma$ را به صورت زیر تعریف کنید:
    \[ \Sigma = \{ \varphi_i  \mid i \in I \} \cup \{ \psi_{i, j} \mid i R j \} \]
    یعنی برای هر کشور $i$ فرمول $\varphi_i$ و برای هر دو کشور مجاور $i$ و $j$ فرمول $\psi_{i, j}$ در مجموعهٔ $\Sigma$ هستند. با توجه به نامتناهی بودن نقشه، $\Sigma$ نیز نامتناهی است. از طرفی هم می‌دانیم که هر زیرمجموعهٔ متناهی از $\Sigma$ شامل گزاره‌هایی درباره نقشه‌های متناهی است و طبق قضیه چهار رنگ دست کم یک مدل دارد. طبق قضیه فشردگی برای منطق گزاره‌ای، کل مجموعه $\Sigma$ مدل دارد و در نتیجه، نقشهٔ نامتناهی تعریف‌شده با $\Sigma$ نیز قابل رنگ‌آمیزی است. پس قضیه چهار رنگ برای نقشه‌های نامتناهی نیز برقرار است.
\end{ans}
