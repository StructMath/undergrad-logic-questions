ثابت کنید:
\begin{enumerate}
\item
اگر $c$ تعداد جایگاه‌هایی در فرمول $A$ باشد که رابطی دوتایی در آن قرار گرفته و $s$ تعداد جایگاه‌هایی در $A$ باشد که یک اتم در آن قرار گرفته، داریم $s=c+1$.
\item
اگر $A$ فرمولی باشد که در آن از $\neg$ استفاده نشده است، طول $A$ فرد است.
\end{enumerate}\quad\vspace{-9mm}
\begin{ans}
  \begin{enumerate}
    \item
    حکم را از طریق استقرا ثابت می‌کنیم. اگر $A$ اتم باشد حکم بدیهی است. حال فرض کنید حکم برای فرمول‌هایی با پیچیدگی کمتر از پیچیدگی $A$ ثابت شده است. اگر $A=(\neg A_1)$ و $c_1$ تعداد جایگاه‌های رابط‌های دوتایی در $A_1$ و $s_1$ تعداد جایگاه‌های اتم‌ها در $A_1$ باشد، واضح است که $s=s_1$ و $c=c_1$ و بنابراین $s=c+1$. همچنین اگر $A=(A_1\square A_2)$ (که در آن $\square$ رابطی دوتایی است) و $c_i$ تعداد جایگاه‌های رابط‌های دوتایی در $A_i$ و $s_i$ تعداد جایگاه‌های اتم‌ها در $A_i$ باشد، داریم $c=c_1+c_2+1$ و $s=s_1+s_2$. حال با توجه به اینکه فرض کرده‌ایم $s_i=c_i+1$، به‌سادگی می‌توان نشان داد $s=c+1$.
    
    
    \item
    حکم را از طریق استقرا ثابت می‌کنیم. اگر $A$ اتم باشد حکم بدیهی است. حال فرض کنید حکم برای فرمول‌هایی با پیچیدگی کمتر از پیچیدگی $A$ ثابت شده است. تنها لازم است حالتی را بررسی کنیم که در آن $A=(A_1\square A_2)$ (که در آن $\square$ رابطی دوتایی است) زیرا در صورتی که $A=(\neg A_1)$ فرض استفاده نشدن از نقیض برقرار نیست. حال واضح است که اگر طول $A_1$ و $A_1$ فرد باشد طول $A$ نیز فرد است.
    
  \end{enumerate}
\end{ans}
