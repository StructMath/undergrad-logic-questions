برای $n$ دلخواه ثابت کنید
$$
A\vee (B_1\wedge\ldots\wedge B_n)\Dashv\vDash(A\vee B_1)\wedge\ldots\wedge(A\vee B_n)
$$
\begin{ans}
فرض می‌کنیم همهٔ فرمول‌های $A$ و $B_1$ تا $B_n$ اتم‌اند. واضح است که اگر حکم را در این حالت ثابت کنیم، حکم در باقی حالت‌ها نیز ثابت شده است چرا که هر مصداق از یک توتولوژی توتولوژی است.

برای اثبات حکم از استقرا استفاده می‌کنیم. برای پایهٔ استقرا قرار دهید $n=2$. در این حالت کافی است جدول درستی دو سوی هم‌ارزی را رسم کنیم:

\begin{tabular}{c|c|c|c|c}
$A$ & $B_1$ & $B_2$ & $A\vee(B_1\wedge B_2)$ & $(A\vee B_1)\wedge(A\vee B_2)$ \\
\hline
$T$ & $T$ & $T$ & $T$ & $T$ \\
$T$ & $T$ & $F$ & $T$ & $T$ \\
$T$ & $F$ & $T$ & $T$ & $T$ \\
$T$ & $F$ & $F$ & $T$ & $T$ \\
$F$ & $T$ & $T$ & $T$ & $T$ \\
$F$ & $T$ & $F$ & $F$ & $F$ \\
$F$ & $F$ & $T$ & $F$ & $F$ \\
$F$ & $F$ & $F$ & $F$ & $F$ \\
\end{tabular}

حال فرض کنید حکم برای $n=k$ ثابت شده است. این هم‌ارزی از حالت پایهٔ استقرا نتیجه می‌شود:
$$
A\vee ((B_1\wedge\ldots\wedge B_k)\wedge B_{k+1})\Dashv\vDash (A\vee (B_1\wedge\ldots\wedge B_k)\wedge (A\vee B_{k+1})
$$
نیز، با استفاده از قضایای ثابت‌شده در مبحث جایگزینی و مفروض بودن حکم برای $n=k$ می‌توانیم نتیجه بگیریم:
$$
(A\vee (B_1\wedge\ldots\wedge B_k)\wedge (A\vee B_{k+1})\Dashv\vDash(A\vee B_1)\wedge\ldots(A\vee B_k)\wedge(A\vee B_{k+1})
$$

\end{ans}