
	ساختارهای زیر از نوع شباهت
	$\langle\mathord{-};2,1;1\rangle$
	را در نظر بگیرید:
	\begin{enumerate}[label=(\alph*)]
		\item $\mathcal{Z^+}=\langle\mathbb{Z},+,-,0\rangle$
		که
		$+$ جمع اعداد صحیح و
		$-$ عملگری است که $n$ را به $-n$ می‌برد.
		\item $\mathcal{Z^\times}=\langle\mathbb{Z},\times,-, 1\rangle$
		$\times$ که ضرب اعداد صحیح و
		$-$ عملگری است که $n$ را به $-n$ می‌برد.
		\item $\mathcal{Z}_4=\langle\{0,1,2,3\},+,-, 0\rangle$
		که $+$ عملگر جمع به پیمانهٔ چهار است یعنی برای جمع هر دو عدد ابتدا آن‌ها را جمع می‌کنیم و سپس باقی‌ماندهٔ حاصل بر چهار را در نظر می‌گیریم
		(برای مثال
		$2+3=1$
		و
		$2+2=0$)
		و
		$-$
		عملگری است که صفر را به صفر و دیگر $n$ها را به $4-n$ می‌برد.
		\item $\mathcal{S}_3=\langle S_3, \circ,^{-1},\text{id}\rangle$
		که
		$S_3$
		مجموعهٔ توابع یک‌به‌یک و پوشا از مجموعهٔ
		$\{0,1,2\}$
		به خودش،
		$\circ$
		ترکیب دو تابع،
		$^{-1}$
		عملگری که هر تابع را به وارون آن می‌برد و
		$\text{id}$
		تابع همانی است.
	\end{enumerate}
	در هر مورد بررسی کنید آیا اصول موضوعهٔ گروه
	(صفحهٔ ۸۰ فن‌دالن)
	در این ساختار صحیح هستند یا نه. سپس دربارهٔ ساختارهایی که گروه هستند بررسی کنید که فرمول‌های زیر در آن‌ها صادق‌اند یا کاذب:
	\begin{align*}
		\varphi_1 &= \forall x\forall y (x\cdot y = y\cdot x) \\
		\varphi_2 &= \exists x (\neg x = e \wedge x\cdot x = x) \\
		\varphi_3 &= \forall x (((x\cdot x)\cdot x)\cdot x = e)
	\end{align*}
	
	\quad\vspace {-0.5cm}
	\begin{ans}
			اصول موضوعه فن‌دالن : \\
			$$
				\forall x,y,z \;\;(x.y).z = x.(y.z)
			$$
			$$
				\exists e \;  \forall x\;\;(e.x = x.e = x)
			$$
			$$
				\forall x \; \exists x^{-1}\;\; (x.x^{-1} = x^{-1}. x = e)
			$$
		\begin{enumerate}[label = \alph*)]
			\item
			کافیست که زبان را گسترش دهیم و سپس مطابق کتاب فن‌دالن ابتدا جانشانی ثوابت به جای متغیر‌های پابند را انجام دهیم. برای مثال در مورد گزاره اول چون از مبانی ریاضی می‌دانیم که ضرب تعریف شده روی اعداد صحیح شرکت‌پذیر است بنابراین هر جانشانی ثوابت به جای متغیر‌ها درست است. بنابراین اگر این گزاره را 
			$\varphi$
			بنامیم داریم
			$$
				min\{\llbracket \varphi[\bar a / x][\bar b / y][\bar c / z] \rrbracket _ \mathfrak{M} \;\;|\;\; a, b, c \in \mid \mathfrak{M} \mid \} = 1
			$$
			همچنین گزاره‌های دوم نیز صادق است.چرا که می‌خواهیم ابتدا به جای 
			$e$
			جانشانی انجام دهیم که تحت مدل ارزش گزاره
			$\forall x\;\;(e.x = x.e = x)$
			به ازای هر‌جانشانی برای 
			$x$
			برابر ۱ باشد. 
			 اگر گزاره دوم را 
			$\psi$
			بنامیم داریم :‌
			$$
				max\{min\{\llbracket \varphi[\bar b / e][\bar a / x] \rrbracket_\mathfrak{M} \;\;|\;\; a \in \mid \mathfrak{M} \mid \} \;\;|\;\; b \in \mid \mathfrak{M} \mid \} = 1
			$$
			چرا که
			$$
				min\{\llbracket \varphi[\bar 0 / e][\bar a / x] \rrbracket_\mathfrak{M} \;\;|\;\; a \in \mid \mathfrak{M} \mid \}
			$$
			برای گزاره سوم نیز کافیست آن را 
			$\varphi$
			بنامیم. آنگاه داریم
			$$
				min\{max\{\llbracket \varphi[\bar b / x][\bar a / x^{-1}] \rrbracket_\mathfrak{M} \;\;|\;\; a \in \mid \mathfrak{M} \mid \} \;\;|\;\; b \in \mid \mathfrak{M} \mid \} = 1
			$$
			چرا که می‌دانیم برای هر جانشانی b به جای x
			$$
				\llbracket \varphi[\bar b / x][ \bar{-b} / x] \rrbracket_\mathfrak{M} \;\;|\;\; a \in \mid \mathfrak{M} \mid = 1
			$$
			همچنین دقت داریم که منظور از 
			$-b$
			همان
			$ 0 - b $
			است.
			و برای موارد دیگر خواسته شده به روش مشابه حکم صادق است.\\
			حال فرض کنیم که یک ساختار گروه است. گزاره 
			$\varphi _ 1$
			لزوما درست نیست. مثلا گروه ماتریس‌های
			 $n \times n$
			  که وارون پذیر هستند را در نظر بگیریم.
			عضو همانی همان 
			$I_n$
			است. و ضرب ماتریس‌ها شرکت‌پذیر است اما لزوما جا‌به‌جایی نیست. ولی در ساختار الف این گزاره برقرار است.
			برای گزاره 
			$\varphi_2$
			 این گزاره همواره غلط است.
			 $$
			 	x.x = x \Longrightarrow x = x.(x.x^{-1}) = (x.x).x^{-1} = x.x^{-1} = e
			 $$
			 گزاره آخر نیز درمورد وجود عضو مرتبه ۴ در گروه است. که همواره درست یا غلط نیست. برای مثال در دستگاه کامل‌ مانده‌های ۵ عضو مرتبه ۴ نداریم اما در دستگاه کامل مانده‌های ۸ مرتبه ۲ برابر ۴ است. گروه جمعی دستگاه کامل مانده‌های ۵ به صورت
			 $\{0, 1, 2, 3, 4\}$
			 است. که حمع آن به صورت زیر است
			 $$
			 	a.b := (a + b) \;\; mod 5
			 $$
		\end{enumerate}
	\end{ans}
