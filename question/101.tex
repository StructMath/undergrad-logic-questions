ساختارهای زیر از نوع شباهت
$\langle\mathord{-};2,1;1\rangle$
را در نظر بگیرید:
\begin{enumerate}[label=(\alph*)]
    \item $\mathcal{Z^+}=\langle\mathbb{Z},+,-,0\rangle$
    که
    $+$ جمع اعداد صحیح و
    $-$ عملگری است که $n$ را به $-n$ می‌برد.
    \item $\mathcal{Z^\times}=\langle\mathbb{Z},\times,-, 1\rangle$
    $\times$ که ضرب اعداد صحیح و
    $-$ عملگری است که $n$ را به $-n$ می‌برد.
    \item $\mathcal{Z}_4=\langle\{0,1,2,3\},+,-, 0\rangle$
    که $+$ عملگر جمع به پیمانهٔ چهار است یعنی برای جمع هر دو عدد ابتدا آن‌ها را جمع می‌کنیم و سپس باقی‌ماندهٔ حاصل بر چهار را در نظر می‌گیریم
    (برای مثال
    $2+3=1$
    و
    $2+2=4$)
    و
    $-$
    عملگری است که $n$ را به $4-n$ می‌برد.
    \item $\mathcal{S}_3=\langle S_3, \circ,^{-1},\text{id}\rangle$
    که
    $S_3$
    مجموعهٔ توابع یک‌به‌یک و پوشا از مجموعهٔ
    $\{0,1,2\}$
    به خودش،
    $\circ$
    ترکیب دو تابع،
    $^{-1}$
    عملگری که هر تابع را به وارون آن می‌برد و
    $\text{id}$
    تابع همانی است.
\end{enumerate}
در هر مورد بررسی کنید آیا اصول موضوعهٔ گروه
(صفحهٔ ۸۰ فن‌دالن)
در این ساختار صحیح هستند یا نه. سپس دربارهٔ ساختارهایی که گروه هستند بررسی کنید که فرمول‌های زیر در آن‌ها صادق‌اند یا کاذب:
\begin{align*}
    \varphi_1 &= \forall x\forall (y x\cdot y = y\cdot x) \\
    \varphi_2 &= \exists x (\neg x = e \wedge x\cdot x = x) \\
    \varphi_3 &= \forall x (((((x\cdot x)\cdot x)\cdot x)\cdot x)\cdot x = e)
\end{align*}