به کمک روش تابلو بررسی کنید کدام یک از استنتاج‌های زیر برقرار است:
\begin{enumerate}[label=(\alph*)]
  \item $p \rightarrow (q \vee r) \vdash (p \rightarrow q) \vee (p \rightarrow r)$
  \item $p \rightarrow (q \wedge r) \vdash (p \rightarrow q) \wedge (p \rightarrow r)$
  \item $\neg ((p \rightarrow r) \vee (q \rightarrow r)) \vdash \neg ((p \vee q) \rightarrow r)$
  \item $\neg ((p \vee q) \rightarrow r) \vdash \neg ((p \rightarrow r) \vee (q \rightarrow r))$
\end{enumerate}
\quad
\begin{ans}
  می‌دانیم $\Gamma \vdash A$ اگرر $\Gamma \cup \{\neg A\}$ ناسازگار باشد. بنابراین کافی است در هر مورد نشان دهیم تابلوی مربوط به $\Gamma \cup \{\neg A\}$ بسته می‌شود.
  \begin{multicols}{2}
    (آ)
    \begin{forest}m
      [
        $p \rightarrow (q \vee r)$\\
        $\neg ((p \rightarrow q) \vee (p \rightarrow r))$
        [
          $\neg (p \rightarrow q)$\\
          $\neg (p \rightarrow r)$
          [
            $p$\\
            $\neg q$
            [
              $p$\\
              $\neg r$
              [
                $\neg p$\\
                $\times$
              ]
              [
                $q \vee r$
                [
                  $q$\\
                  $\times$
                ]
                [
                  $r$\\
                  $\times$
                ]
              ]
            ]
          ]
        ]
      ]
    \end{forest}

    (ب)
    \begin{forest}m
      [
        $p \rightarrow (q \wedge r)$\\
        $\neg ((p \rightarrow q) \wedge (p \rightarrow r))$
        [
          $\neg (p \rightarrow q)$
          [
            $p$\\
            $\neg q$
            [
              $\neg p$\\
              $\times$
            ]
            [
              $q \wedge r$
              [
                $q$\\
                $r$\\
                $\times$
              ]
            ]
          ]
        ]
        [
          $\neg (p \rightarrow r)$
          [
            $p$\\
            $\neg r$
            [
              $\neg p$\\
              $\times$
            ]
            [
              $q \wedge r$
              [
                $q$\\
                $r$\\
                $\times$
              ]
            ] 
          ]
        ]
      ]
    \end{forest}
  \end{multicols}
  % \pagebreak
  \begin{multicols}{2}
    (ج)
    \begin{forest}m
      [
        $\neg ((p \rightarrow r) \vee (q \rightarrow r))$\\
        $\neg \neg ((p \vee q) \rightarrow r)$
        [
          $\neg (p \rightarrow r)$\\
          $\neg (q \rightarrow r)$
          [
            $p$\\
            $\neg r$
            [
              $q$\\
              $\neg r$
              [
                $(p \vee q) \rightarrow r$
                [
                  $\neg (p \vee q)$
                  [
                    $\neg p$\\
                    $\neg q$\\
                    $\times$
                  ]
                ]
                [
                  $r$\\
                  $\times$
                ]
              ]
            ]
          ]
        ]
      ]
    \end{forest}

    (د)
    \begin{forest}m
      [
        $\neg ((p \vee q) \rightarrow r)$\\
        $\neg \neg ((p \rightarrow r) \vee (q \rightarrow r))$
        [
          $p \vee q$\\
          $\neg r$
          [
            $(p \rightarrow r) \vee (q \rightarrow r)$
            [
              $p \rightarrow r$
              [
                $\neg p$
                [
                  $p$\\
                  $\times$
                ]
                [
                  $q$\\
                  $\bigcirc$
                ]
              ]
              [
                $r$\\
                $\times$
              ]
            ]
            [
              $q \rightarrow r$
              [
                $\neg q$
                [
                  $p$\\
                  $\bigcirc$
                ]
                [
                  $q$\\
                  $\times$
                ]
              ]
              [
                $r$\\
                $\times$
              ]
            ]
          ]
        ]
      ]
    \end{forest}
  \end{multicols}
  همان‌طور که مشاهده می‌شود، تابلوهای مربوط به همهٔ موارد به جز مورد آخر بسته می‌شود. دقت کنید که در نتیجه همهٔ موارد به جز مورد آخر برقرار هستند. دقت کنید دو مدل $v_1$ و $v_2$ با تعریف $v_1(p) = F$، $v_1(q) = T$ و $v_1(r) = F$ و $v_2(p) = T$، $v_2(q) = F$ و $v_2(r) = F$ دو مثال نقض برای برقراری استنتاج در مورد آخر هستند.
\end{ans}