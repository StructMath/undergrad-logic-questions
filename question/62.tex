
به رابطه 
$\leq \ \subseteq A \times A$
روی مجموعه
$A$
ترتیب جزئی می گوییم اگر بازتابی، پادمتقارن و تراگذری باشد. ترتیب جزئی
$\leq$
را تریب خطی می نامیم اگر برای هر دو
$a,b\in A$
داشته باشیم 
$a\leq b$
یا
$b\leq a$.
می گوییم ترتیب 
$\leq_2$
گسترشی از ترتیب 
$\leq_1$
است اگر 
$\leq_1\ \subseteq \leq_2$.
می دانیم هر ترتیب جزئی روی مجموعه‌ای متناهی را می توان به یک ترتیب خطی گسترش داد. 
(لازم نیست این را اثبات کنید.)
نشان دهید همین حکم برای ترتیب های جزئی روی مجموعه های نامتناهی هم صادق است.
\begin{ans}
    ابتدا ویژگی های یک رابطه ترتیب جزئی را به زبان منطق گزاره ها در می آوریم و یک مجموعه از فرمول ها جمع می کنیم. این مجموعه از فرمول ها را 
    $\Sigma$
    می نامیم.
    \\
    \\
    فرض کنیم 
    $A$
    یک مجموعه نامتناهی است و رابطه 
    $\leq$
    یک رابطه ترتیب جزئی روی 
    $A$
    است. به ازای هر 
    $a,b, \in A$
    اتم 
    $LE_{a,b}$
    را اینگونه تعریف می کنیم:
    \begin{flushleft}
        $LE_{a,b} \ \ if \ \ and \ \ only \ \ if \ \ a \leq b$
    \end{flushleft}
    برای هر 
    $a,b \in A$
    فرمول
    $EQ_{a,b}$
    را اینگونه تعریف می کنیم:
    \begin{flushleft}
        $EQ_{a,b} \leftrightarrow a = b$
    \end{flushleft}

    برای هر 
    $a,b \in A$
    که 
    $a \leq b$
    فرمول 
    $LE_{a,b}$
    را به 
    $\Sigma$
    اضافه می کنیم. همچنین به ازای هر 
    $a\in A$
    فرمول 
    $LE_{a,a}$
    را نیز به 
    $\Sigma$
    اضافه می کنیم. 
    <<توضیح: این فرمول ویژگی بازتابی بودن رابطه ترتیب را تضمین می کند>>

    به ازای هر 
    $a,b \in A$
    اگر 
    $a \neq b$
    آنگاه 
    $\neg EQ_{a,b}$
    را به 
    $\Sigma$
    اضافه می کنیم.

    همچنین اگر 
    $a = b$
    آنگاه 
    $EQ_{a,b}$
    را به 
    $\Sigma$
    اضافه می کنیم. 
    \\
    \\
    برای هر 
    $a,b,c \in A$
    فرمول های
    \[((LE_{a,b}\land LE_{b,c})\rightarrow LE_{a,c})\]
    \[((LE_{a,b}\land LE_{b,a}\rightarrow EQ_{a,b}))\]
    را به 
    $\Sigma$
    اضافه می کنیم.
    << توضیح: فرمول اول ویژگی تراگذری بودن رابطه ترتیب را تضمین می کند. همچنین فرمول دوم ویژگی پادمتقارن بودن رابطه ترتیب را به ما می دهد>>

    و در نهایت به ازای هر 
    $a,b \in A$
    فرمول 
    \[LE_{a,b} \lor LE_{b,a}\]
    را به 
    $\Sigma$
    اضافه می کنیم.
    <<توضیح: اضافه کردن این فرمول خطی بودن رابطه ترتیب را به ما می دهد.>>

    حال مجموعه متناهی
    $\Sigma' \subset \Sigma$
    را در نظر بگیرید. با توجه به اعضای مجوعه 
    $\Sigma$
    می دانیم که اعضای مجموعه 
    $\Sigma'$
    شامل فرمول هایی است که درباره امکان پذیری گسترش
    $\leq$
    به یک زیر مجموعه متناهی از 
    $A$
    است. پس طبق فرض این مجموعه ارضاپذیر است. 

    با استدلال بالا و طبق قضیه فشردگی برای منطق گزاره‌ای نتیجه می گیریم که 
    $\Sigma$
    ارضاپذیر است. 

    ارضاپذیری 
    $\Sigma$
    نتیجه می دهد که گسترش 
    $\leq$
    به یک رابطه ترتیب خطی 
    امکان پذیر است.


\end{ans}
