به رابطهٔ
$\leq\ \subseteq A\times A$
روی مجموعهٔ
$A$
ترتیب جزئی می‌گوییم اگر بازتابی، پادمتقارن و تراگذری باشد. ترتیب جزئی
$\leq$
را ترتیب خطی می‌نامیم اگر برای هر دو عضو
$a,b\in A$
داشته باشیم
$a\leq b$
یا
$b\leq a$.
می‌گوییم ترتیب
$\leq_2$
گسترشی از ترتیب
$\leq_1$
است اگر
$\leq_1\ \subseteq\ \leq_2$.
می‌دانیم هر ترتیب جزئی روی مجموعه‌ای متناهی را می‌توان به یک ترتیب خطی گسترش داد. (لازم نیست این را اثبات کنید.)
نشان دهید همین حکم برای ترتیب‌های جزئی روی مجموعه‌های نامتناهی هم صادق است.
