(\emph{قضیهٔ دِ بروین--اردوش})
به دوتایی
$G=(V,E)$
گراف می‌گوییم اگر
$E\subseteq V\times V$
و
$E$
متقارن و پادبازتابی باشد؛ یعنی برای هر
$v_1$
و
$v_2$
از
$V$
اگر
$(v_1,v_2)\in E$
آنگاه
$(v_2,v_1)\in E$
و
برای هر
$v$
از
$V$
داشته باشیم
$(v,v)\not\in V$.
گراف
$G$
را
$k$-رنگ‌شونده
می‌نامیم اگر رنگ‌آمیزی
$f\colon\{0,\cdots,k-1\}\to V$
وجود داشته باشد که به هیچ دو رأس مجاور رنگ یکسان نسبت ندهد؛ به عبارت دیگر برای هر دو رأس
$v_1$
و
$v_2$
از
$V$
اگر
$(v_1,v_2)\in E$
داشته باشیم
$f(v_1)\neq f(v_2)$.
به گراف
$G_0=(V_0,E_0)$
زیرگرافی از
$G=(V,E)$
می‌گوییم
اگر
$V_0\subseteq V$
و
$E_0\subseteq E$.
نشان دهید هر گراف
$G$
$k$-رنگ‌شونده
است اگرر
هر زیرگراف متناهی آن
$k$-رنگ‌شونده
باشد.

(مطابق قضیهٔ چهار رنگ هر گراف متناهی مسطح
$4$-رنگ‌شونده
است. با استدلالی مشابه نتیجه می‌شود هر گراف مسطح  $4$-رنگ‌شونده
است.)
