توتولوژی
$(((p\to q)\to p)\to p)$
مشهور به قانون پِرس
(\lr{Peirce's law})
است.
\begin{enumerate}[label=(\alph*)]
\item
توتولوژی بودن قانون پرس را با استفاده از دستگاه استنتاج طبیعی گنتزن ثابت کنید.
\item
فرض کنید به جای دو ارزش‌صدق $1$ و $0$ مجموعه‌ی ارزش‌های صدق ما مجموعه‌ی $\{0,\frac{1}{2},1\}$ باشد.
همچنین فرض کنید همهٔ ارزیاب‌ها از این قواعد پیروی کنند:
\begin{align*}
  v(\neg\varphi)&=\begin{cases*}
    1 & \lr{if \ }$v(\varphi)=0$ \\
    0 & \lr{otherwise}
  \end{cases*}\\
  v(\varphi\vee\psi)&=\max(v(\varphi),v(\psi)) \\
  v(\varphi\wedge\psi)&=\min(v(\varphi),v(\psi)) \\
  v(\varphi\to\psi)&=\begin{cases*}
    1 & \lr{if} \ $v(\varphi)\leq v(\psi)$ \\
    0 & \lr{otherwise}
  \end{cases*}
\end{align*}
برای تمام ادات‌ها جدول ارزش رسم کنید.\footnote{
این منطق توسط گودل معرفی شده و به آن منطق
$G_3$
گفته می‌شود.
}
\item
مقادیری از $p$ و $q$ را بیابید که برای آن‌ها ارزش‌صدق $(((p\to q)\to p)\to p)$ برابر $1$ نیست.
\item
نشان بدهید هر استنتاج $\mathcal{D}$
در دستگاه استنتاج طبیعی گنتزن که از
\lr{RAA}
استفاده نکند در منطق سه‌ارزشی معرفی‌شده در بخش قبل دارای این ویژگی است: اگر $A$ نتیجه‌ی $D$ باشد و $v$ ارزیابی باشد که $v(A)<1$، مقدمه‌ی حذف‌نشده‌ی $B$ای در $D$ هست چنانکه $v(B)\leq v(A)$.
\item
با استفاده از نتایج به‌دست‌آمده در بخش‌های قبل ثابت کنید قانون پرس را نمی‌توان بدون استفاده از
\lr{RAA}
اثبات کرد.
\item با استدلالی مشابه نشان دهید توتولوژی‌های
  $p\vee\neg p$
  و
  $\neg\neg p\to p$
  را نمی‌توان بدون
  \lr{RAA}
  ثابت کرد.
\end{enumerate}\quad\vspace{-1cm}
\begin{ans}
  \begin{enumerate}[label=(\alph*)]
    \item\quad\LTR
    \begin{prooftree}
      \AXC{$[(p \rightarrow q) \rightarrow p]^2$}
      \AXC{$[p]^3$}
      \AXC{$[\neg p]^1$}
      \LeftLabel{$3$}\RightLabel{$\neg E$}
      \BIC{$\bot$}
      \RightLabel{$\bot$}
      \UIC{$q$}
      \RightLabel{$\rightarrow I$}
      \UIC{$p \rightarrow q$}
      \RightLabel{$\rightarrow E$}
      \BIC{$p$}
      \AXC{$[\neg p]^2$}
      \RightLabel{$\neg E$}
      \BIC{$\bot$}
      \LeftLabel{$1$}\RightLabel{$RAA$}
      \UIC{$p$}
      \LeftLabel{$2$}\RightLabel{$\rightarrow I$}
      \UIC{$((p \rightarrow q) \rightarrow p) \rightarrow p$}
    \end{prooftree}
    \RTL
    \item با توجه به تعریف
    \LTR
    \begin{tabular}{c | c c c}
      \diagbox[linecolor=white]{$p$}{$q$} & $0$ & $\frac{1}{2}$ & $1$ \\ \hline
      $0$ & $1$ & $1$ & $1$ \\
      $\frac{1}{2}$ & $0$ & $1$ & $1$ \\
      $1$ & $0$ & $\frac{1}{2}$ & $1$
    \end{tabular}
    \RTL
    \item براساس بخش قبل، فقط اگر $v(p) = {1\over2}$ و $v(q) = 0$ باشد مقدار گزاره مخالف $1$ است.
    \LTR
    \begin{tabular}{c | c | r c l c l c l}
      $p$ & $q$ & $((p$ & $\rightarrow$ & $q)$ & $\rightarrow$ & $p)$ & $\rightarrow$ & $p$ \\
      \hline
      $0$ & $0$ & & $1$ & & $0$ & & $1$ & \\
      $1\over2$ & $0$ & & $0$ & & $1$ & & $1\over2$ & \\
      $1$ & $0$ & & $0$ & & $1$ & & $1$ & \\
      $0$ & $1\over2$ & & $1$ & & $0$ & & $1$ & \\
      $1\over2$ & $1\over2$ & & $1$ & & $1\over2$ & & $1$ & \\
      $1$ & $1\over2$ & & $1\over2$ & & $1$ & & $1$ & \\
      $0$ & $1$ & & $1$ & & $0$ & & $1$ & \\
      $1\over2$ & $1$ & & $1$ & & $1\over2$ & & $1$ & \\
      $1$ & $1$ & & $1$ & & $1$ & & $1$ &
    \end{tabular}
    \RTL

    \item استنتاجی دلخواه مانند $D$ فرض کنید که $A$ را نتیجه می‌دهد. همچنین ارزیاب دلخواهی مانند $v$ فرض کنید به طوری که $v(A)$ برابر با $0$ یا $1\over2$ باشد. با استقرا روی اندازه‌ی $D$ نشان می‌دهیم مقدمه‌ی حذف نشده‌ای در $D$ وجود دارد که $v$ ارزشی کم‌تر یا مساوی ارزش $A$ به آن نسبت می‌دهد. بنابراین، فرض استقرا تضمین می‌کند در هر استنتاج کوتاه‌تر از $D$ که نتیجه‌ای با ارزش کم‌تر از $1$ داشته باشد، مقدمه‌ی حذف نشده‌ای با ارزش کم‌تر یا مساوی نتیجه‌ی استنتاج وجود دارد.
    
    مسئله را در چهار حالت، براساس ارزش $A$ در $v$ و آخرین قاعده‌ی استفاده شده در $D$ بررسی می‌کنیم.
    \begin{enumerate}[label=(\roman*)]
      \item اگر آخرین قاعده $\rightarrow E$ باشد و $v(A) = 0$، یعنی گزاره‌ی $B$ و دو استنتاج کوتاه‌تر از $D$ مانند $D_1$ و $D_2$ وجود دارند که به ترتیب $B$ و $B \rightarrow A$ را نتیجه می‌دهند.
      \LTR\begin{prooftree}
        \AXC{$D_1$}
        \noLine
        \UIC{$B$}
        \AXC{$D_2$}
        \noLine
        \UIC{$B \rightarrow A$}
        \RightLabel{$\rightarrow E$}
        \BIC{$A$}
      \end{prooftree}\RTL
      چون $v(A) = 0$، از جدول درستی سه-ارزشی شرطی می‌توان نتیجه گرفت که $v(B \rightarrow A) = 0$ یا $v(B \rightarrow A) = 1$. اگر $v(B \rightarrow A) = 0$، پس از فرض استقرا می‌دانیم $D_2$ مقدمه‌ی حذف نشده‌ای مانند $B_2$ دارد که مقدمه‌ی $D$ نیز هست و داریم
      $$v(B_2) \leq v(B \rightarrow A) \leq v(A)$$
      اگر $v(B \rightarrow A) = 1$، آنگاه بر اساس جدول درستی $v(B) = 0$. در این صورت نیز $D_1$ در فرض استقرا صدق می‌کند و در نتیجه، مقدمه‌ی حذف نشده‌ای مانند $B_1$ دارد که مقدمه‌ی $D$ هم هست و داریم
      $$v(B_1) \leq v(B) \leq v(A)$$
      در هر دو حال حکم واضح است.

      \item اگر آخرین قاعده $\rightarrow E$ باشد و $v(A) = \frac{1}{2}$، طبق جدول درستی ${v(B \rightarrow A) = \frac{1}{2}}$ یا $v(B \rightarrow A) = 1$. اگر $v(B \rightarrow A) = \frac{1}{2}$، باز هم مانند حالت قبل، $D_2$ در فرض استقرا صدق می‌کند. پس یک مقدمه‌ی حذف نشده با ارزش کم‌تر یا مساوی $v(B \rightarrow A)$، و در نتیجه کم‌تر یا مساوی $v(A)$ دارد. اگر $v(B \rightarrow A) = 1$، آنگاه $v(B) = 0$ یا $v(B) = \frac{1}{2}$، که در هر دو صورت $D_1$ در فرض استقرا صدق می‌کند و مشابهاً حکم اثبات می‌شود.
      
      \item اگر آخرین قاعده $\rightarrow I$ باشد،
      یعنی$B$ و $C$ وجود دارد که $A = B \rightarrow C$ است و $C$ توسط استنتاجی کوتاه‌تر از $D$ به نام $D'$ با فرض $B$ اثبات می‌شود.
      \LTR\begin{prooftree}
        \AXC{$[B]$}
        \noLine
        \UIC{$D'$}
        \noLine
        \UIC{$C$}
        \RightLabel{$\rightarrow I$}
        \UIC{$B \rightarrow C$}
      \end{prooftree}\RTL
      فرض کنیم $v(A) = v(B \rightarrow C) = 0$ است، در نتیجه، طبق جدول درستی با $v(C) = 0$ باشد. براساس فرض استقرا مقدمه‌ی حذف نشده‌ای مانند $B'$ در $C$ وجود دارد که $v(B') \leq v(C)$. واضح است که $v(B') \leq v(A)$، اما برای این که مطمئن باشیم $B'$ مقدمه‌ی $A$ هم هست، باید نشان دهیم $B \neq B'$. اما می‌دانیم $v(B \rightarrow C) = 0$، پس طبق جدول درستی $v(B)$ نمی‌تواند $0$ باشد، در حالی که $v(B') \leq v(C) = 0$.

      \item فرض کنیم $v(A) = v(B \rightarrow C) = \frac{1}{2}$، که مستلزم آن است که $v(B) = 1$ و $v(C) = \frac{1}{2}$. باز هم با استدلال مشابه، بر اساس فرض استقرا، مقدمه‌ی حذف نشده‌ای برای $C$ وجود دارد که در شرایط حکم صدق می‌کند و مقدمه‌ای برای $A$ نیز هست.
    \end{enumerate}

    \item بر اساس نتیجه‌ی بخش (د)،‌ اگر بتوان قانون پرس را در دستگاهی که فقط شامل ادات شرطی است اثبات کرد، باید ارزش آن در هر ارزیاب سه-ارزشی برابر با $1$ باشد. اما طبق نتیجه‌ی بخش (ج) می‌دانیم که در یکی از ارزیاب‌ها این مقدار برابر با $\frac{1}{2}$ است. پس قانون پرس در این دستگاه قابل اثبات نیست.
  \end{enumerate}
\end{ans}