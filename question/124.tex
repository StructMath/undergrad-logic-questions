~\marginpar[left]{\textbf{(۱۵ نمره)}}
فرض کنید $T$ یک نظریهٔ مرتبهٔ اوّل است که حدّ اقل دو مدل متناهی از اندازه‌های متمایز دارد. ثابت کنید $T$ کامل نیست.
\begin{ans}
    فرض کنید $T$ مدل‌هایی به اندازه‌های $n$ و $m$ دارد که $n\neq m$.
    همچنین فرض کنید
    $\varphi_n$
    فرمولی باشد که فقط در مدل‌های $n$-عضوی صادق است (تمرین ۴). واضح است که مدل $m$-عضوی $\varphi_n$ را مدل نمی‌کند در حالی که مدل $n$-عضوی $\neg\varphi_n$ را مدل نمی‌کند. بنابراین داریم
    $T\not\vdash \varphi_n$
    و
    $T\not\vdash \neg\varphi_n$.
\end{ans}