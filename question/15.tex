\begin{enumerate}[label=(\alph*)]
  \item
  ثابت کنید
  $\neg(p_1\vee p_2)\sledom\models\neg p_1\wedge\neg p_2$.
  
  \item
  با استفاده از قضایای ثابت‌شده در مبحث جایگزینی ثابت کنید
  $$\neg((p_1\vee p_2)\vee p_3)\sledom\models (\neg p_1 \wedge \neg p_2)\wedge\neg p_3$$
  
  \item
  با استقرا ثابت کنید به ازای هر $n$ داریم
  $$\neg(\ldots(p_1\vee p_2)\vee \ldots)\vee p_n)\sledom\models (\ldots(\neg p_1\wedge \neg p_2)\wedge\ldots)\wedge\neg p_n$$
\end{enumerate}\quad\vspace{-9mm}
%ANS Needs editing
  \begin{ans}
    \begin{enumerate}[label=(\alph*)]
      \item جدول درستی گزاره‌های دو سمت هم‌ارزی را مجاسبه می‌کنیم:
      \LTR
      \begin{center}
        \begin{tabular}{ c | c | c | c }
          $p_1$ & $p_2$ & $\neg (p_1 \vee p_2)$ & $\neg p_1 \wedge \neg p_2$ \\
          \hline
          $F$ & $F$ & $T$ & $T$ \\
          $F$ & $T$ & $F$ & $F$ \\
          $T$ & $F$ & $F$ & $F$ \\
          $T$ & $T$ & $F$ & $F$
        \end{tabular}
      \end{center}
      \RTL
      دو ستون آخر یکسان هستند. در نتیجه هم‌ارزی برقرار است.
  
      \item
      می‌دانیم جایگزین کردن گزاره‌های هم‌ارز به جای یک اتم در یک گزاره، تغییری در ارزش آن گزاره نمی‌دهد. به عبارت دیگر اگر
      $A \sledom\models B$ و
      $C \sledom\models D$
      آنگاه
      $A[C/p_i] \sledom\models B[D/p_i]$
      برای هر اتم $p_i$.
  
      هم‌ارزی اثبات‌شده در بخش قبل را در نظر بگیرید. ابتدا با جایگزینی‌های متوالی $p_2$ را به $p_3$ تغییر می‌دهیم.
      حال با جایگزینی
      $(p_1 \vee p_2)$
      به جای
      $p_1$
      خواهیم داشت
      $$ \neg ((p_1 \vee p_2) \vee p3) \sledom\models (\neg (p_1 \vee p_2)) \wedge \neg p_3 $$
      حال گزارهٔ
      $p_1 \wedge \neg p_3$
      را در نظر بگیرید. با جایگزینی دو سمت هم‌ارزی اثبات‌شده در بخش قبل در این گزاره به جای
      $p_1$
      داریم
      $$ (\neg (p_1 \vee p_2)) \wedge \neg p_3 \sledom\models (\neg p_1 \wedge \neg p_2) \wedge \neg p_3 $$
      چون هم‌ارزی تراگذری است، می‌توان نتیجه گرفت
      $$ \neg((p_1\vee p_2)\vee p_3)\sledom\models (\neg p_1 \wedge \neg p_2)\wedge\neg p_3 $$
  
      \item
      حکم را با استقرا روی
      $n$
      ثابت می‌کنیم:
  
      پایهٔ استقرا:
      فرض کنیم $n = 3$. طبق بخش قبل حکم برقرار است.
  
      گام استقرایی:
      فرض کنیم $n = m+1$ و حکم برای $m$ برقرار باشد:
      $$\neg(\ldots(p_1\vee p_2)\vee \ldots)\vee p_m)\sledom\models (\ldots(\neg p_1\wedge \neg p_2)\wedge\ldots)\wedge\neg p_m$$
      ابتدا با جایگزینی‌های متوالی، همهٔ اتم‌های $p_2$ تا $p_m$ را به ترتیب با $p_3$ تا $p_{m+1}$ جایگزین می‌کنیم:
      $$\neg(\ldots(p_1\vee p_3)\vee \ldots)\vee p_{m+1})\sledom\models (\ldots(\neg p_1\wedge \neg p_3)\wedge\ldots)\wedge\neg p_{m+1}$$
      ادامهٔ کار مانند بخش قبل خواهد بود. هم‌ارزی بخش اوّل را به جای $p_1$ در هم‌ارزی بالا جایگزین می‌کنیم:
      $$\neg(\ldots((p_1 \vee p_2) \vee p_3)\vee \ldots)\vee p_{m+1})\sledom\models (\ldots((\neg (p_1 \vee p_2))\wedge \neg p_3)\wedge\ldots)\wedge\neg p_{m+1}$$
      حال گزارهٔ زیر را در نظر بگیرید:
      $$(\ldots(p_1\wedge \neg p_3)\wedge\ldots)\wedge\neg p_{m+1}$$
      اگر دو سمت هم‌ارزی بخش اوّل را در این گزاره با $p_1$ جایگزین کنیم، خواهیم داشت
      $$ (\ldots((\neg (p_1 \vee p_2))\wedge \neg p_3)\wedge\ldots)\wedge\neg p_{m+1} \sledom\models (\ldots((\neg p_1 \wedge \neg p_2) \wedge \neg p_3)\wedge\ldots)\wedge\neg p_{m+1}  $$
      از تراگذری بودن هم‌ارزی می‌توان نتیجه گرفت
      $$ \neg(\ldots((p_1 \vee p_2) \vee p_3)\vee \ldots)\vee p_{m+1}) \sledom\models (\ldots((\neg p_1 \wedge \neg p_2) \wedge \neg p_3)\wedge\ldots)\wedge\neg p_{m+1}  $$
      پس حکم برای $n = m+1$ برقرار است.
    \end{enumerate}
  \end{ans}
  
