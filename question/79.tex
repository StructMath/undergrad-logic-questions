\emph{(اصل استقرا برای برهان در دستگاه استنتاج هیلبرت) }
	فرض کنید $P$ یک ویژگی از گزاره‌ها و $\Gamma$ مجموعه‌ای از گزاره‌ها باشد به گونه‌ای که داشته باشیم:
	\begin{enumerate}
		\item $P$ برای اصول دستگاه هیلبرت برقرار است.
		\item $P$ برای اعضای $\Gamma$ برقرار است.
		\item اگر $P$ برای $\varphi$ و $\varphi \rightarrow \psi$ برقرار باشد آن‌گاه برای $\psi$ هم برقرار است.
	\end{enumerate}
	در این صورت نشان دهید $P$ برای هر گزاره‌ای مانند $\varphi$ که $\Gamma \vdashHP \varphi$ برقرار است.
	
	\emph{(راهنمایی: از استقرا بر روی طول برهان $\varphi$ استفاده کنید.)}
	
	\quad\vspace {0.5cm}
	\begin{ans}
		از راهنمایی استفاده می‌کنیم. حالت پایه این است که طول برهان برای 
		$\varphi$
		برابر ۱ باشد.
		آنگاه یا 
		$\varphi$
		در 
		$\Gamma$
		است و یا از اصول هیلبرت است که طبق فرض می‌دانیم ویژگی برای آن برقرار است.\\
		حال به استقرا فرض کنیم که برای تمام فرمول‌هایی که از 
		$\Gamma$
		با طول برهان کمتر از k نتیجه می‌شوند،صادق است. برای فرمول دلخواه 
		$\varphi$
		که طول برهان آن برابر k است حکم را ثابت می‌کنیم.
		برهان از 
		$\Gamma$
		 به 
		$\varphi$
		را در نظر بگیریم. برای مرحله آخر این برهان ۲ حالت داریم:
		\quad\vspace{0.5cm}
		\begin{enumerate}
			\item از یکی از اصول دستگاه استفاده کرده‌ایم. که طبق فرض خاصیت برای آنان برقرار است.
			\item از یکی از اعضا 
			$\Gamma$
			استفاده شده است، که طبق فرض برقرار است. 
			\item از قائده 
			$\text{MP}$
			استفاده کرده‌ایم. پس این مرحله به صورت زیر است.
			\infer[\text{(MP)}]{\varphi}{
				\psi
				&
				\psi \to \varphi
			}
			همچنین
			$\psi$
			و
			$\psi \to \varphi$ 
			هر دو در مراحل قبلی برهان ظاهر شده‌اند. که چون طول برهان از 
			$\Gamma$
			به
			$\psi$
			و
			$\psi \to \varphi$
			کمتر از k است خاصیت برای آن‌ها برقرار است و طبق فرض سوال نتیجه ‌می‌شود که خاصیت برای 
			$varphi$
			هم برقرار است.
		\end{enumerate}
	\end{ans}	
