\emph{(اصل استقرا برای برهان در دستگاه استنتاج هیلبرت) }
	فرض کنید $P$ یک ویژگی از گزاره‌ها و $\Gamma$ مجموعه‌ای از گزاره‌ها باشد به گونه‌ای که داشته باشیم:
	\begin{enumerate}
		\item $P$ برای اصول دستگاه هیلبرت برقرار است.
		\item $P$ برای اعضای $\Gamma$ برقرار است.
		\item اگر $P$ برای $\varphi$ و $\varphi \rightarrow \psi$ برقرار باشد آن‌گاه برای $\psi$ هم برقرار است.
	\end{enumerate}
	در این صورت نشان دهید $P$ برای هر گزاره‌ای مانند $\varphi$ که $\Gamma \vdashHP \varphi$ برقرار است.
	
	\emph{(راهنمایی: از استقرا بر روی طول برهان $\varphi$ استفاده کنید.)}
	
	\quad\vspace {0.5cm}
	\begin{ans}
		از راهنمایی استفاده می‌کنیم. حالت پایه این است که طول برهان برای 
		$\varphi$
		برابر $1$ باشد.
		آنگاه یا 
		$\varphi$
		در 
		$\Gamma$
		است و یا از اصول هیلبرت است که طبق فرض‌های ۱ و ۲ می‌دانیم ویژگی $P$ برای آن برقرار است.\\
		حال به استقرا فرض کنیم که $P$ برای تمام فرمول‌هایی که از 
		$\Gamma$
		با طول برهان کمتر از $k$ نتیجه می‌شوند، صادق است. برای فرمول دلخواه 
		$\varphi$
		که طول برهان آن برابر $k$ است حکم را ثابت می‌کنیم.
		برهان از 
		$\Gamma$
		 به 
		$\varphi$
		را در نظر بگیریم. برای مرحله آخر این برهان ۲ حالت داریم:
		\quad\vspace{0.5cm}
		\begin{enumerate}
			\item از یکی از اصول دستگاه استفاده کرده‌ایم. یعنی $\varphi$ اصل است، که طبق فرض ۱ ویژگی $P$ برای آنان برقرار است.
			\item $\varphi$ یکی از اعضا 
			$\Gamma$
			است، که طبق فرض ۲ ویژگی $P$ برقرار است. 
			\item از قائدهٔ 
			\lr{MP}
			استفاده کرده‌ایم. پس دو فرمول مثل $\psi \rightarrow \varphi$ و $\psi$ نیز در برهان $\varphi$ وجود دارند.
			از آن‌جا که این دو فرمول پیش از $\varphi$ در برهان ظاهر شده‌اند، پس می‌توانیم بگویمم هرکدام یک برهان با طول کمتر از $k$ از مفروضات $\Gamma$ دارند. با استفاده از فرض استقرا می‌دانیم $P$ برای $\psi \rightarrow \varphi$ و $\psi$ برقرار است. با استفاده از فرض ۳ نتیجه می‌گیریم $P$ برای $\varphi$ برقرار است.
		\end{enumerate}
	\end{ans}	
