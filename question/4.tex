
    نشان دهید مجموعه اعداد گویا شماراست.
    \begin{ans}
        ابتدا نشان می دهیم که مجموعه دوتایی های مرتبی که هر دو عضو از 
        $N$
        آمده اند شماراست.
        \\
        تابع
        \[f:N^{2}\rightarrow N\]
        را اینگونه تعریف می کنیم:

        \[f(a,b) = \frac{(a+b)(a+b+1)}{2} + a - 1\]
        \\
        ادعا: 
        $f$
        یک به یک و پوشاست. 
        \\
        اثبات:
        ابتدا نشان می دهیم 
        $f$
        یک به یک است:
        \\
        فرض کنیم برای اعداد طبیعی
        $a,a',b,b'$
        داریم
        $f(a,b) = f(a',b')$
        آنگاه خواهیم داشت:
        \[\frac{(a+b)(a+b+1)}{2} + a - 1 = \frac{(a'+b')(a'+b'+1)}{2} + a' - 1\]
        بدون کاستن از کلیت  مسئله ((برهان خلف)) فرض می کنیم که 
        $a' > a$.
        سپس داریم:
        \[\frac{(a+b)(a+b+1)}{2} = \frac{(a'+b')(a'+b'+1)}{2} + a' - a\]
        با توجه به این که مجموع اعداد طبیعی 
        $0$
        تا
        $a+b$
        از فرمول 
        $\frac{(a+b)(a+b+1)}{2}$
        بدست می آید. می دانیم 
        $a'-a >= a'+b'+1$
        که معادل است با
        $0 >= a + b' + 1$
        که تناقض است زیرا طبق فرض ما 
        $a,b'$
        اعداد طبیعی هستند و حاصل جمع آن ها نمی تواند یک عدد منفی باشد. مشابها اگر فرض گنیم
        $a > a'$
        به نتیجه مشابه می رسیم. پس لزوما
        $a = a'$
        حال که می دانیم
        $a = a'$
        داریم
        \[f(a,b) = f(a,b')\]
        \[\frac{(a+b)(a+b+1)}{2} + a - 1 = \frac{(a+b')(a+b'+1)}{2} + a - 1\]
        \[\frac{(a+b)(a+b+1)}{2} = \frac{(a+b')(a+b'+1)}{2}\]
        \[ab + b^{2} + b = ab' + b'^{2} + b'\]
        \[a(b-b') = (b'-b)(b'+b+1)\]
        تساوی بدست آمده نشان می دهد اگر 
        $b \neq b'$
        آنگاه 
        $a < 0$
        که با فرض طبیعی بودن 
        $a$
        تناقض دارد.
        \\
        پس 
        $b = b'$

        حال که نشان دادیم 
        $f$
        یک به یک است. کافی است نشان دهیم پوشا نیز است:
        فرض کنیم 
        $z \in N$
        ما الگوریتمی برای یافتن دوتایی مرتب 
        $(x,y) \in N^{2}$
        ای ارائه می دهیم که 
        $f(x,y) = z$.

        فرض کنیم 
        $a$
        بزرگترین عدد طبیعی ای باشد که 
        $\frac{a(a+1)}{2} <= z$
        قرار می دهیم:
        \[x = z - \frac{a(a+1)}{2} + 1\]
        \[y = a - x\]
        حال داریم
        \[f(x,y) = \frac{(x+y)(x+y+1)}{2} + x - 1\]
        \[f(x,y) = \frac{a(a+1)}{2} + z - \frac{a(a+1)}{2} + 1 -1\]
        \[f(x,y) = z\]

        در نتیجه 
        $f$
        پوشاست.
        \\
        ما در اینجا نشان می دهیم که مجموعه اعداد گویای نامنفی شماراست. می دانیم:
        \[Q^{2} = \{\frac{a}{b} \ | \ a,b \in N \ \ and \ \ b \neq 0\}\]
        تابع 
        $g$
        را اینگونه تعریف می کنیم:
        \[g:Q^{+}\rightarrow N^{2}\]
        \[g(\frac{a}{b}) = (a,b)\]
        طبق تعریف، واضح است که 
        $g$
        یک به یک است.
        \\
        حال از قضیه زیر استفاده می کنیم:
        \\
        قضیه: فرض کنیم 
        $A$
        یک مجموعه و 
        $B \subseteq A$
        . اگر 
        $A$
        شمارا باشد، آنگاه 
        $B$
        نیز شماراست.
        \\
        ما نشان دادیم که 
        $N^{2}$
        شماراست.
        همچنین می دانیم
        $g(Q^{+}) \subset N^{2}$
        پس طبق قضیه ذکر شده، 
        $g(Q^{+})$
        شمارا و چون 
        $g$
        یک به یک است. 
        $Q^{+}$
        شمارا است. 

        با استدلال مشابه می توان نشان داد که
        $Q^{-}$
        نیز شمارا و در نتیجه
        $Q$
        شماراست.


    \end{ans}
