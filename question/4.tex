معمای زیر توسط ریموند اسمولیان طراحی شده است. آن را در منطق گزاره‌ها صورت‌بندی کنید و با کمک آموخته‌های خود به آن پاسخ بدهید:

\begin{quote}
به جزیره‌ای سفر کرده‌اید که برخی ساکنان آن راستگو هستند (یعنی در پاسخ به هر پرسشی پاسخ درست را بیان می‌کنند) و باقی ساکنان آن دروغگو هستند (یعنی در پاسخ به هر پرسشی نقیض پاسخ درست را بیان می‌کنند). شما به یکی از بومیان این جزیره برمی‌خورید و از او می‌پرسید: «آیا در جزیره می‌شود طلا پیدا کرد؟» او تنها پاسخ می‌دهد: «اگر من راستگو باشم، در جزیره طلا هست.» آیا می‌شود از این نتیجه گرفت که آیا در جزیره طلا هست یا نه و آیا این شخص راستگو است یا دروغگو؟
\end{quote}

\begin{ans}
  فرض کنید $T$ نمایانگر این گزاره باشد که شخص بومی راستگو است. (در این صورت $\neg T$ نمایانگر دروغگو بودن بومی خواهد بود.) همچنین فرض کنید $G$ نمایانگر این گزاره باشد که در جزیره طلا هست. حال می‌دانیم اگر شخص راستگو باشد گفتهٔ او صحیح و اگر شخص دروغگو باشد نقیض گفتهٔ او صحیح است؛ به عبارت دیگر:
  $$
  (T\to (T\to G))\wedge(\neg T\to \neg(T\to G))
  $$
  حال می‌توان به‌سادگی بررسی کرد که از میان چهار مقدار متفاوتی که می‌توان برای اتم‌های $T$ و $G$ در نظر گرفت، تنها در صورتی گزارهٔ بالا صادق است که $T$ و $G$ هر دو صادق باشند. بنابراین شخص بومی راستگو است و در جزیره طلا هست.
\end{ans}
