معمای زیر توسط ریموند اسمولیان طراحی شده است. آن را در منطق گزاره‌ها صورت‌بندی کنید و با کمک آموخته‌های خود به آن پاسخ بدهید:
\begin{quote}
به جزیره‌ای سفر کرده‌اید که برخی ساکنان آن راستگو هستند (یعنی در پاسخ به هر پرسشی پاسخ درست را بیان می‌کنند) و باقی ساکنان آن دروغگو هستند (یعنی در پاسخ به هر پرسشی نقیض پاسخ درست را بیان می‌کنند). شما به یکی از بومیان این جزیره برمی‌خورید و از او می‌پرسید: «آیا در جزیره می‌شود طلا پیدا کرد؟» او تنها پاسخ می‌دهد: «اگر من راستگو باشم، در جزیره طلا هست.» آیا می‌شود از این نتیجه گرفت که آیا در جزیره طلا هست یا نه و آیا این شخص راستگو است یا دروغگو؟
\end{quote}\quad
\begin{ans}
فرض کنید $T$ نمایانگر راستگو بودن شخص و $G$ نمایانگر وجود طلا در جزیره باشد. با توجه به مفروضات مسئله می‌دانیم
$$
(T\to(T\to G))\wedge(\neg T\to\neg(T\to G))
$$
با رسم جدول ارزش عبارت بالا متوجه می‌شویم تنها ارزش‌دهی ممکن به اتم‌های $T$ و $G$ که عبارت بالا را صادق کند ارزش‌دهی‌ای است که به هر دو مقدار صادق نسبت دهد. بنابراین شخص بومی راستگو است و در جزیره طلا وجود دارد.
\end{ans}
