فرض کنید رابطهٔ
$\leq$
روی
$\prop$
را چنین تعریف کنیم که
$\varphi\leq\psi$
اگر
$\vdash\varphi\to\psi$.
\begin{enumerate}[label=(\alph*)]
    \item 
    نشان دهید
    $\leq$
    بازتابی و تراگذری است.%\footnote{
    %     فاصلهٔ
    %     $\leq$
    %     از ترتیب جزئی بودن پادمتقارن بودن است. اگر با یکی فرض کردن فرمول‌هایی که هر دو یکدیگر را نتیجه می‌دهند این خاصیت را نیز به
    %     $\leq$
    %     اضافه کنیم به جبر لیندنباوم--تارسکی منطق کلاسیک می‌رسیم که یک جبر بولی
    %     (\lr{Boolean Algebra})
    %     است.
    % }
    \item اگر
    $\Phi$
    مجموعه‌ای از فرمول‌ها باشد تعریف می‌کنیم
    $\upset\Phi=\{\psi\in\prop : (\forall\varphi\in\Phi)\ \varphi\leq\psi\}$.
    نشان دهید برای هر
    $\Phi$
    متناهی، مجموعهٔ
    $\upset\Phi$
    اینفیمم دارد.
    \item اگر
    $\Phi$
    مجموعه‌ای از فرمول‌ها باشد تعریف می‌کنیم
    $\downset\Phi=\{\psi\in\prop : (\forall\varphi\in\Phi)\ \psi\leq\varphi\}$.
    نشان دهید برای هر
    $\Phi$
    متناهی، مجموعهٔ
    $\downset\Phi$
    سوپریمم دارد.
    \item نشان دهید رابطهٔ ترتیب
    $\leq$
    روی مجموعهٔ
    $\prop$
    عنصر ماکسیمم و مینیمم دارد.
    (نماینده‌ای دلخواه از مجموعهٔ عناصر ماکسیمم را $1$ و نماینده‌ای دلخواه از مجموعهٔ عناصر مینیمم را $0$ می‌نامیم.)
    \item به
    $\psi$
    متمم
    $\varphi$
    می‌گوییم اگر
    $1$
    اینفیمم
    $\upset\{\varphi,\psi\}$
    و
    $0$
    سوپریمم
    $\downset\{\varphi,\psi\}$
    باشد. نشان دهید هر فرمول دارای متمم است.
    \item نشان دهید برای هر دو فرمول
    $\varphi$
    و
    $\psi$
    بزرگترین فرمول
    $\sigma$
    (طبق ترتیب $\leq$)
    هست که
    $\psi$
    از هر عضو
    $\downset\{\varphi,\sigma\}$
    بزرگتر باشد.
\end{enumerate}\quad\vspace{-9mm}
\begin{ans}
    \begin{enumerate}[label=(\alph*)]
        \item\label{q60:a} برای نشان‌دادن بازتابی بودن $\leq$، کافی است برای هر $\varphi \in \Phi$ ثابت کنیم $\vdash \varphi \rightarrow \varphi$، که به راحتی و با یک بار استفاده از معرفی شرطی اثبات می‌شود.
        برای نشان‌دادن تراگذری بودن آن، فرض کنید
        $\varphi\leq\psi$
        و
        $\psi\leq\sigma$.
        این یعنی
        $\vdash\varphi\to\psi$
        و
        $\vdash\psi\to\sigma$، که با استفاده از قضیهٔ استنتاج، یعنی $\varphi\vdash\psi$ و $\psi\vdash\sigma$. در نتیجه $\varphi\vdash\sigma$ و با استفادهٔ مجدد از قضیهٔ استنتاج، $\vdash\varphi\rightarrow\sigma$. در نتیجه $\varphi\leq\sigma$.
        
        در ادامه، بدون این که باز به صراحت از قضیهٔ استنتاج نام ببریم، از آن استفاده کرده و همه‌جا $\varphi \le \psi$ را معادل با $\varphi \vdash \psi$ می‌گیریم.

        \item\label{q60:b}
        فرض کنید $\Phi = \{\varphi_1, \dots, \varphi_n\}$.
        ادعا می‌کنیم $\bigvee\Phi = \varphi_1 \vee \dots \vee \varphi_n$ اینفیمم یا بزرگترین کران پایین $\upset\Phi$ است. ابتدا نشان می‌دهیم $\bigvee\Phi \vdash \psi$.
        یک عضو دلخواه از $\upset \Phi$ مانند $\psi$ را در نظر بگیرید. از تعریف $\upset\Phi$ می‌دانیم که برای هر عضو $\Phi$ مانند $\varphi_i$ داریم $\varphi_i \vdash \psi$. با استقرا روی $n$ نشان می‌دهیم $\bigvee\Phi \vdash \psi$. برای پایه‌ی استقرا، یعنی در حالتی که $n = 1$ باشد، داریم $\bigvee\Phi = \varphi_1$ و حکم از تعریف $\upset\Phi$ نتیجه می‌شود. حال برای گام استقرا، بنا به فرض استقرا می‌توان فرض کرد حکم برای $n - 1$ برقرار است، یعنی داریم $\varphi_1 \vee \dots \vee \varphi_{n-1} \vdash \psi$. درخت برهان زیر را برای $\bigvee \Phi \vdash \psi$ می‌سازیم:
        \begin{LTR}\begin{prooftree}
            \AXC{$\varphi_1 \vee \dots \vee \varphi_n$}
            \AXC{{\emph{\rl{فرض استقرا}}}}
            \noLine\UIC{$[\varphi_1 \vee \dots \vee \varphi_{n-1}]^1$}
            \noLine\UIC{$\vdots$}
            \noLine\UIC{$\psi$}
            \AXC{{\emph{\rl{تعریف $\upset\Phi$}}}}
            \noLine\UIC{$[\varphi_n]^1$}
            \noLine\UIC{$\vdots$}
            \noLine\UIC{$\psi$}
            \veeE{1}{$\psi$}
        \end{prooftree}\end{LTR}
        در نتیجه $\bigvee\Phi$ یک کران پایین برای $\upset\Phi$ است. برای این که بگوییم $\bigvee\Phi$ بزرگترین کران پایین برای $\upset\Phi$ است، کافی است نشان دهیم $\bigvee\Phi \in \upset\Phi$، یعنی برای هر $\varphi \in \Phi$ بگوییم $\varphi \vdash \bigvee\Phi$، که با استفاده از استفاده از قاعده‌ی معرفی ترکیب فصلی و جابه‌جایی ترکیب فصلی ساده است.

        \item\label{q60:c}
        فرض کنید $\Phi = \{\varphi_1, \dots, \varphi_n\}$.
        ادعا می‌کنیم $\bigwedge\Phi = \varphi_1 \wedge \dots \wedge \varphi_n$ سوپریمم یا کوچکترین کران بالای $\downset\Phi$ است. اثبات مشابه بخش \ref{q60:b} است. ابتدا با استقرا روی $n$ نشان می‌دهیم $\bigwedge\Phi$ از هر عضو $\downset\Phi$ مانند $\psi$ بزرگتر است. در این‌جا نیز می‌دانیم بنا به تعریف $\downset\Phi$، برای هر $\varphi_i$ داریم $\psi \vdash \varphi_i$. در حالت $n = 1$، داریم $\bigwedge\Phi = \varphi_1$ و حکم از تعریف $\downset\Phi$ نتیجه می‌شود. در گام استقرا، بنا به فرض استقرا داریم $\psi \vdash \varphi_1 \wedge \dots \wedge \varphi_{n-1}$. برای اثبات $\psi \vdash \bigwedge\Phi$ درخت برهان زیر را می‌سازیم:
        \begin{LTR}\begin{prooftree}
            \AXC{{\emph{\rl{فرض استقرا}}}}    
            \noLine\UIC{$\psi$}
            \noLine\UIC{$\vdots$}
            \noLine\UIC{$\varphi_1 \wedge \dots \wedge \varphi_{n-1}$}
            \AXC{{\emph{\rl{تعریف $\downset\Phi$}}}}
            \noLine\UIC{$\psi$}
            \noLine\UIC{$\vdots$}
            \noLine\UIC{$\varphi_n$}
            \wedI{$\varphi_1 \wedge \dots \wedge \varphi_n$}
        \end{prooftree}\end{LTR}
        پس $\bigwedge \Phi$ یک کران بالا برای $\downset\Phi$ است. به طریق مشابه بخش \ref{q60:b} می‌توان نشان داد که $\bigwedge\Phi \in \downset\Phi$ و در نتیجه $\bigwedge\Phi$ کوچکترین کران بالا برای $\downset\Phi$ است.

        \item\label{q60:d}
        بنا به نتایج بخش‌های \ref{q60:b} و \ref{q60:c}، می‌دانیم مجموعه‌ی $\upset\emptyset$ اینفیمم و مجموعه‌ی $\downset\emptyset$ سوپریمم دارد. این دو فرمول را به ترتیب $0$ و $1$ می‌نامیم.
        بنا به تعریف، و به انتفای مقدم، داریم $\upset\emptyset = \downset\emptyset = \prop$. در نتیجه $0$ و $1$ به ترتیب مینیمم و ماکسیمم ترتیب تعریف شده روی $\prop$ هستند.

        \item\label{q60:e}
        به ازای هر $\varphi$، ادعا می‌کنیم $\neg \varphi$ متمم مورد نظر با دو خاصیت ذکرشده است. بنا به تعریف، $0$ و $1$ عناصر مینیمم و ماکسیمم ترتیب هستند. بنابراین، کافی است نشان دهیم اینفیمم $\upset\{\varphi, \neg\varphi\}$ از هر فرمول دلخواه بزرگتر، و سوپریمم $\downset\{\varphi, \neg\varphi\}$ از هر فرمول دلخواه کوچکتر است. در این‌جا تنها مورد اوّل را اثبات می‌کنیم، چرا که مورد دوم نیز مشابه مورد اوّل اثبات می‌شود. 
        اینفیمم $\upset\{\varphi, \neg\varphi\}$ را $I$ بنامید. بنا به تعریف، برای هر $\psi \in \upset\{\varphi, \neg\varphi\}$ داریم $\varphi \vdash \psi$ و $\neg\varphi \vdash \psi$. با استفاده از قاعده‌ی حذف ترکیب فصلی داریم $\varphi \vee \neg\varphi \vdash \psi$ و درنتیجه $\varphi \vee \neg\varphi$ یک کران پایین برای $\upset\{\varphi, \neg\varphi\}$ است. پس $\varphi \vee \neg\varphi \vdash I$. از طرفی می‌دانیم $\vdash \varphi \vee \neg\varphi$، در نتیجه $\vdash I$. به راحتی از قضیهٔ تضعیف می‌توان برای هر فرمول دلخواه $\theta$ نتیجه گرفت $\theta \vdash I$. پس $I$ عضو ماکسیمم ترتیب است. مورد دوم نیز به همین شکل و به کمک قاعدهٔ معرفی ترکیب عطفی و حذف تناقض اثبات می‌شود.

        \item\label{q60:f}
        ادعا می‌کنیم برای هر $\varphi$ و $\psi$ دلخواه، $\varphi \rightarrow \psi$ همان $\sigma$ با شرایط ذکرشده است. یعنی باید نشان دهیم اوّلاً $\psi$ کران بالایی برای $\downset\{\varphi, \varphi \rightarrow \psi\}$ است، و ثانیاً برای هر $\sigma$ که $\psi$ کران بالایی برای $\downset\{\varphi, \sigma\}$ باشد، $\sigma \vdash \varphi \rightarrow \psi$. برای اثبات مورد اول، فرض کنید $\theta \in \downset\{\varphi, \varphi \rightarrow \psi\}$، یعنی $\theta \vdash \varphi$ و $\theta \vdash \varphi \rightarrow \psi$. با استفاده از قاعدهٔ حذف شرطی می‌توان نشان داد $\theta \vdash \psi$. پس $\psi$ از هر عضو $\downset\{\varphi, \varphi \rightarrow \psi\}$ بزرگتر است. برای اثبات مورد دوم، فرض کنید برای یک $\sigma$ دلخواه، $\psi$ از هر عضو $\downset\{\varphi, \sigma\}$ بزرگتر باشد. می‌دانیم $\varphi \wedge \sigma \in \downset\{\varphi, \varphi \rightarrow \psi\}$، در نتیجه $\varphi \wedge \sigma \vdash \psi$. از طرفی با استفاده از قاعدهٔ معرفی ترکیب عطفی داریم $\varphi, \sigma \vdash \varphi \wedge \sigma$، و در نتیجه $\varphi, \sigma \vdash \psi$. به کمک قاعدهٔ معرفی شرطی خوایم داشت $\sigma \vdash \varphi \rightarrow \psi$.
    \end{enumerate}
\end{ans}
