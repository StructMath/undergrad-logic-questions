فرض کنید رابطهٔ
$\leq$
روی
$\prop$
را چنین تعریف کنیم که
$\varphi\leq\psi$
اگر
$\vdash\varphi\to\psi$.
\begin{enumerate}[label=(\alph*)]
   \item 
        وا
        فرض کنید
        $\varphi\leq\psi$
        و
        $\psi\leq\sigma$.
        این یعنی
        $\vdash\varphi\to\psi$
        و
        $\vdash\psi\to\sigma$.

        \item فرمول 
        $\bigvee\Phi$
        را در نظر بگیرید. 
        برای هر 
        $\varphi\in\Phi$
        داریم 
        $\vdash\varphi\to\bigvee\Phi$
        <<این نتیجه از آنجا می آید که اگر 
        $\vdash\psi\to\iota$
        آنگاه 
        $\vdash\psi\to(\iota\vee\kappa)$
        >>
        پس 
        $\bigvee\Phi\in\uparrow\Phi$
        .فرض کنیم که 
        $\psi$
        یک کران پایین برای 
        $\uparrow\Phi$
        باشد. پس برای هر 
        $\varphi\in\uparrow\Phi$
        داریم
        $\vdash\psi\to\varphi$
        از طرفی چون
        $\uparrow\Phi\subseteq\Phi$
        داریم 
        $\vdash\psi\to\bigvee\Phi$
        .پس 
        $\bigvee\Phi$
        اینفیمم مد نظر است.

        \item فرمول
        $\bigwedge\Phi$
        را در نظر بگیرید. برای هر 
        $\varphi\in\Phi$
        داریم
        $\vdash\bigwedge\Phi\to\varphi$
        همچنین برای هر 
        $\psi\in\downarrow\Phi$،
        $\vdash\psi\to\bigwedge\Phi$.
        پس 
        $\bigwedge\Phi\in\downarrow\Phi$
        و
        $\bigwedge\Phi$
        یک کران بالا برای 
        $\downarrow\Phi$
        است. فرض کنید 
        $\iota\in\prop$
        یک کران بالا برای 
        $\downarrow\Phi$
        است. پس به ازای هر 
        $\sigma\in\downarrow\Phi$
        داریم 
        $\vdash\sigma\to\iota$.
        چون 
        $\downarrow\Phi\subseteq\Phi$،
        نتیجه می گیریم
        $\vdash\bigwedge\Phi\to\iota$
        <<این مطلب از این موضوع بدست می آید که 
        اگر 
        $\vdash\kappa_{1}\to\gamma$،
        آنگاه برای
        $\kappa_{2}\in\prop$
        دلخواه داریم
        $\vdash((\kappa_{1}\wedge\kappa_{2})\to\gamma)$.
        >>
        در نتیجه 
        $\bigwedge\Phi$
        کوچکترین کران بالای 
        $\downarrow\Phi$
        است.

        \item قرار دهید
        $\Phi=\emptyset$.
        \[\uparrow\Phi = \{\psi\in \ \prop :(\forall\varphi\in\Phi)\varphi\leq\psi\}\]
        \[\downarrow\Phi = \{\psi\in \ \prop :(\forall\varphi\in\Phi)\psi\leq\varphi\}\]
        با توجه به این که 
        $\Phi$
        متناهی است. طبق ب و ج دو مجموعه بالا به ترتیب دارای اینفیمم و سوپریمم هستند. 
        اینفیمم اینجا ماکزیمم ما و سوپریمم ما اینجا مینیمم ما است. 

        \item ادعا می کنیم که 
        $\neg\varphi$
        متمم
        $\varphi$
        است. فرض کنیم وجود دارد 
        $\phi\in\prop$
        که 
        $\vdash\varphi\to\phi$
        و
        $\vdash\varphi\to\neg\phi$.
        پس 
        $\vdash\varphi\to(\phi\wedge\neg\phi)$.
        این معادل است با 
        $\vdash\phi\to\bot$.
        چون 
        $\bot$
        مینیمم 
        $\prop$
        است 
        << طبق ج و د >>
        ،سوپریمم
        $\downarrow\{\varphi,\neg\varphi\}$
        صفر است.
        \\
        \\
        مشابها فرض کنیم برای یک 
        $\iota\in\prop$
        داریم
        $\vdash\varphi\to\iota$
        و
        $\vdash\neg\varphi\to\iota$.
        از این دو بدست می آید
        $\vdash(\neg\varphi\vee\varphi)\to\iota$.
        چون 
        $\vdash(\neg\varphi\vee\varphi)$
        و این نکته که به ازای هر 
        $\gamma,\eta\in\prop$
        اگر 
        $\vdash\gamma$،
        آنگاه 
        $\vdash\eta\to\gamma$،
        و این موضوع که 
        $(\gamma\to\psi),(\psi\to\theta)\vdash\gamma\to\theta$
        نتیجه می گیریم که برای هر 
        $\eta\in\prop$،
        $\vdash\eta\to\iota$.
        پس 
        $\iota$
        ماکزیمم 
        $\prop$
        است.

        \item بجای
        $\sigma$
        قرار دهید
        $0$.
        در این حالت سوپریمم
        $\downarrow\{\varphi,0\}$
        می شود
        $0$
        که از هر عضو 
        $\prop$
        کوچکتر است.

    \end{enumerate}
\end{ans}
