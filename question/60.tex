فرض کنید رابطهٔ
$\leq$
روی
$\prop$
را چنین تعریف کنیم که
$\varphi\leq\psi$
اگر
$\vdash\varphi\to\psi$.
\begin{enumerate}[label=(\alph*)]
    \item نشان دهید
    $\leq$
    بازتابی و تراگذری است.%\footnote{
    %     فاصلهٔ
    %     $\leq$
    %     از ترتیب جزئی بودن پادمتقارن بودن است. اگر با یکی فرض کردن فرمول‌هایی که هر دو یکدیگر را نتیجه می‌دهند این خاصیت را نیز به
    %     $\leq$
    %     اضافه کنیم به جبر لیندنباوم--تارسکی منطق کلاسیک می‌رسیم که یک جبر بولی
    %     (\lr{Boolean Algebra})
    %     است.
    % }
    \item اگر
    $\Phi$
    مجموعه‌ای از فرمول‌ها باشد تعریف می‌کنیم
    $\uparrow\Phi=\{\psi\in\prop : (\forall\varphi\in\Phi)\ \varphi\leq\psi\}$.
    نشان دهید برای هر
    $\Phi$
    متناهی، مجموعهٔ
    $\uparrow\Phi$
    اینفیمم دارد.
    \item اگر
    $\Phi$
    مجموعه‌ای از فرمول‌ها باشد تعریف می‌کنیم
    $\downarrow\Phi=\{\psi\in\prop : (\forall\varphi\in\Phi)\ \psi\leq\varphi\}$.
    نشان دهید برای هر
    $\Phi$
    متناهی، مجموعهٔ
    $\downarrow\Phi$
    سوپریمم دارد.
    \item نشان دهید رابطهٔ ترتیب
    $\leq$
    روی مجموعهٔ
    $\prop$
    عنصر ماکسیمم و مینیمم دارد.
    (نماینده‌ای دلخواه از مجموعهٔ عناصر ماکسیمم را $1$ و نماینده‌ای دلخواه از مجموعهٔ عناصر مینیمم را $0$ می‌نامیم.)
    \item به
    $\psi$
    متمم
    $\varphi$
    می‌گوییم اگر
    $1$
    اینفیمم
    $\uparrow\{\varphi,\psi\}$
    و
    $0$
    سوپریمم
    $\downarrow\{\varphi,\psi\}$
    باشد. نشان دهید هر فرمول دارای متمم است.
    \item نشان دهید برای هر دو فرمول
    $\varphi$
    و
    $\psi$
    بزرگترین فرمول
    $\sigma$
    (طبق ترتیب $\leq$)
    هست که
    $\psi$
    از هر عضو
    $\downarrow\{\varphi,\sigma\}$
    بزرگتر باشد.
\end{enumerate}\quad\vspace{-9mm}
\begin{ans}
    \begin{enumerate}[label=(\alph*)]
        \item\label{q60:a} برای نشان‌دادن بازتابی بودن $\leq$، کافی است برای هر $\varphi \in \Phi$ ثابت کنیم $\vdash \varphi \rightarrow \varphi$، که به راحتی و با یک بار استفاده از معرفی شرطی اثبات می‌شود.
        برای نشان‌دادن تراگذری بودن آن، فرض کنید
        $\varphi\leq\psi$
        و
        $\psi\leq\sigma$.
        این یعنی
        $\vdash\varphi\to\psi$
        و
        $\vdash\psi\to\sigma$، که با استفاده از قضیهٔ استنتاج، یعنی $\varphi\vdash\psi$ و $\psi\vdash\sigma$. در نتیجه $\varphi\vdash\sigma$ و با استفادهٔ مجدد از قضیهٔ استنتاج، $\vdash\varphi\rightarrow\sigma$. در نتیجه $\varphi\leq\sigma$.
        
        \item\label{q60:b} ادعا می‌کنیم برای
        $\uparrow\Phi$
        متناهی عطف تمام اعضای
        $\Phi$
        (که آن را با
        $\bigwedge\Phi$
        نشان می‌دهیم)
        اینفیمم (یا بزرگترین کران پایین) این مجموعه است.
        فرض کنید
        $\psi\in\uparrow\Phi$.
        این یعنی برای هر
        $\varphi\in\Phi$
        داریم
        $\vdash\varphi\to\psi$.
        حال برای نشان دادن
        $\vdash\bigwedge\Phi\to\psi$
        کافی است
        $\bigwedge\Phi$
        را فرض بگیریم، با استفاده از قاعدهٔ حذف عطف یکی از اعضای
        $\Phi$
        مثل
        $\varphi$
        را نتیجه بگیریم، و سپس با نوشتن استنتاجی که
        $\varphi\to\psi$
        را اثبات می‌کند و قاعدهٔ حذف شرطی
        $\psi$
        را نتیجه بگیریم و سپس با قاعدهٔ معرفی شرطی مقدمهٔ
        $\bigwedge\Phi$
        را حذف کنیم و نتیجه بگیریم
        $\bigwedge\Phi\to\psi$. پس برای هر ‌$\psi\in\uparrow\Phi$ داریم $\bigwedge\Phi\leq\psi$، و در نتیجه $\bigwedge\Phi$ یک کران پایین برای $\uparrow\Phi$ است. برای نشان دادن بزرگترین کران پایین بودن آن نیز فرض می‌کنیم $\sigma$ کران پایین دیگری برای $\uparrow\Phi$ باشد و نشان می‌دهیم $\sigma\leq\bigwedge\Phi$.
        
        \item\label{q60:c} 
        به ازای هر 
        $\psi,\varphi\in\downarrow\Phi$
        داریم
        $\psi\leq\varphi$
        و
        $\varphi\leq\psi$
        <<این موضوع از تعریف 
        $\downarrow\Phi$
        نتیجه می‌شود. >>
        پس هر عضو 
        $\downarrow\Phi$
        می‌تواند یک سوپریمم باشد.
        
        \item\label{q60:d}  
        فرض کنید 
        $\varphi$
        یک توتولوژی باشد. 
        می‌دانیم به ازای هر 
        $\psi\in\prop$
        داریم
        $\models(\psi\to\varphi)$
        همچنین طبق قضیه تمامیت برای منطق گزاره‌ای داریم:
        $\vdash(\psi\to\varphi)$
        در نتیجه هر توتولوژی یا
        $\top$
        ماکزیمم 
        $\prop$
        با رابطه ترتیب 
        $\leq$
        است.
        به طور مشابه 
        $\bot$
        را در نظر بگیرید. 
        می‌دانیم برای هر 
        $\psi\in\prop$
        داریم 
        $\models\bot\to\psi$
        و مجدد طبق قضیه تمامیت
        $\vdash(\bot\to\psi)$
        در نتیجه 
        $\bot$
        مینیمم 
        $\prop$
        با رابطه ترتیب
        $\leq$
        است.

        \item\label{q60:e} 
        برای فرمول 
        $\varphi\in\prop$
        فرمول 
        $\neg\varphi$
        را در نظر بگیرید. 
        فرض کنیم که برای 
        $\gamma\in\Phi$
        داریم 
        \[\models\varphi\to\gamma\]
        \[\models\neg\varphi\to\gamma\]
        آنگاه می‌توانیم نتیجه بگیریم که 
        $\models(\varphi\vee\neg\varphi)\to\gamma$
        \\
        چون 
        $(\varphi\vee\neg\varphi)$
        یک توتولوژی است. 
        $\gamma$
        نیز باید توتولوژی باشد. و طبق \ref{q60:d} می‌دانیم که 
        $\gamma$
        ماکزیمم
        $\Phi$
        است.
        \\
        از طرف دیگر می‌دانیم به ازای هر 
        $\iota\in\prop$
        اگر 
        $\vdash\iota\to\varphi$
        و
        $\vdash\iota\to\neg\varphi$
        آنگاه
        $\vdash\iota\to\bot$
        در نتیجه سوپریمم نیز اینجا 
        $\bot$
        خواهد بود.

        \item\label{q60:f} 
        اگر بجای
        $\sigma$
        متمم 
        $\varphi$
        را قرار دهیم به ازای هر 
        $\gamma\in\prop$
        خواهیم داشت:
        \\
        اگر 
        $\gamma\in\downarrow\{\varphi,\neg\varphi\}$
        آنگاه
        $\vdash\gamma\to\bot$
        \\
        در نتیجه 
        $\gamma\leq\psi$
    \end{enumerate}
\end{ans}
