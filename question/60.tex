فرض کنید رابطهٔ
$\leq$
روی
$\prop$
را چنین تعریف کنیم که
$\varphi\leq\psi$
اگر
$\vdash\varphi\to\psi$.
\begin{enumerate}[label=(\alph*)]
    \item نشان دهید
    $\leq$
    بازتابی و تراگذری است.%\footnote{
    %     فاصلهٔ
    %     $\leq$
    %     از ترتیب جزئی بودن پادمتقارن بودن است. اگر با یکی فرض کردن فرمول‌هایی که هر دو یکدیگر را نتیجه می‌دهند این خاصیت را نیز به
    %     $\leq$
    %     اضافه کنیم به جبر لیندنباوم--تارسکی منطق کلاسیک می‌رسیم که یک جبر بولی
    %     (\lr{Boolean Algebra})
    %     است.
    % }
    \item اگر
    $\Phi$
    مجموعه‌ای از فرمول‌ها باشد تعریف می‌کنیم
    $\uparrow\Phi=\{\psi\in\prop : (\forall\varphi\in\Phi)\ \varphi\leq\psi\}$.
    نشان دهید برای هر
    $\Phi$
    متناهی، مجموعهٔ
    $\uparrow\Phi$
    اینفیمم دارد.
    \item اگر
    $\Phi$
    مجموعه‌ای از فرمول‌ها باشد تعریف می‌کنیم
    $\downarrow\Phi=\{\psi\in\prop : (\forall\varphi\in\Phi)\ \psi\leq\varphi\}$.
    نشان دهید برای هر
    $\Phi$
    متناهی، مجموعهٔ
    $\downarrow\Phi$
    سوپریمم دارد.
    \item نشان دهید رابطهٔ ترتیب
    $\leq$
    روی مجموعهٔ
    $\prop$
    عنصر ماکسیمم و مینیمم دارد.
    (نماینده‌ای دلخواه از مجموعهٔ عناصر ماکسیمم را $1$ و نماینده‌ای دلخواه از مجموعهٔ عناصر مینیمم را $0$ می‌نامیم.)
    \item به
    $\psi$
    متمم
    $\varphi$
    می‌گوییم اگر
    $1$
    اینفیمم
    $\uparrow\{\varphi,\psi\}$
    و
    $0$
    سوپریمم
    $\downarrow\{\varphi,\psi\}$
    باشد. نشان دهید هر فرمول دارای متمم است.
    \item نشان دهید برای هر دو فرمول
    $\varphi$
    و
    $\psi$
    بزرگترین فرمول
    $\sigma$
    (طبق ترتیب $\leq$)
    هست که
    $\psi$
    از هر عضو
    $\downarrow\{\varphi,\sigma\}$
    بزرگتر باشد.
\end{enumerate}\quad\vspace{-9mm}
%ANS Incomplete
\begin{ans}
    \begin{enumerate}[label=(\alph*)]
        \item 
        وا
        فرض کنید
        $\varphi\leq\psi$
        و
        $\psi\leq\sigma$.
        این یعنی
        $\vdash\varphi\to\psi$
        و
        $\vdash\psi\to\sigma$.
        \item ادعا می‌کنیم برای
        $\uparrow\Phi$
        متناهی عطف تمام اعضای
        $\Phi$
        (که با
        $\bigwedge\Phi$
        آن را نشان می‌دهیم)
        اینفیمم این مجموعه است.
        فرض کنید
        $\psi\in\uparrow\Phi$.
        این یعنی برای هر
        $\varphi\in\Phi$
        داریم
        $\vdash\varphi\to\psi$.
        حال اگر برای ساختن استنتاجی برای
        $\bigwedge\Phi\to\psi$
        کافی است
        $\bigwedge\Phi$
        را فرض بگیریم، با استفاده از قاعدهٔ حذف عطف یکی از اعضای
        $\Phi$
        مثل
        $\varphi$
        را نتیجه بگیریم، و سپس با نوشتن استنتاجی که
        $\varphi\to\psi$
        را اثبات می‌کند و قاعدهٔ حذف عطف
        $\psi$
        را نتیجه بگیریم و سپس با قاعدهٔ حذف عطف مقدمهٔ
        $\bigwedge\Phi$
        را حذف کنیم و نتیجه بگیریم
        $\bigwedge\Phi\to\psi$.
    \end{enumerate}
\end{ans}