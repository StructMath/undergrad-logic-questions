فرض کنید $T_0 \varsubsetneq T_1 \varsubsetneq \dots$ دنباله‌ای از نظریه‌های مرتبهٔ اوّل و $\Delta$ مجموعه‌ای متناهی از جملات مرتبهٔ اوّل باشد. قرار دهید $T = \bigcup_{i=0}^\infty T_i$. نشان دهید
\begin{enumerate}[label=(\alph*)]
  \item $T$ یک نظریه است.
  ~\marginpar[left]{\textbf{(۵ نمره)}}
  \item یا جمله‌ای مانند $\sigma$ در $\Delta$ وجود دارد که $T \not\vdash \sigma$، یا جمله‌ای مانند $\sigma$ در $T$ وجود دارد که $\Delta \not\vdash \sigma$.
  ~\marginpar[left]{\textbf{(۱۵ نمره)}}
\end{enumerate}\quad
\begin{ans}
  \begin{enumerate}[label=(\alph*)]
    \item فرض کنید $T\vdash\varphi$.
    با توجه به آنکه مقدمات به‌کاررفته در استنتاج $\varphi$ متناهی است $n$ ای وجود دارد که تمام مقدمات استنتاج
    $\varphi$
    در
    $T_n$
    هستند. بنابراین
    $T_n\vdash\varphi$
    و طبق نظریه بودن $T_n$
    داریم
    $\varphi\in T_n\subset T$.
    \item کافی است ثابت کنیم اگر برای هر
    $\delta\in\Delta$ داشته باشیم
    $T\vdash\delta$،
    وجود دارد
    $\sigma\in T$
    چنانکه
    $\Delta\not\vdash\sigma$. مشابه با بخش قبل استدلال می‌کنیم وجود دارد $n$ چنانکه برای هر
    $\delta\in\Delta$
    داریم
    $T_n\vdash\delta$.
    حال $\sigma\in T_{n+1}\setminus T_n\subset T$
    را در نظر بگیرید. اگر
    $\Delta\vdash\sigma$
    خواهیم داشت
    $T_n\vdash\sigma$
    و از آنجا که
    $T_n$
    نظریه است،
    $\sigma\in T_n$
    که تناقض است.
  \end{enumerate}
\end{ans}