فرض کنید
$A(p_1,\ldots,p_n)$
یک فرمول و $v$ یک ارزیاب باشد. فرمول‌های $q_1$، \ldots، $q_n$ و فرمول $B$ را چنین تعریف می‌کنیم:
\begin{itemize}[label={--}]
\item
برای هر $1\leq i\leq n$، اگر $v(p_i)=T$ آنگاه $q_i=p_i$ و در غیر این صورت $q_i=\neg q_i$.

\item
اگر $v(A)=T$ آنگاه $B=A$ و در غیر این صورت $B=\neg A$.
\end{itemize}
ثابت کنید
$\{q_1,\ldots,q_n\}\vdash B$.
\begin{ans}
  فرض کنید $u$ ارزیابی باشد که
  $\{q_1,\ldots,q_n\}$
  را ارضا کند. واضح است که به ازای هر  $1\leq i\leq n$،
  $v(p_i)=u(p_i)$.
  بنابراین از آنجا که ارزش‌صدق هر فرمول طبق هر ارزیاب تنها به فرمول‌های واقع‌شده در آن فرمول وابسته است داریم
  $v(B)=u(B)$.
  اما واضح است که $v(B)=T$ و بنابراین $u(B)=T$. از آنجا که ارزیاب $u$ ارزیاب دلخواهی بود که
  $\{q_1,\ldots,q_n\}$
  را ارضا کند داریم
  $\{q_1,\ldots,q_n\}\vDash B$.
  حال طبق تمامیت داریم
  $\{q_1,\ldots,q_n\}\vdash B$.
\end{ans}