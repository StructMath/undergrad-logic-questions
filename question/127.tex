~\marginpar[left]{\textbf{(۱۵ نمره)}}
فرض کنید $\mathfrak{R} = \langle \mathbb{R}, <, +, -, \cdot, 0, 1 \rangle$ میدان مرتّب اعداد حقیقی باشد. نشان دهید ساختار $\mathfrak{R}'$ موجود است به طوری که
\begin{enumerate}[label=(\alph*)]
  \item $\mathfrak{R}$ و $\mathfrak{R}'$ هم‌ارز مقدماتی هستند. ($\mathfrak{R} \equiv \mathfrak{R}'$)
  \item عضوی مانند $\varepsilon$ در عالم سخن $\mathfrak{R}'$ موجود است به طوری که $0 < \varepsilon$ و برای هر $n \in \mathbb{N}$ داریم $\underbracket[.5pt][5pt]{\varepsilon + \dots + \varepsilon}_{\text{ بار} n} < 1$.
\end{enumerate}\quad
\begin{ans}
  به زبان ثابت $c$ را اضافه کنید و قرار دهید
  $$
  \Gamma=Th(\mathfrak{R}) \cup \{c>0, c<1, c+c<1, c+c+c<1,\cdots\}
  $$
  اگر $\Gamma_0$ زیرمجموعه‌ای متناهی از $\Gamma$ باشد قرار دهید
  $$N=\max\{n|\underbracket[.5pt][5pt]{c + \dots + c}_{\text{ بار} n} < 1\in \Gamma_0\}$$
  واضح است که اگر ساختار
  $\mathfrak{R}$
  را با ثابت
  $\frac{1}{N+1}$ به عنوان تعبیر $c$ گسترش بدهیم مدلی از $\Gamma_0$ خواهد بود. بنابراین هر زیرمجموعهٔ متناهی از $\Gamma$ مدل دارد. طبق فشردگی $\Gamma$ نیز مدل خواهد داشت. واضح است که مدل $\Gamma$ شروط سؤال را برآورده می‌کند.
\end{ans}