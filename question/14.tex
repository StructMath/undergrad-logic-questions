فرض کنید $\Sigma$ و $\Gamma$ دو مجموعه از فرمول‌ها باشند. گوییم $\Gamma$ و $\Sigma$ هم‌ارزند اگر و تنها اگر مجموعه‌ی فرمول‌های یکسانی را ارضا کنند. همچنین گوییم $\Gamma$ مستقل است اگر برای هر $A\in\Gamma$ داشته باشیم
$$
\Gamma\textbackslash\{A\}\not\models A
$$
گزاره‌های زیر را اثبات یا رد کنید:
\begin{enumerate}
\item
هر مجموعه‌ی متناهی از فرمول‌ها دارای یک زیرمجموعه‌ی مستقل هم‌ارز با خودش است.
\item
هر مجموعه‌ی نامتناهی از فرمول‌ها دارای یک زیرمجموعه‌ی مستقل هم‌ارز با خودش است.
\item[(پ)]
برای هر مجموعه از فرمول‌ها مثل $\Gamma$ یک مجموعه‌ی مستقل از فرمول‌ها مثل $\Sigma$ وجود دارد که با $\Gamma$ هم‌ارز است.
\end{enumerate}\quad\vspace{-9mm}
\begin{ans}
  ابتدا توجّه کنید که می‌توان استقلال را به این صورت هم تعریف کرد: می‌گوییم $\Gamma$ مستقل است اگر $\Gamma$ زیرمجموعه‌ی سره‌ی هم‌ارز با خود نداشته باشد. معادل بودن این تعریف با تعریفی که در صورت سؤال آمده واضح است.
  \begin{enumerate}
    \item
    اگر مجموعه مستقل باشد حکم واضح است. پس حکم را برای مجموعه‌ی متناهی غیر مستقل اثبات می‌کنیم.

    از استقرای قوی روی اندازه‌ی مجموعه استفاده می‌کنیم. فرض کنید $\Gamma$ یک مجموعه‌ی غیر مستقل و حکم برای هر مجموعه‌ی کوچک‌تر از آن برقرار باشد.
    چون $\Gamma$ مستقل نیست، طبق تعریفِ معادل، باید زیرمجموعه‌ی سره‌ی هم‌ارزی مانند $\Delta$ داشته باشد. از آنجا که $\Delta$ کوچک‌تر از $\Gamma$ است، طبق فرض استقرا، زیرمجموعه‌ی مستقل هم‌ارزی مانند $\Delta'$ دارد. واضح است که $\Delta'$ زیرمجموعه‌ی $\Gamma$ و هم‌ارز با آن نیز هست.

    \item
    با مثال نقض رد می‌کنیم.

    مجموعه‌ی زیر را در نظر بگیرید.
    $$ \Gamma = \{p_1,~ p_1 \wedge p_2,~ p_1 \wedge p_2 \wedge p_3, \dots \} $$
    برای هر دو عضو از $\Gamma$، یکی دیگری را نتیجه می‌دهد. هیچ عضوی هم به تنهایی تمام $\Gamma$ را نتیجه نمی‌دهد. بنابراین $\Gamma$ زیرمجموعه‌ی مستقل هم‌ارز ندارد.

    \item
    فرض کنید
    $$ \Gamma = \{ A_n \mid 0 \leq n \} $$
    یک مجموعه‌ی نامتناهی باشد.
    گزاره‌ی $B_n$ و مجموعه‌ی $\Sigma$ را به صورت زیر را درنظر بگیرید.
    $$ B_n = (\bigwedge_{i=0}^{n-1} A_i) \rightarrow A_n $$
    $$ \Sigma = \{ B_n \mid 0 \leq n \} $$

    نشان می‌دهیم $\Gamma$ و $\Sigma$ هم‌ارز هستند.

    اعضای $\Sigma$ شرطی هستند و از تعریف ارزشدهی می‌دانیم برای درست بودن شرطی در یک ارزشدهی کافی است تالی آن شرطی در آن ارزشدهی درست باشد. حال ارزشدهی دلخواهی را فرض کنید که همه‌ی اعضای $\Gamma$ در آن درست هستند. واضح است که تالیِ همه‌ی اعضای $\Sigma$، و در نتیجه همه‌ی اعضای $\Sigma$ در آن تابع ارزش درست هستند. در نتیجه $\Gamma \models \Sigma$.

    برای اثبات $\Sigma \models \Gamma$، ابتدا
    مجموعه‌های $\Sigma_n$ و $\Gamma_n$ را به صورت زیر تعریف می‌کنیم.
    $$ \Sigma_n = \{ B_k \mid 0 \leq k \leq n \} $$
    $$ \Gamma_n = \{ A_k \mid 0 \leq k \leq n \} $$

    با استقرا روی $n$ نشان می‌دهیم برای هر $n$، $\Sigma_n \models \Gamma_n$.
    اگر $n=0$، واضح است که $\{A_0\} \models \{A_0\}$. فرض کنید $n=m+1$ و داشته باشیم $\Sigma_m \models \Gamma_m$. فرض کنید $v$ یک ارزشدهی دلخواه باشد که همه‌ی اعضای $\Sigma_{m+1} = \Sigma_m \cup \{ B_{m+1} \}$ در آن درست باشند. از فرض استقرا می‌دانیم همه‌ی اعضای $\Gamma_n$، یعنی $A_0$،...،$A_m$ هم در $v$ درست هستند. به عبارت دیگر مقدم $B_{m+1}$، و در نتیجه تالی آن، یعنی $A_{m+1}$ نیز در $v$ درست است. پس $\Gamma_{m+1}$ در $v$ درست است. بنابراین برای هر $n$ داریم $\Sigma_n \models \Gamma_n$. برای نشان دادن $\Sigma \models \Gamma$، تابع ارزش $v$ را طوری در نظر بگیرید که همه‌ی اعضای $\Sigma$ در آن درست باشند. فرض کنید $A_n$ عضو دلخواهی از $\Gamma$ باشد. از آن‌جا که همه‌ی اعضای $\Sigma$، و به طور خاص اعضای $\Sigma_n$ در $v$ درست هستند، طبق نتیجه‌ی قبل اعضای $\Gamma_n$، از جمله $A_n$ هم در $v$ درست هستند.

    اکنون $\tilde{\Sigma}$ را طوری تعریف کنید که شامل همه‌ی اعضای $\Sigma$ به جز توتولوژی‌های آن باشد. بدیهی است که $\tilde{\Sigma}$ با $\Sigma$، و در نتیجه با $\Gamma$ هم‌ارز است. حال کافی است نشان دهیم $\tilde{\Sigma}$ مستقل است.

    فرمول دلخواهی از $\tilde{\Sigma}$ مانند $B_n$ را در نظر بگیرید. از آنجا که $B_n$ توتولوژی نیست، پس باید یک ارزشدهی مانند $v$ موجود باشد که $v(B_n)$ نادرست است. در نتیجه، بنا به تعریف ارزشدهی روی شرطی، $v(A_n)$ نادرست و $v(\bigwedge_{i=0}^{n-1} A_i)$ درست است. از نادرست بودن $v(A_n)$ داریم برای همه‌ی $i$های بزرگ‌تر از $n$، مقدم $B_i$ در $v$ نادرست، و در نتیجه $B_i$ درست است. همچنین از درست بودن $v(\bigwedge_{i=0}^{n-1} A_i)$ داریم برای همه‌ی $i$های کوچک‌تر از $n$، تالی $B_i$ در $v$ درست، و درنتیجه $v(B_i)$ درست است. بنابراین برای هر $i$ که $i \neq n$، $v(B_i)$ درست است. پس $\tilde{\Sigma}\textbackslash\{B_n\} \not\models B_n$.

  \end{enumerate}
\end{ans}