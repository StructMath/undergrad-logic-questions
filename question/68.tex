عبارت زیر را در زبان مرتبه اول مناسبی بیان کنید.
	\begin{center}
		دل‌فریبان نباتی همه زیور بستند\\ دلبر ماست که با حُسن خداداد آمد
	\end{center}
	
	\quad\vspace{0.5 cm}
	\begin{ans}
		\\جناب حافظ می‌فرماید که همه دل‌فریبان عالم به جز معشوق ایشان به جلوه زیورآلات دل‌فریب هستند.
	$$
		\mathcal{M} := \{\{\text{انسان‌ها}\}, \{P_1(x), P_2(x), P_3(x)\}, \emptyset, \{\text{دلبر ما}\}\}
	$$
	$$
		{P_1}(x) := \text{x دل‌فریب نباتی است}
	$$
	$$
		{P_2}(x) := \text{x حسن خداداد دارد}
	$$
	$$
		{P_3}(x) := \text{x زیور میبندد}
	$$
	$$
		\forall x \Bigg(\bigg({P_1}(x) \rightarrow {P_3}(x)\bigg) \wedge \bigg(\Big({P_1}(x) \wedge \neg {P_3}(x)\Big) \leftrightarrow \Big(x = \text{دلبر ما} \Big) \bigg)\Bigg)
	$$
	\end{ans}
