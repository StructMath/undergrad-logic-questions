ثابت کنید مجموعه‌ی تمام عبارات زبان منطق گزاره‌ها با مجموعه‌ی اتم‌های شمارا، شمارا است.
\begin{ans}
  مطابق قضیه‌ی کانتور-شرودر-برنشتاین کافی است تابعی یک‌به‌یک از $\mathbb{N}$ به توی مجموعه‌ی فرمول‌ها و تابعی یک‌به‌یک از مجموعه‌ی فرمول‌ها به توی $\mathbb{N}$ معرفی کنیم. برای تابع اول این تابع را در نظر بگیرید:
  $$
  f(n)=p_n
  $$
  یک‌به‌یک بودن  این تابع واضح است. برای تابع دوم این تابع را در نظر بگیرید (چنین تابعی را یک \emph{عددگذاری گودلی} می‌نامیم):
  $$
  g(A)=
  \begin{cases}
  2\cdot 3^{n+1} & \text{if}~~A = p_n\\
  2^2\cdot 3^{g(A_1)} & \text{if}~~A=(\neg A_1)\\
  2^3\cdot 3^{g(A_1)}\cdot 5^{g(A_2)} & \text{if}~~A=(A_1\wedge A_2) \\
  2^4\cdot 3^{g(A_1)}\cdot 5^{g(A_2)} & \text{if}~~A=(A_1\vee A_2) \\
  2^5\cdot 3^{g(A_1)}\cdot 5^{g(A_2)} & \text{if}~~A=(A_1\rightarrow A_2)
  \end{cases}
  $$
  با استفاده از قضیه‌ی اساسی حساب می‌توان نشان داد تنها در صورتی $g(A)=g(B)$ که $A=B$ و بنابراین $g$ یک‌به‌یک است.  
\end{ans}