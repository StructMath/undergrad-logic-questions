ثابت کنید مجموعهٔ تمام عبارات زبان منطق گزاره‌ها با مجموعه‌ی اتم‌های شمارا، شمارا است.
\begin{ans}
  برای اثبات شمارا بودن مجموعهٔ تمام عبارات زبان منطق گزاره‌ها با مجموعهٔ اتم‌های شمارا کافی است تابعی دوسویی میان این مجموعه و مجموعهٔ اعداد طبیعی معرفی کنیم. برای هر عدد طبیعی 
  $n>0$
  مجموعهٔ
  $P_n$
  را به عنوان مجموعهٔ فرمول‌هایی با طول حداکثر $n$ کاراکتر تعریف می‌کنیم که تنها از اتم‌های
  $p_0\cdots p_n$
  استفاده می‌کنند. واضح است که برای هر $n$ مجموعهٔ $P_n$ متناهی است. همچنین واضح است که هر فرمول در یک $P_n$ قرار می‌گیرد.
  حال ابتدا مطابق ترتیبی دلخواه اعضای مجموعهٔ
  $P_{10}$
  را پشت سر هم مرتب می‌کنیم و به هر یک از آن‌ها یک عدد طبیعی از صفر تا $|P_{10}|-1$ نسبت می‌دهیم. سپس همین کار را برای اعضای $P_{100}/P_{10}$ انجام می‌دهیم و به آن‌ها اعداد $|P_{10}|$ تا $|P_{100}|-1$ را نسبت می‌دهیم. بعد برای $P_{1000}/P_{100}$ به همین فرایند ادامه می‌دهیم و همینطور الی آخر. با ادامهٔ همین روند به هر فرمول گزاره‌ای عددی یکتا نسبت می‌دهیم. واضح است که تابعی که به این شیوه ساخته می‌شود هم یک‌به‌یک است و هم پوشا.
  %%%%%% اثبات زیر مبتنی بر قضیهٔ شرودر-برنشتاین است. اثبات بالا تابعی دوسویی را مصرحاً معرفی می‌کند.
  % مطابق قضیه‌ی کانتور-شرودر-برنشتاین کافی است تابعی یک‌به‌یک از $\mathbb{N}$ به توی مجموعه‌ی فرمول‌ها و تابعی یک‌به‌یک از مجموعه‌ی فرمول‌ها به توی $\mathbb{N}$ معرفی کنیم. برای تابع اول این تابع را در نظر بگیرید:
  % $$
  % f(n)=p_n
  % $$
  % یک‌به‌یک بودن  این تابع واضح است. برای تابع دوم این تابع را در نظر بگیرید (چنین تابعی را یک \emph{عددگذاری گودلی} می‌نامیم):
  % $$
  % g(A)=
  % \begin{cases}
  % 2\cdot 3^{n+1} & \text{if}~~A = p_n\\
  % 2^2\cdot 3^{g(A_1)} & \text{if}~~A=(\neg A_1)\\
  % 2^3\cdot 3^{g(A_1)}\cdot 5^{g(A_2)} & \text{if}~~A=(A_1\wedge A_2) \\
  % 2^4\cdot 3^{g(A_1)}\cdot 5^{g(A_2)} & \text{if}~~A=(A_1\vee A_2) \\
  % 2^5\cdot 3^{g(A_1)}\cdot 5^{g(A_2)} & \text{if}~~A=(A_1\rightarrow A_2)
  % \end{cases}
  % $$
  % با استفاده از قضیه‌ی اساسی حساب می‌توان نشان داد تنها در صورتی $g(A)=g(B)$ که $A=B$ و بنابراین $g$ یک‌به‌یک است.  
\end{ans}
