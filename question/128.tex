~\marginpar[left]{\textbf{(۲۰ نمره)}}
\emph{(پرسش ویژه)} \;
رابطهٔ دوتایی $\preceq$ را روی $\mathbb{N} \cup (\mathbb{Q} \times \mathbb{Z})$ به صورت یک ترتیب خطّی تعریف کنید که در آن ابتدا تمام اعداد طبیعی با ترتیب معمول، و سپس زوج مرتّب‌های اعداد گویا و صحیح با ترتیب لغت‌نامه‌ای مرتّب شده‌اند. به بیان دقیق‌تر،
\begin{align*}
  \preceq & = \quad \{ \langle n, m \rangle \mid n, m \in \mathbb{N}, n \le_\mathbb{N} m \} \\
  \quad & \quad \cup \{ \langle n, x \rangle \mid n \in \mathbb{N}, x \in \mathbb{Q} \times \mathbb{Z} \} \\
  \quad & \quad \cup \{ \langle \langle q, a \rangle, \langle r, b \rangle \rangle \mid q, r \in \mathbb{Q}, a, b \in \mathbb{Z}, q \neq r, q \le_\mathbb{Q} r \} \\
  \quad & \quad \cup \{ \langle \langle q, a \rangle, \langle q, b \rangle \rangle \mid q \in \mathbb{Q}, a, b \in \mathbb{Z}, a \le_\mathbb{Z} b \}
\end{align*}
نشان دهید ترتیب هر مدل نااستاندارد شمارای حساب اعداد طبیعی با ساختار $\langle \mathbb{N} \cup (\mathbb{Q} \times \mathbb{Z}), \preceq \rangle$ یک‌ریخت است.