~\marginpar[left]{\textbf{(۲۰ نمره)}}
\emph{(پرسش ویژه)} \;
رابطهٔ دوتایی $\preceq$ را روی $\mathbb{N} \cup (\mathbb{Q} \times \mathbb{Z})$ به صورت یک ترتیب خطّی تعریف کنید که در آن ابتدا تمام اعداد طبیعی با ترتیب معمول، و سپس زوج مرتّب‌های اعداد گویا و صحیح با ترتیب لغت‌نامه‌ای مرتّب شده‌اند. به بیان دقیق‌تر،
\begin{align*}
  \preceq & = \quad \{ \langle n, m \rangle \mid n, m \in \mathbb{N}, n \le_\mathbb{N} m \} \\
  \quad & \quad \cup \{ \langle n, x \rangle \mid n \in \mathbb{N}, x \in \mathbb{Q} \times \mathbb{Z} \} \\
  \quad & \quad \cup \{ \langle \langle q, a \rangle, \langle r, b \rangle \rangle \mid q, r \in \mathbb{Q}, a, b \in \mathbb{Z}, q \neq r, q \le_\mathbb{Q} r \} \\
  \quad & \quad \cup \{ \langle \langle q, a \rangle, \langle q, b \rangle \rangle \mid q \in \mathbb{Q}, a, b \in \mathbb{Z}, a \le_\mathbb{Z} b \}
\end{align*}
نشان دهید ترتیب هر مدل نااستاندارد شمارای حساب اعداد طبیعی با ساختار $\langle \mathbb{N} \cup (\mathbb{Q} \times \mathbb{Z}), \preceq \rangle$ یک‌ریخت است.
\begin{ans}
  فرض کنید $\mathfrak{N}'$ یک مدل نااستاندارد حساب باشد. می‌دانیم در ابتدای $\mathfrak{N}'$ یک نسخه از $\mathfrak{N}$ وجود دارد. بنابراین کافی است اثبات کنیم ترتیب بخش نااستانداردْ یک‌ریخت با $\mathbb{Q} \times \mathbb{Z}$ با ترتیب لغت‌نامه‌ای است. به جملات زیر توجه کنید:
  \begin{align*}
    & \sigma_1 := \forall y \exists x ((y + 1 = x) \wedge \neg (x = y)) \\
    & \sigma_2 := \forall y (\neg (y = 0) \to \exists x ((x + 1 = y) \wedge \neg (x = y))) \\
    & \sigma_3 := \forall y \neg \exists x (y < x \wedge x < y + 1) \\
    & \sigma_4 := \forall y \exists x ((y = (1 + 1) \cdot x) \vee (y + 1 = (1 + 1) \cdot x))
  \end{align*}
  این چهار جمله در $\mathfrak{N}$ و در نتیجه در $\mathfrak{N}'$ صادق هستند.

  حال عضو نااستاندارد $a_0$ در $\mathfrak{N}$ را در نظر بگیرید. طبق $\sigma_1$، دو عضو متمایز $a_1$ و $a_{-1}$ در $\mathfrak{N}'$ وجود دارند که قبل و بعد از $a_0$ قرار دارند. همچنین از $\sigma_2$ می‌دانیم میان $a_0$ و $a_1$ و میان $a_0$ و $a_{-1}$ هیچ عنصری واقع نشده است. همچنین $a_1$ و $a_{-1}$ نااستاندارد هستند، زیرا در غیر این صورت $a_0$ نیز استاندارد می‌بود. با تکرار همین استدلال می‌توان برای هر $n \in \mathbb{Z}$ عضو نااستاندارد $a_n$ را ساخت. بنابراین هر عضو نااستاندارد داخل بلوکی همریخت با $\langle \mathbb{Z}, \le_\mathbb{Z} \rangle$ قرار دارد.

  اکنون فرض کنید $a_0$ و $b_0$ دو عضو نااستاندارد باشند که در دو بلوک مجزّا قرار دارند. فرض کنید $a_n$ و $b_m$ هم دو عضو هم‌بلوک با $a_0$ و $b_0$ باشند. با توجه به هم‌بلوک بودن $a_0$ و $a_n$ عدد صحیح $n$ وجود دارد که اگر نامنفی باشد داریم $a_n = a_0 + \overline{n}$ و اگر منفی باشد داریم $a_n + \overline{(-n)} = a_0$. به طور مشابه عدد صحیح $m$ وجود دارد. ادعا می‌کنیم که اگر $a_0 < b_0$ آن‌گاه $a_n < b_n$. با توجه به تعریف ترتیب، عدد نامنفی $c$ وجود دارد که $a_0 + c = b_0$. فرض کنید $m - n$ نامنفی باشد. در این صورت $a_n + c + \overline{m - n} = b_m$. اگر $m - n$ منفی باشد، توجه کنید $c$ نااستاندارد است، پس $c_{m - n}$ وجود دارد که $a_n + c = b_m$ و $c_{m - n} + \overline{m - n} = c$. بنابراین $a_n < b_m$. می‌توان نتیجه گرفت ترتیب روی بخش نااستاندارد $\mathfrak{N}'$ ترتیبی روی بلوک‌ها القاء می‌کند.

  اکنون کافی است اثبات کنیم ترتیب القایی روی بلوک‌ها یک‌ریخت با $\langle \mathbb{Q}, \le_\mathbb{Q} \rangle$ است. می‌دانیم ترتیب چگال خطّی بدون نقاط پایانی $\aleph_0$-جازم است. خطّی یودن ترتیب روی بلوک‌ها به آسانی از خطّی بودن ترتیب $\mathfrak{N}'$ نتیجه می‌شود. اگر $a$ یک عضو نااستاندارد باشد، $a + a$ عضو نااستاندارد دیگری است که در بلوک متفاوتی قرار دارد و از $a$ برزگتر است. همچنین از آن‌جا که $\sigma_4$ در $\mathfrak{N}'$ صادق است، برای هر عضو نااستاندارد مانند $a$ می‌توان عضو نااستاندارد کوچکتری از آن مثل $b$ پیدا کرد که $a = b + b$ یا $a + 1 = b + b$. روشن است که $b$ در بلوک متفاوتی از $a$ قرار دارد. بنابراین ترتیب روی بلوک‌ها نقطهٔ پایانی ندارد.

  برای اثبات چگال بودن، توجه کنید که اگر $a < b$ دو عضو نااستاندارد از دو بلوک متفاوت باشند، عضو نااستاندارد $c$ وجود دارد که $a + c = b$. با استدلال مشابه می‌توان نشان داد $c'$ نااستاندارد وجود دارد که $c = c' + c'$ یا $c + 1 = c' + c'$. می‌توان نشان داد که $a + c'$ در بلوک متفاوتی از $a$ و $b$ قرار دارد. در نتیجه ترتیب روی بلوک‌ها یک ترتیب خطّی چگال بدون نقاط پایانی است. از آن‌جا که تعداد بلوک‌ها $\aleph_0$ است، جکم ثابت شده است.
\end{ans}