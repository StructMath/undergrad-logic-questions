~\marginpar[left]{\textbf{(۲۰ نمره)}}
\emph{(پرسش ویژه)} \;
رابطهٔ دوتایی $\preceq$ را روی $\mathbb{N} \cup (\mathbb{Z} \times \mathbb{Q})$ به صورت یک ترتیب خطّی تعریف کنید که در آن ابتدا تمام اعداد طبیعی با ترتیب معمول، و سپس زوج مرتّب‌های اعداد صحیح و گویا با ترتیب لغت‌نامه‌ای مرتّب شده‌اند. به بیان دقیق‌تر،
\begin{align*}
  \preceq & = \quad \{ \langle n, m \rangle \mid n, m \in \mathbb{N}, n \le_\mathbb{N} m \} \\
  \quad & \quad \cup \{ \langle n, x \rangle \mid n \in \mathbb{N}, x \in \mathbb{Z} \times \mathbb{Q} \} \\
  \quad & \quad \cup \{ \langle \langle a, r \rangle, \langle b, q \rangle \rangle \mid a, b \in \mathbb{Z}, r, q \in \mathbb{Q}, a \le_\mathbb{Z} b \} \\
  \quad & \quad \cup \{ \langle \langle a, r \rangle, \langle a, q \rangle \rangle \mid a \in \mathbb{Z}, r, q \in \mathbb{Q}, r \le_\mathbb{Q} q \}
\end{align*}
نشان دهید ترتیب هر مدل نااستاندارد شمارای حساب اعداد طبیعی با ساختار $\langle \mathbb{N} \cup (\mathbb{Z} \times \mathbb{Q}), \preceq \rangle$ یک‌ریخت است.