فرض کنید $\models A\rightarrow B$ و $A$ و $B$ دارای اتم‌های مشترک نیستند. ثابت کنید یا $A$ ارضاناپذیر است یا $B$ توتولوژی است (یا هر دو). توضیح بدهید که آیا شرط اتم مشترک نداشتن برای این حکم ضروری است یا نه.
\begin{ans}
  با استفاده از قضیه‌ی استنتاج می‌توانیم نتیجه بگیریم $A\models B$. بنابراین هر ارزیاب $v$ که $A$ را ارضا کند، $B$ را نیز ارضا می‌کند. برای اثبات حکم کافی است فرض کنیم $A$ ارضاناپذیر نیست و نتیجه بگیریم $B$ توتولوژی است. اگر $A$ ارضاناپذیر نباشد، ارزیاب $v$ای وجود دارد که $A$ را ارضا می‌کند. حال یک ارزیاب دلخواه $u$ را در نظر می‌گیریم و نشان می‌دهیم $B$ را ارضا می‌کند. ارزیاب $v'$ را به شکل زیر تعریف می‌کنیم:

  $$
  v'(p_n)=
  \begin{cases}
  u(p_n) & \text{if}~~p_n\in atoms(B) \\
  v(p_n) & \text{otherwise}
  \end{cases}
  $$

  حال واضح است که $v'$ نیز همانند $v$ فرمول $A$ را ارضا می‌کند زیرا مقدار این دو ارزیاب در اتم‌های موجود در $A$ یکسان است و در جزوه ثابت کرده‌ایم اگر دو ارزیاب مقادیر یکسانی به اتم‌های داخل یک فرمول نسبت بدهند، به فرمول نیز مقدار یکسانی نسبت می‌دهند. بنابراین از آنجا که $v'$ فرمول $A$ را ارضا می‌کند فرمول $B$ را نیز ارضا می‌کند. نیز، از آنجا که مقداری که ارزیاب $v'$ به اتم‌های $B$ نسبت می‌دهد همان مقداری است که ارزیاب $u$ به آن‌ها نسبت می‌دهد، مقداری که این دو ارزیاب به $B$ نسبت می‌دهند یکسان است. بنابراین ارزیاب $u$ نیز $B$ را ارضا می‌کند. با توجه به اینکه ارزیاب $u$ دلخواه است، هر ارزیابی $B$ را ارضا می‌کند و بنابراین $B$ توتولوژی است.

  در صورتی که شرط مشترک نبودن اتم‌ها را حذف کنیم حکم برقرار نیست؛ مثلاً
  $\models p\rightarrow p$
  اما $p$ نه ارضاناپذیر است و نه توتولوژی.
\end{ans}
