تبلیغ یک مجلهٔ تنیس می گوید «اگر من در حال بازی کردن تنیس نیستم، آنگاه در حال تماشای تنیس هستم.» و «اگر در حال تماشای تنیس نیستم، در حال مطالعه درباره تنیس هستم.» اگر فرض کنیم گوینده نمی‌تواند دو کار را همزمان انجام دهد، گوینده مشغول چه کاری است؟ مفروضات را در منطق گزاره‌ای صورت‌بندی کنید و به سؤال پاسخ بدهید.
\begin{ans}

	فرض کنیم 
    $p$
    معادل با گزاره (گوینده در حال تنیس بازی کردن است) و
    $q$
    معادل با (گوینده در حال تماشای تنیس است) و
    $s$
    معادل با (گوینده در حال مطالعه درباره تنیس است باشد.)باشد. پس گزاره های صادق ما 
    $((\neg p) \leftarrow q)$
    و 
    $((\neg q) \leftarrow s)$
    هستند. پس باید تابع ارزیابی را بیابیم که فقط یکی از سه اتم 
    $p$
    یا 
    $q$
    یا 
    $s$
    را صادق ببیند و هر دو گزاره صادق ما را نیز صادق ببیند. 
    تابع ارزیاب 
    $\nu$
    را به این شکل تعریف کنید:
    \[\nu(p) = \bot\]
    \[\nu(q) = \top\]
    \[\nu(s) = \bot\]
    تابع 
    $\nu$
    تنها یک اتم 
    $q$
    ارضا می کند و هر دو گزاره شرطی ما را نیز صادق می بیند. پس گوینده درحال تماشای تنیس است.لازم به ذکر است اگر تابع 
    $\nu$
    اتمی غیر از اتم
    $q$
    را صادق ببیند. نمی تواند هر دو گزاره شرطی مفروض ما را صادق ببیند.
    

\end{ans}
