تبلیغ یک مجلهٔ تنیس می گوید «اگر من در حال بازی کردن تنیس نیستم، آنگاه در حال تماشای تنیس هستم.» و «اگر در حال تماشای تنیس نیستم، در حال مطالعه درباره تنیس هستم.» اگر فرض کنیم گوینده نمی‌تواند دو کار را همزمان انجام دهد، گوینده مشغول چه کاری است؟ مفروضات را در منطق گزاره‌ای صورت‌بندی کنید و به سؤال پاسخ بدهید.
\begin{ans}
	فرض کنیم
    $p$
    معادل با گزارهٔ
    «گوینده در حال تنیس بازی کردن است»
    و
    $q$
    معادل با گزارهٔ
    «گوینده در حال تماشای تنیس است»
    و
    $s$
    معادل با
    «گوینده در حال مطالعه دربارهٔ تنیس است»
    باشد. بنابراین اطلاعات مسئله را در این فرمول می‌توان صورت‌بندی کرد:
    $$(\neg p \to q)\wedge(\neg q \to s)$$
    همچنین گفته شده گوینده نمی‌تواند دو کار را به طور همزمان انجام بدهید که این را می‌توان با این فرمول صورت‌بندی کرد:
    $$(p \to (\neg q \wedge \neg s)) \wedge (q \to (\neg p \wedge \neg s)) \wedge (s \to (\neg p \wedge \neg q))$$
    واضح است که تنها چهار ارزیاب وجود دارند که فرمول دوم را صادق ببینند:
    \begin{alignat*}{3}
        v_1(p) = 0 &\qquad v_1(q) = 0 &\qquad v_1(s) = 0 \\
        v_2(p) = 1 &\qquad v_2(q) = 0 &\qquad v_2(s) = 0 \\
        v_3(p) = 0 &\qquad v_3(q) = 1 &\qquad v_3(s) = 0 \\
        v_4(p) = 0 &\qquad v_4(q) = 0 &\qquad v_4(s) = 1
    \end{alignat*}
    حال با ترسیم جدول ارزش مشخص می‌شود که تنها
    $v_3$
    فرمول اول را صادق ارزیابی می‌کند. 
\end{ans}
