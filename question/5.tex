
نشان دهید مجموعهٔ اعداد طبیعی و مجموعهٔ اعداد صحیح هم‌عدد هستند.
\begin{ans}
    تابع 
    $f$
    را اینگونه تعریف می کنیم:

    $f:N\xrightarrow{} Z$
    \[
    f(x) = 
    \begin{cases}
        x & \textbf{if} \ x = 0\\
        (x+1)/2 & \textbf{if} \ \ x \ is \ odd\\
        -x/2 & \textbf{if} \ \ x \ is \ even\\
    \end{cases}
    \]
    حال نشان می دهیم که 
    $f$
    یک به یک و پوشاست.
    \\
    \textbf{اثبات یک به یک بودن:}\\
    فرض کنیم 
    $x$
    و
    $y$
    دو عدد طبیعی هستند. و 
    $f(x) = f(y)$
    نشان می دهیم که 
    $x = y$.
    \\
    اگر 
    $f(x) = 0$
    آنگاه لزوما
    $x = y = 0$
    زیرا تنها اعدادی که توسط تابع 
    $f$
    به 
    $0$
    نظیر می شوند خود عدد 
    $0$
    است.
    \\
    حال فرض کنیم 
    $x$
    و
    $y$
    هر دو زوج یا هر دو فرد هستند. در اینصورت با توجه به این که ضابطه 
    $f$
    برای اعداد زوج و فرد هر دو خطی و در نتیجه یک به یک هستند، حکم در این حالت نیز برقرار خواهد بود و داریم:
    $x = y$.
    \\
    حالا کافی است نشان دهیم که اگر یکی از دو عدد زوج و یکی فرد باشد، خروجی تابع برای این دو برابر نخواهد بود. فرض کنیم 
    
    $x$
    زوج و
    $y$
    فرد است. آنگاه طبق فرض داریم:

    $(-x/2) = (y+1)/2$

    از این تساوی نتیجه می شود:\\
    $-x = y+1$
    \\
    که می رسیم به
    \\
    $y+x = -1$
    \\
    که یک تناقض است زیرا
    $x$
    و
    $y$
    هر دو اعداد طبیعی هستند و حاصل جمع آن ها نمی تواند یک عدد منفی باشد.
    \\
    \textbf{اثبات پوشا بودن}:\\

    برای نشان دادن پوشا بودن تابع 
    $f$
    باید نشان دهیم به ازای عدد صحیح دلخواه 
    $z$
    عدد طبیعی
    $x$
    وجود دارد که 
    $f(x) = z$
    \\
    ما در اینجا با حالت بندی پوشا بودن 
    $f$
    را نشان می دهیم:
    \\
    \textbf{حالت اول: } 
    $z = 0$
    \\
    در این حالت اگر
    $x = 0$
    آنگاه
    $f(x) = z$
    \\
    \textbf{حالت دوم:}
    $z > 0$
    \\
    در این حالت قرار می دهیم 
    $x = 2z - 1$
    \\
    آنگاه خواهیم داشت:
    \[f(x) = f(2z-1) = ((2z-1)+1)/2 = z\]
    \\
    \textbf{حالت سوم:}
    $z < 0$
    \\
    در این حالت قرار می دهیم
    $x = -2z$
    \\
    و مجدد
    \[f(x) = f(-2z) = z\]
    در نتیجه
    $f$
    پوشاست 

\end{ans}
