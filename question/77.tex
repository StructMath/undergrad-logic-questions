\begin{enumerate}[label=(\alph*)]
    \item فرض کنید $\Gamma$ زیرمجموعه‌ای از گزاره‌ها و $\varphi$ یک گزاره باشد به گونه‌ای که $\Gamma \not\vdashHP \varphi$. نشان دهید مجموعهٔ ماکسیمال $\Gamma^*$ از گزاره‌ها موجود است به گونه‌ای که $\Gamma \subseteq \Gamma^*$ و $\Gamma^* \not\vdashHP \varphi$. (ماکسیمال بودن $\Gamma^*$ به این معناست که برای هر گزارهٔ $\psi \not\in \Gamma^*$ داریم $\Gamma^*,\psi \vdashHP \varphi$)
    \item به عنوان حالت خاصی از حکم بخش قبل، نشان دهید اگر $\Gamma$ مجموعه‌ای سازگار از گزاره‌ها باشد، مجموعهٔ ماکسیمال $\Gamma^*$ موجود است که $\Gamma \in \Gamma^*$ و $\Gamma^*$ سازگار است.
    \item با استفاده از نتیجهٔ بخش قبل، لم وجود مدل را برای دستگاه استنتاج هیلبرت ثابت کنید. (لم وجود مدل برای دستگاه استنتاج هیلبرت بیان می‌دارد که برای هر زیرمجموعهٔ سازگار در این دستگاه، یک مدل وجود دارد.)
\end{enumerate}
% \quad\vspace{-0.5}
% \begin{ans}
%     \begin{enumerate}[label=(\alph*)]
%         \item
%         مجموعهٔ زیر را در نظر بگیرید:
%         $$A = \{\Sigma | \Gamma \in \Sigma, \Sigma \not\vdashHP \varphi\}$$
%         داریم $\Gamma \in A$ و در نتیجه $A \neq \varnothing$. حال نشان می‌دهیم $A$ در شرایط لم زرن صدق می‌کند.\\
%         فرض کنید $\Sigma_1 \subseteq \Sigma_2 \subseteq \dots$ زنجیری صعودی از اعضای $A$ باشد. نشان می‌دهیم $\Sigma := \bigcup_{i\in\mathbb{N}} \Sigma_i \in A$ و در نتیجه زنجیر در $A$ کران‌دار است.\\
%         چون $\Gamma \subseteq \Sigma_1 \subseteq \Sigma$، مجموعهٔ تعریف شده در شرط اول ورود به $A$ صدق می‌کند.\\
%         حال نشان می‌دهیم $\Sigma \not\vdashHP \varphi$. فرض کنید چنین نباشد و $\Sigma \vdashHP \varphi$. یک برهان برای $\varphi$ با مفروضات $\Sigma$ در نظر بگیرید. چون طول برهان متناهی است، فقط از متناهی عضو $\Sigma$ در این برهان استفاده شده. این متناهی عضو را $\psi_1, \psi_2, \dots, \psi_n$ بنامید. با توجه به تعریف $\Sigma$، هر یک از $\psi_i$ها در یکی از $\Sigma_i$ها آمده است. فرض کنید $\psi_i$ در $\Sigma_{f(i)}$ آمده باشد ($f(i)$ را بین مجموعه‌های شامل $\psi_i$ به دل‌خواه انتخاب کنید.). قرار دهید $k := \max_{i=1}^n(f(i))$. چون تعداد $\psi_i$ها متناهی است گرفتن عضو بیشینه بامعنی است. واضح است که تمام $\psi_i$ها عضو $\Sigma_k$ هستند. بنابراین همان برهانی که با مفروضات $\Sigma$ برای $\varphi$ وجود داشت، با مفروضات $\Sigma_k$ نیز معتبر است. بنابراین $\Sigma_k \vdashHP \varphi$. اما این موضوع با عضویت $\Sigma_k$ در $A$ متناقض است. بنابراین فرض خلف باطل بوده و $\Sigma \not\vdashHP \varphi$.\\
%         بنابراین $\Sigma$ در شرط دوم ورود به $A$ نیز صدق می‌کند و در نتیجه $\Sigma \in A$.\\
%         پس ثابت شد $A$ در شرایط لم زرن صدق می‌کند. پس طبق لم زرن $A$ عضوی ماکسیمال دارد. این عضو ماکسیمال را $\Gamma^*$ بنامید. واضح است که $\Gamma^*$ پیدا شده، در تمام ویژگی‌های مورد نظر صدق می‌کند.

%         \item
%         کافی است در گزارهٔ بخش قبل قرار دهید $\varphi = \bot$ و حکم ثابت می‌شود.

%         \item 
%         فرض کنید $\Gamma$ مجموعه‌ای سازگار از گزاره‌ها باشد. مطابق نتیجهٔ بخش قبل، $\Gamma$ را به یک مجموعهٔ سازگار ماکسیمال مانند $\Gamma^*$ گسترش دهید. حال ارزیاب $v$ را به صورت زیر تعریف کنید:
%         $$v(p) =
%         \begin{cases}
%             1 &\quad p \in \Gamma^*\\
%             0 &\quad \text{otherwise}
%         \end{cases}
%         $$
%         نشان می‌دهیم $v$ یک مدل برای $\Gamma^*$ و در نتیجه (چون $\Gamma \subseteq \Gamma^*$) برای $\Gamma$ است. برای این منظور باید ثابت کنیم برای هر $\varphi \in \Gamma^*$، $v(\varphi) = 1$. ضمنا به عنوان گزاره‌ای کمکی، نشان می‌دهیم عکس این موضوع نیز صادق است. یعنی برای هر $\varphi \not\in \Gamma^*$، $v(\varphi) = 0$. حکم را به استقرا روی پیچیدگی $\varphi$ ثابت می‌کنیم.\\
%         اگر $\varphi = p$ به گونه‌ای که $p$ اتم باشد، مطابق نحوهٔ ساخت $v$ حکم واضح است.\\
%         اگر $\varphi = \neg\psi$، در صورتی که $\varphi \in \Gamma^*$، واضح است که $\psi$ عضو $\Gamma^*$ نخواهد بود (چون در غیر این صورت هردوی $\psi$ و $\neg\psi$ عضو $\Gamma^*$ اند که با سازگاری آن در تناقض است). بنابراین طبق فرض استقرا، $v(\psi) = 0$ و در نتیجه $v(\varphi) = 1$.\\
%         در صورتی که $\varphi \not\in \Gamma^*$، باید داشته باشیم $\psi \in \Gamma^*$. چرا که در غیر این صورت، $\Gamma^* \cup \{\psi\}$ مجموعه‌ای سازگار خواهد بود که $\Gamma^*$ زیرمجموعه‌ی سرهٔ آن است که این موضوع با ماکسیمال بود $\Gamma^*$ متناقض است. پس طبق فرض استقرا، $v(\psi) = 1$ و در نتیجه $v(\varphi) = 0$.\\
%         اگر $\varphi = \psi \vee \chi$، در صورتی که $\varphi \in \Gamma^*$، نشان می‌دهیم حداقل یکی از $\psi$ و $\chi$ نیز عضو $\Gamma^*$ اند. فرض کنید چنین نباشد، آن‌گاه بنا به ماکسیمال بودن $\Gamma^*$ باید داشته باشیم:
%         $$\Gamma^*, \psi \vdashHP \bot \text{و} \Gamma^*,\chi \vdashHP \bot$$
%         $$\xRightarrow{قضیهٔ استنتاج} \Gamma^* \vdashHP \psi\rightarrow\bot \text{و} \Gamma^* \vdashHP \chi\rightarrow\bot$$
%         $$\xRightarrow{اصل ۹ دستگاه هیلبرت} \Gamma^* \vdashHP \neg\psi \text{و} \Gamma^* \vdashHP \neg\chi$$
%         اما طبق اصل ۳ دستگاه هیلبرت می‌دانیم $\vdashHP \neg\psi \rightarrow (\neg\chi \rightarrow (\neg\psi \wedge \neg\chi))$. بنابراین با دو بار استفاده از قاعدهٔ استنتاج دستگاه هیلبرت (\LR{MP}) خواهیم داشت:
%         $$\Gamma^* \vdashHP \neg\psi \wedge \neg\chi$$
%         حال نشان می‌دهیم $\Gamma^*, \psi \vee \chi \vdashHP \bot$ که این موضوع، با این نکته که $\psi \vee \chi$ عضو $\Gamma^*$ است و $\Gamma^*$ مجموعه‌ای سازگار است، در تناقض است. داریم:
%         $$\Gamma^*, \psi\vee\chi \vdashHP \bot \xLeftrightarrow{قضیهٔ استنتاج} \Gamma^* \vdashHP (\psi\vee\chi) \rightarrow \bot \xLeftrightarrow{اصل ۹ دستگاه هیلبرت} \Gamma^* \vdashHP \neg(\psi\vee\chi) \Leftrightarrow \Gamma^* \vdashHP \neg\psi \wedge \neg\chi$$
%         و گزارهٔ آخر آن‌چنان که نشان دادیم درست است. بنابراین تناقض مورد نظر حاصل می‌شود. به این ترتیب، حداقل یکی از $\psi$ و $\chi$ عضو $\Gamma^*$ اند. بنابراین طبق فرض استقرا، حداقل یکی از مقادیر $v(\psi)$ و $v(\chi)$ برابر $1$ است و در نتیجه، $v(\varphi) = 1$.\\
%         در صورتی که $\varphi \not\in \Gamma^*$، بر اساس ماکسیمال بودن $\Gamma^*$ باید داشته باشیم $\Gamma^*, \psi \vee \chi \vdashHP \bot$. در این صورت به راحتی و مشابه آن‌چه انجام شد ثابت می‌شود $\psi$ و $\chi$ نیز عضو $\Gamma^*$ نیستند و در نتیجه، هردو مقدار $v(\psi)$ و $v(\chi)$ برابر $0$ اند. در نتیجه $v(\varphi) = 0$ و حکم ثابت می‌شود.\\
%         اثبات حالات $\varphi = \psi \wedge \chi$ و $\varphi \rightarrow \chi$ نیز به طور مشابه انجام می‌شود.\\
%         بدین‌ترتیب ثابت می‌شود $v$ مدلی برای $\Gamma$ است و اثبات کامل می‌شود.
%     \end{enumerate}
% \end{ans}
